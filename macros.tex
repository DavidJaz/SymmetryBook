%% packages
\usepackage{amssymb,amsthm}
\usepackage[numbers]{natbib}
\usepackage{mathtools}          %to get \vcentcolon
\usepackage{thmtools}
\usepackage{letltxmacro}        %to rename \equiv
\LetLtxMacro{\oldequiv}{\equiv}
\usepackage{xspace}
\usepackage[all]{xy}
\usepackage{pgfplots,tikz,tikz-cd}
\pgfplotsset{compat=newest}
%\usepackage{soul} % for striking out, can be removed in the end
\usepackage{enumitem}

\usetikzlibrary{decorations.markings,decorations.pathreplacing,matrix,arrows,chains,positioning,scopes} 
        %,hobby} % for hobby splines
%% useful for debugging bezier paths
%\tikzset{%
%  show curve controls/.style={
%    postaction={
%      decoration={
%        show path construction,
%        curveto code={
%          \draw [blue] 
%            (\tikzinputsegmentfirst) -- (\tikzinputsegmentsupporta)
%            (\tikzinputsegmentlast) -- (\tikzinputsegmentsupportb);
%          \fill [red, opacity=0.5] 
%            (\tikzinputsegmentsupporta) circle [radius=.5ex]
%            (\tikzinputsegmentsupportb) circle [radius=.5ex];
%        }
%      },
%      decorate
%    }}}

% hyperref should be the package loaded last
\usepackage[backref=page,
            colorlinks,
            citecolor=linkcolor,
            linkcolor=linkcolor,
            urlcolor=linkcolor,
            unicode,
            pdfauthor={CAS},
            pdftitle={Symmetry},
            pdfsubject={Mathematics},
            pdfkeywords={type theory, group theory, univalence axiom}]{hyperref}
% - except for cleveref!
\usepackage[capitalize]{cleveref}
\usepackage{xifthen}
\definecolor{linkcolor}{rgb}{0,0,0.5}

%% macros
\newcommand{\mytitle}{Symmetry}
\newcommand{\myauthor}{}

\newcommand{\DELETE}[1]{}

%%% Headers
\pagestyle{headings}

%%%%%%%%%%%%%%%%%%%%%%%%%%%%%%%%%%%%%%%%%%%%%%%%%%%%%%%%%%%%%%%%%%%%%%%%%%%%%
%%% THEOREMS
\declaretheoremstyle[headfont=\normalfont\bfseries,bodyfont=\itshape]{cas-thm}
\declaretheoremstyle[headfont=\normalfont\bfseries]{cas-def}
\declaretheorem[parent=section,style=cas-thm]{theorem}
\declaretheorem[sibling=theorem,style=cas-thm]{lemma}
\declaretheorem[sibling=theorem,style=cas-thm]{corollary}
\declaretheorem[sibling=theorem,style=cas-thm]{conjecture}
\declaretheorem[sibling=theorem,style=cas-thm]{axiom}
\declaretheorem[sibling=theorem,style=cas-thm]{construction}
\declaretheorem[sibling=theorem,style=cas-def]{definition}
\declaretheorem[sibling=theorem,style=cas-def]{remark}
\declaretheorem[sibling=theorem,style=cas-def]{example}
\declaretheorem[sibling=theorem,style=cas-def]{exercise}
\declaretheorem[sibling=theorem,style=cas-def,name=Exercise]{xca}
\declaretheorem[sibling=theorem,style=cas-def,name={}]{principle}

\numberwithin{equation}{section}
%% end
%%%%%%%%%%%%%%%%%%%%%%%%%%%%%%%%%%%%%%%%%%%%%%%%%%%%%%%%%%%%%%%%%%%%%%%%%%%%%

\newcommand{\arxiv}[1]{preprint available at \href{http://arxiv.org/abs/#1}{arXiv:#1}}


% Should these be sans serif, roman or italic?!
% Rule (under discussion): constructors italic, defined elements roman, typeformers sf

\newcommand{\constructor}[1]{\mathit{#1}}
\newcommand{\function}[1]{\mathrm{#1}}
\newcommand{\typeformer}[1]{\mathsf{#1}}

% Typeformers
\newcommand{\bool}{\typeformer{Bool}}


% Constructors

\newcommand{\yes}{\constructor{yes}}
\newcommand{\no}{\constructor{no}}
\newcommand{\refl}[1]{\mathop{\mathit{refl}_{#1}}}

% Functions and defined elements

\let\olddiv\div
\renewcommand{\div}{{\mathord{\scalebox{.5}{$\olddiv$}}}} %experimental

\newcommand{\refloi}[1]{\mathop{\mathit{refl}^{-o}_{#1}}} % exception
\newcommand{\inl}[1]{\mathop{{\it inl}_{#1}}}
\newcommand{\inr}[1]{\mathop{{\it inr}_{#1}}}
\newcommand{\true}{\mathop{\mathit{True}}}
\newcommand{\false}{\mathop{\mathit{False}}}
\newcommand{\triv}{\mathop{\mathit{triv}}}


% Functions and defined elements

\newcommand{\fact}{\mathop{\mathrm{fact}}}
\newcommand{\id}{\mathord{\function{id}}}
\newcommand{\pt}{\function{pt}}
\newcommand{\symm}{\mathop{\mathrm{symm}}}
\newcommand{\trans}{\mathop{\mathrm{trans}}}
\newcommand{\trp}[2][]{\mathop{\mathrm{trp}^{#1}_{#2}}}
\newcommand{\dg}[1]{\mathop{\color{red}\delta_{#1}}}
\newcommand{\fst}{\mathop{\mathrm{fst}}}
\newcommand{\snd}{\mathop{\mathrm{snd}}}
\newcommand{\zpos}{\mathop{\mathrm{pos}}}
\newcommand{\zneg}{\mathop{\mathrm{neg}}}
\newcommand{\zzero}{\mathop{\mathrm{zero}}}
\newcommand{\mZ}{\mathop{\mathrm{m}}}
\newcommand{\preim}{\mathop{\mathrm{preim}}}
\newcommand{\tot}{\mathop{\mathrm{tot}}}
\newcommand{\funext}{\mathop{\mathrm{funext}}}
\newcommand{\ptw}{\mathop{\mathrm{ptw}}}
\newcommand{\ap}[1]{\mathop{\mathrm{ap}_{#1}}}
\newcommand{\apd}[1]{\mathop{\mathrm{apd}_{#1}}}
\newcommand{\apap}[3]{\mathop{\mathrm{apap}_{#1}(#2)(#3)}}
\newcommand{\apc}{\mathop{\mathrm{ap}{\ct}}}






% misc, not yet sorted out

\newcommand{\set}[1]{\{#1\}}

%\renewcommand{\div}{{\mathrm t}} %experimental

\newcommand{\Type}{\mathord{\mathrm{Type}}}
\newcommand{\Prop}{\mathord{\mathrm{Prop}}}
\newcommand{\Set}{\mathord{\mathrm{Set}}}
\newcommand{\GSet}[1][G]{\mathord{#1\text{-}\mathrm{Set}}}
\newcommand{\Group}{\mathord{\mathrm{Group}}}
\newcommand{\AbGroup}{\mathord{\mathrm{AbGroup}}}
\newcommand{\Bunch}{\mathord{\mathrm{Bunch}}}
\newcommand{\AbBunch}{\mathord{\mathrm{AbBunch}}}
\newcommand{\Band}{\mathord{\mathrm{Band}}}
\newcommand{\AbBand}{\mathord{\mathrm{AbBand}}}

\DeclareMathOperator\Bop{B}         % with extra space
\newcommand{\B}{\mathrm B}          % without extra space
\newcommand{\N}{\mathrm N}
\newcommand{\weq}{\simeq}
\newcommand{\QQ}{\mathbb{Q}}
\newcommand{\ZZ}{\mathbb{Z}}
\newcommand{\NN}{\mathbb{N}}
\newcommand{\CC}{\mathbb{C}}
\newcommand{\RR}{\mathbb{R}}
\newcommand{\isom}{\cong}
\newcommand{\ct}{*}
\newcommand{\cto}{*_{\mathrm{o}}}
\newcommand{\rrfl}{\mathit{rrfl}}
\newcommand{\cp}[1]{\mathit{cp}_{#1}}
\newcommand*{\dblslash}{\mathbin{/\kern-3pt/}}

\renewcommand{\equiv}{\simeq}
\newcommand{\liff}{\equiv}
\newcommand{\jdeq}{\oldequiv}
\newcommand{\defeq}{\vcentcolon\jdeq}
\newcommand{\defequi}{\defeq}%definitionally equal}
\newcommand{\defis}{\vcentcolon=}

\DeclareMathOperator\Aut{Aut}
\DeclareMathOperator\Out{Out}
\DeclareMathOperator\Inn{Inn}
\DeclareMathOperator\Ker{Ker}
\DeclareMathOperator\Img{Im}
\DeclareMathOperator\Sym{Sym}
\DeclareMathOperator\Card{Card}
\DeclareMathOperator\bunch{bunch}
\DeclareMathOperator\band{band}
\DeclareMathOperator\fiber{fiber}
\DeclareMathOperator\Succ{Succ}
\DeclareMathOperator\fin{Fin}
\DeclareMathOperator\El{El}
\DeclareMathOperator\clf{B}


\DeclarePairedDelimiter\Trunc{\lVert}{\rVert}
\DeclarePairedDelimiter\trunc{\lvert}{\rvert} % truncation
\DeclarePairedDelimiter\merely{\lVert}{\rVert_{-1}}
\DeclarePairedDelimiter\angled{\langle}{\rangle}
\DeclarePairedDelimiter\Fin[]
\DeclarePairedDelimiterX\setof[2]\lbrace\rbrace{#1 \mid #2}

\newcommand{\nonempty}[1]{\Trunc{#1}}
\newcommand{\setTrunc}[1]{{\Trunc{#1}}_0}
\newcommand{\settrunc}[1]{{\trunc{#1}}_0}
\newcommand{\grpdTrunc}[1]{{\Trunc{#1}}_1}
\newcommand{\grpdtrunc}[1]{{\trunc{#1}}_1}
\newcommand{\nTrunc}[2]{{\Trunc{#1}}_{#2}}
\newcommand{\ntrunc}[2]{{\trunc{#1}}_{#2}}

\DeclareMathOperator\fundgrp{\pi_1}
\DeclareMathOperator\fundgrpd{\Pi_1}

\newcommand\blfootnote[1]{%
  \begingroup
  \renewcommand\thefootnote{}\footnote{#1}%
  \addtocounter{footnote}{-1}%
  \endgroup}

%%%%%%%%%%%%%%%%%%%%%%%%%%%%%%%%%%%%%%%%%%%%%%%%%%%%%%%%%%%%%%%%%%%%%%%%%%
%from circle on
\newcommand{\covering}{set bundle\xspace}
\newcommand{\coverings}{set bundles\xspace}
\newcommand{\Covering}{Set bundle\xspace}
\newcommand{\Coverings}{Set bundles\xspace}


%%%%%%%%%%%%%%%%%%%%%%%%%%%%%%%%%%%%%%%%%%%%%%%%%%%%%%%%%%%%%%%%%%%%%%%%%%%%%
% these arise from the group theory chapter
%%%%%%%%%%%%%%%%%%%%%%%%%%%%%%%%%%%%%%%%%%%%%%%%%%%%%%%%%%%%%%%%%%%%%%%%%%%%%

%%%%%%%%%%%%%%%%%%%%%%%%%%%%%%%%%%%%%%%%%%%%%%%%%%%%%%%%%%%%%%%%%%%%%%%%%%%%
%originates in group.tex (BID)
\newcommand{\ie}{{\it i.e.},\xspace}%\xspace}fixlater
\newcommand{\eg}{{\it e.g.},\xspace}%\xspace}fixlater
\newcommand{\ev}{\mathrm{ev}}
\newcommand{\ve}{\mathrm{ve}}
\newcommand{\el}{\mathrm{elim}}
\newcommand{\we}{\overset\sim\to}



\newcommand{\iscontr}{\mathrm{isContr}}
\newcommand{\isprop}{\mathrm{isProp}}
\newcommand{\isset}{\mathrm{isSet}}
\newcommand{\isgrpd}{\mathrm{isGrpd}}
\newcommand{\isEq}{\mathrm{isEquiv}}
\newcommand{\isonetype}{\mathrm{1Type}}
\newcommand{\isconn}{\mathrm{isConn}}

\newcommand{\conn}{\mathrm{conn}}
\newcommand{\aut}{\mathrm{Aut}}
\newcommand{\Hom}{\mathrm{Hom}}
\newcommand{\setgroup}[1]{||#1||}
\newcommand{\inftygp}{$\infty$-group\xspace}
\newcommand{\aninftygp}{an $\infty$-group\xspace}
\newcommand{\inftygps}{$\infty$-groups\xspace}
\newcommand{\princ}[1]{\mathrm{Pr}_{{#1}}}%{\mathrm{Princ}}
\newcommand{\pathsp}[1]{\mathrm{P}_{\!#1}} % NB negative thin space
\newcommand{\uc}[1]{{\color{red}\pathsp{#1}}}%universal set bundle
\DeclareMathOperator\abstrOp{abst}
\newcommand{\abstr}{\mathrm{abs}}
\newcommand{\Ad}{\mathrm{Ad}}
\newcommand{\agp}[1]{\mathcal #1}%generic abstract group
\newcommand{\grpcenter}{\operatorname{Z}}
\newcommand{\grpcenterinc}[1]{\mathfrak z_{{#1}}}

\newcommand{\pre}{\mathrm{pre}}%these may be open for discussion
\newcommand{\preinv}{\mathrm{preinv}}
\newcommand{\post}{\mathrm{post}}
\newcommand{\adjoint}{\mathrm{ad}}
\newcommand{\concr}{\mathrm{concr}}

%some special types
\newcommand{\Gtorsor}{\mathrm{tors}_G}
\newcommand{\Xtorsor}[1]{\mathrm{tors}_{#1}}%      added by MAB
\newcommand{\Ztorsor}{\Xtorsor{\zet}}
\newcommand{\TorZ}{{\mathrm{T}\ZZ}} % {\mathbf{TorZor}}% alternative: BZ.   
\DeclareMathOperator\SetBundle{\mathrm{SetBundle}}             % end MAB
\newcommand{\typegroup}{\mathbf{Group}}
\newcommand{\typeabgroup}{\mathbf{AbGroup}}
\newcommand{\typesubgroup}{\mathbf{Sub}}%"gp" removed - is evident from the type of the subscript G
\newcommand{\typeset}{\Set}
\newcommand{\typeinftygp}{{\infty}\mathbf{Group}}
\newcommand{\UU}{\mathcal{U}}
\newcommand{\UUp}{\UU_*}
\newcommand{\pttype}{\UUp}
\newcommand{\typeabsgp}{\mathbf{Group}_{\mathrm{Abstract}}}
\newcommand{\typemonoid}{\mathbf{Monoid}}
\newcommand{\typetorsor}{\mathbf{Tors}}
\newcommand{\permgrp}[1]{\Sigma_{{#1}}}%
\newcommand{\BSigma}{B\Sigma}%previously \Set_{(S)} - the component of S:\Set
\newcommand{\revers}{\mathit{r}}
\newcommand{\twist}{\mathit{twist}}%loop in BC_2
\newcommand{\Sc}{{S^1}}%the circle
%\newcommand{\sbt}{\begin{picture}(-1,1)(-1,-3)\circle*{2}\end{picture}}%
\newcommand{\sbt}{\tikz[anchor=base,baseline]{\node[scale=.7,inner
    sep=0, outer sep=0, circle]%
    {$\bullet$};}}%
\newcommand{\base}{{\sbt}}%point in circle
\newcommand{\Zloop}{\circlearrowleft} %was \mathrm{loop}}%MAB: loop TorZor
\newcommand{\Sloop}{\circlearrowleft}%loop in circle
\newcommand{\bn}[1]{\mathbf{#1}} \newcommand{\Eq}{\mathrm{Eq}}
\newcommand{\emptytype}{\emptyset}

\newcommand{\etop}[1]{\bar {#1}}
\newcommand{\ptoe}[1]{\tilde {#1}}
\newcommand{\cast}{\mathrm{cast}}
\newcommand{\ua}{\mathrm{ua}}%univalence inverse
\newcommand{\zet}{\mathrm{Z}}%the SET of integers

\newcommand{\cst}[1]{\operatorname{\mathit{c}}_{#1}}%
\newcommand{\weqto}{\xrightarrow{\sim}}%
\newcommand{\conncomp}[2]{{{#1}_{\left(#2\right)}}}%
\newcommand{\univcover}[2]{{{#1}^0_{\left(#2\right)}}}%

\newcommand{\blank}{\_}%
\newcommand{\inv}[1]{{#1}^{-1}}%
\newcommand{\ptdto}{\to_\ast}%

\newcommand{\loopspace}[1][\null]{\operatorname{\Omega^{#1}}}

%% paths over paths

%% \newcommand{\pathover}[4]{#1 \overset{#2}{\underset{#3}=} #4}
\newcommand{\pathoverdisplay}[4]{{#1} \overset{#2}{\underset{#3}=} #4}
\newcommand{\pathover}[4]{#1 =^{#2}_{#3} #4}

\newcommand{\pair}{\mathop{\mathrm{pair}}}
\newcommand{\rec}{\mathop{\mathrm{rec}}}
\newcommand{\ind}{\mathop{\mathrm{ind}}}
\newcommand\pathpair[2]{\overline{({#1},{#2})}}

%% Euclidean geometry

\newcommand\EE{\mathbb E}
\newcommand\VV{\mathbb V}
\newcommand\ES{{\tilde \EE}}
\newcommand\OS{{\tilde \VV}}
\newcommand\OrthGp[1]{\mathrm O(#1)}
\newcommand\EucGp[1]{\mathrm E(#1)}
\DeclareMathOperator\Vectors{Vec}
\DeclareMathOperator\Points{Pts}
\newcommand{\typeRealVectorSpace}{\mathbf{Vect}_\RR}

%%%%%%%%%%%%%%%%%%%%%%%%%%%%%%%%%%%%%%%%%%%%%%%%%%%%%%%%%%%%%%%%%%%%%%%%%%%%

% Peter & Benedikt's macros for referring to coqdoc
% d2c4e86
% see https://tex.stackexchange.com/a/35314/ for help understanding the following:

\newcommand{\longhash}{e47ce20acce953129e34e021a10976ed27948a39}
\newcommand{\shorthash}{e47ce20}

%fragile, better to freeze with stable hash
\newcommand{\coqdocbasebaseurl}{https://unimath.github.io/doc/UniMath/\shorthash/}

%\coqident call are relative to this long path
\newcommand{\coqdocbaseurl}{\coqdocbasebaseurl UniMath.} 
\newcommand{\urlhash}{\#}

\newcommand{\coqdocurl}[2]{\coqdocbaseurl #1.html\urlhash #2}

%nolinkurl from url or hyperref package
\newcommand{\nolinkcoqident}[1]{\nolinkurl{#1}} % TODO: give better def for this?
\makeatletter
\newcommand{\coqident}{\begingroup\@makeother\#\@coqident}
\newcommand{\@coqident}[3][]{% empty default first and optional argument
  \ifthenelse{\isempty{#2}}% 
  {\nolinkcoqident{#3}}%           [optional]{}{printme}
  {\ifthenelse{\isempty{#1}}%
  {\href{\coqdocurl{#2}{#3}}{\nolinkcoqident{#3}}}% []{file}{identifier+printed}
  {\href{\coqdocurl{#2}{#3}}{\nolinkcoqident{#1}}}}% [printme]{file}{identifier}
\endgroup}
% optional argument allows link text to differ from link url
\newcommand{\coqfile}[2]{%
  \ifthenelse{\isempty{#1}}%
  {\href{\coqdocbaseurl #2.html}{#2.v}}%
  {\href{\coqdocbaseurl #1.#2.html}{#2.v}}} 
\makeatother

%sususe is a replacement for subsubsection.  sususe is the same as subsubsection but numbers correctly bid
\newenvironment{sususe}[1]{\refstepcounter{theorem}%
\vspace{.5\baselineskip}\par\medskip\noindent%
{\normalfont\normalsize\bfseries{\thetheorem. #1}}%
\vspace{.5\baselineskip}\newline}

%%% Local Variables:
%%% mode: latex
%%% TeX-master: "book"
%%% End:
