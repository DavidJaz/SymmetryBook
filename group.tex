\label{ch:groups}



%\section{Now it starts}
The identity type is not just any type:  in the previous sections we have seen that the identity type $a=_Aa$ reflects the ``symmetries'' of an element $a$ in a type $A$.  
Symmetries have special properties; for instance you can rotate a square by $90^o$, and you can rotate it by $-90^o$, undoing the first rotation.
Symmetries can also be composed, and this composition respects certain rules that holds in all examples.  One way to study the concept of ``symmetries'', would be to isolate the common rules for all our examples, but also show, conversely, that anything satisfying these rules actually \emph{is} an example. 



%As an instance of a property that holds in \emph{some} examples but not in others, we have seen that sometimes the order in which we use our symmetries matters, and sometimes it does not, see \cref{ch:intro}.  Hence, the concept of a group should not have a rule allowing you to change the order arbitrarily.

With inspiration of geometric and algebraic origins, it became clear to mathematicians at the end of the 19'th century that the properties of such symmetries could be codified by saying that they form an abstract \emph{group}. 
In \cref{sec:identity-types} we saw that the identity type was ``reflexive, symmetric and transitive'' -- and an abstract group is just a set with such operations satisfying certain rules.

%This is the purpose of the mathematical term ``group''.

We attack the issue more concretely; instead of focusing on the abstract properties we promote the types exhibiting the symmetries, and the rules follow from the rules for identity types without needing us to worry about them.  However, we \emph{will} show that the two approaches give the same end result.  

In this chapter we lay the foundations and provide some basic examples of groups.  ((describe the contents of the chapter))

\section{The type of groups}
\label{sec:typegroup}

\begin{example}\label{ex:base=base}
  We defined the circle $S^1$ in \cref{def:circle} by declaring that it has a point $\base$ and an element $\Sloop:\base=_{S^1}\base$, and we proved in \cref{cor:S1groupoid} that $\base=_{S^1}\base$ is equivalent to the set $\zet$ (of integers), where $n\in\zet$ corresponds to the $n$-fold composition of $\Sloop$ (which works for both positive and negative $n$).  
We can think of this as describing the symmetries of $\base$: we have one ``generator'' $\Sloop$, and this can be applied any number of times, giving a new symmetry for each new number.  
Here, composition of loops corresponds to usual addition of integers.  Hence, the circle is a very cheap packaging of the ``{group}'' of integers, the declaration of $\base$ and $\Sloop$ not only gives the \emph{set} $\zet$ of integers, but at the same time the addition.
\end{example}
\begin{example}
  Recall the finite set $\bn{2} =\bool:\fin_2$ from \cref{def:finiteset}, containing two elements.   
According to \cref{xca:C2}, $\bn{2} =\bn{2} $ has exactly two elements, $\refl{\mathrm 2}$ and $\twist$, and doing $\twist$ twice gives you back $\refl{\bn{2} }$.  
We see that this is exactly all the symmetries you'd expect to have in a two point set: you can let everything be ($\refl{\bn{2} }$) or you could swap the two elements ($\twist$); and if you swap twice everything is let be.  
The type $\fin_2$ (of ``finite sets with two elements'') is our embodyment of these symmetries.  

Observe that (by the definition of $S^1$) there is an interesting function $S^1\to\fin_2$, sending $\base:S^1$ to $\bn{2} :\fin_2$ and $\Sloop$ to $\twist$.
\end{example}


The examples Klein and Lie were interested in were of a type making it admissible to say that a group is the identity type $a=_Aa$ for \emph{some} type $A$ and \emph{some} element $a:A$.
However, in elementary texts it is customary to restrict the notion of a group to the case when $a=_Aa$ is a \emph{set} as we will do, starting in \cref{sec:identity-type-as-abstract}.  This makes some proofs easier, since if are we given two elements $g,h:a=_Aa$, then the identity type $g=h$ is a proposition, \ie $g$ can be equal to $h$ in at most one way.  Hence questions relating to uniqueness will never be a problem.



See \cref{sec:grouphistory} for a brief summary of the early history of groups.
\begin{remark}
  The reader may wonder about the status of the identity type $a=_Aa'$ where $a,a':A$ are different elements.  One problem is of course that if $p,q:(a=_Aa')$ there is no obvious way of composing $p$ and $q$, and another is that $a=_Aa'$ does not have a distinguished element such as $\mathrm{refl{}_a}:a=_Aa$.
Given $f:a=_Aa'$ we can use transport along $f$ to compare $a=_Aa'$ with $a=_Aa$ (much as affine planes can be compared with the standard plane or a finite dimensional real vector space is isomorphic to some Euclidean space), but absent existence and choice of such an $f$ the identity types $a=_Aa'$ and $a=_Aa$ are different animals.  We will return to this example when we've defined torsors.
\end{remark}


\begin{remark}
  When considering the identity type $a=_Aa$, only the elements $x:A$ with $x$ equal to $a$ are relevant, and we are free to consider only \emph{connected} $A$, \ie where $x=_Aa$ is never empty (c.f.~\cref{def:connected}).  Also, our preference for $a=_Aa$ to be a set indicates that we should consider only the connected types $A$ that are \emph{groupoids}.
\end{remark}


With this established, we let the \emph{type} of groups be defined as follows:

\begin{definition}\label{def:typegroup}
%\footnote{we must define  $\isset$ and propositional truncation.  Alternatively we must define $\isonetype$ and $\conn$}
  A \emph{group} is a pointed connected groupoid; the \emph{type of groups} is the type 
%$$\typegroup=\sum_{A:\UU}A\times\isonetype(A)\times \conn_0A.$$
$$\typegroup\defequi\sum_{A:\UU}\sum_{a:A}\isset(a=_Aa)\times\prod_{x:A}||x=_Aa||$$
of pointed connected groupoids.
%We refer to an element of $\typegroup$ as a \emph{group}.  
A group $G=(A,a,p,q):\typegroup$ will be referred to simply as $$\aut_A(a).$$  The underlying pointed type $$BG\defequi(A,a)$$ is referred to as the \emph{classifying space of $G$}.  The element $\pt_G\defequi a$ will be referred to as the \emph{base point}. 
\end{definition}
Informally, we may also refer to the type $BG_\div\defequi A$ as the classifying space of $G$.
\begin{remark}\label{rem:aut}
There is no ambiguity in writing $\aut_A(a)$ instead of $(A,a,p,q)$: being a connected groupoid is asserted by 
$$\isset(a=_Aa)\times\prod_{x:A}||x=_Aa||$$ which is a proposition  (\cref{lem:props-are-props}) and so the witness $(p,q)$  is unique.  In this sense, once you know that the classifying space is a connected groupoid, $BG$ carries all the information about $G$: $$G\oldequiv\aut_{BG_\div}(pt_{G}).$$
\end{remark}
\end{definition}
\subsection{First examples}
\label{sec:firstgroupexamples}
   \begin{example}\label{excirclegroup}
   The circle $S^1$, which we defined in \cref{def:circle}, is a connected groupoid (\cref{lem:circleisconnected}, \cref{cor:S1groupoid}) and is pointed at $\base$. The identity type $\base=_{S^1}\base$ is equivalent to to the set of integers $\zet$ and composition corresponds to addition.  This justifies our definition of the \emph{group of integers} as 
$$\ZZ=\aut_{S^1}(\base).$$
It is noteworthy that along the way we gave several versions of the circle, each of which has its own merits, the version in \cref{def:S1toC}
$$C=(\sum_{X:\UU}\sum_{f:X=X}||(\zet,s)=(X,f)||, (\zet,s))$$
being a very convenient one.
 \end{example}

\begin{example}\label{ex:groups}
  % Since any pointed connected groupoid is a group, there is no shortage of examples, but perhaps i
  Apart from the circle, there are some important groups that come almost for free: namely the symmetries in the type of sets.
%It is worthwhile to consider some specially designed examples.
  \begin{enumerate}
  \item Recall that the set $\bn{1} =\true$ has the single element which we can call $*$. Then $\aut_{\bn{1} }(*)$ is a group called the \emph{trivial group}.
  \item If $n:\NN$, then the \emph{permutation group of $n$ letters} is 
$$\Sigma_n\defequi\aut_{\fin_n}(\bn{n} ),$$ 
where $\fin_n$ is the groupoid of sets of cardinality $n$ (c.f.~\ref{def:finiteset}).  Note that even though the sets $\bn{n} =_{\fin}\bn{n} $ and $\bn{n} =_{\fin_n}\bn{n} $ are equal, we must use the component $\fin_n$ rather than the entire groupoid $\fin$ of finite sets to keep the underlying pointed groupoid $B\Sigma_n=(\fin_n,\bn{n} )$ connected.
  \item More generally, if $S$ is a set, is there a pointed connected groupoid $(A,a)$ so that $a=_Aa$ models all the ``permutations'' $S=_{\Set}S$ of $S$?  Again, the only thing wrong with ``$\aut_{\Set}(S)$'' (apart from $\Set$ being large\footnote{how do we deal with that?}) is that $\Set$ is not connected. 
%}!\footnote{it's so simple -- so very simple -- that only a child can do it!}  To be precise, the component of $S$ is
%$$A\defequi\sum_{X\in\Set}||S=X||.$$  
%The connected groupoid $\sum_{X\in\Set}||S=X||$ is pointed at $S$ (and the fact that $S=S$ is nonempty since $\refl S:S=S$).    
% Then 
% $$(S=_AS)=(S=_{\Set}S)$$ 
% (in the identity type of a $\Sigma$-type both the first and the second projections must be equal, but for $A\oldequiv\sum_{X:\Set}||S=X||$ the second projection is a proposition).  
%
 So, 
the \emph{group of permutations of $S$} is defined to be $\Sigma_{S}=\aut_{\Set_{(S)}}(S)$.  

Likewise, if $X$ is any type, the \emph{group of automorphism} or \emph{permutations} of $X$ is defined to be 
$$\Sigma_X=\aut_{\UU_{(X)}}X,$$
 where $U_{(X)}$ is the component of $\UU$ containing $X$.
  \end{enumerate}
\end{example}
\begin{example}\label{ex:cyclicgroups}
  In \cref{thm:coveringsofS1} we studied the symmetries of the ``$m$-fold \covering'' of the circle for $m$ a positive integer, and showed that there were $m$-different symmetries, but that they were just the powers $f^n$ (for $n=0,1,\dots,m-1$) of one (nonunique) symmery $f$ and that $f^{m+k}=f^k$ for any integer $k$.  This very important symmetry pops up in many situation, and is called the \emph{cyclic group of order $m$}.  In other words, the cyclic group of order $m$ is the (pointed) component of the type of \coverings of the circle containing the $m$-fold \covering.  We analyzed this in \cref{thm:coveringsofS1} and found that this (pointed) component was equivalent to 
$$BC_n\defequi(\sum_{X:\Set}\sum_{p:X=X}||(X,p)=\zet/m||,(\zet/m,!)).$$

Here is another, and occasionally more convenient, way of obtaining the cyclic group of order $n$.  Consider the function $$cy_n:S^1\to\fin_n$$ with $cy_n(\base)\defequi\bn n$ and $cy_n(\Sloop):\bn n=\bn n$ the identity corresponding to the equivalence given by cyclic permutation, sending $n-1$ to $0$ and, for $0\leq i<n-1$, sending $i:\bn n$ to $i+1$.  Note that the identity $cy_n\Sloop$ is cyclic in the sense that the $n$-fold iterate $(cy_n\Sloop)^n$ is $\refl{\bn n}$.  Then the $n$-fold \covering can be seen as the first projection 
$$\sum_{z:S^1}cy_n(z)\to S^1.$$

Consider the type 
$$B_n\defequi\sum_{S:\fin_n}\pi_0(cy_n^{-1}(S)),\footnote{I've used set truncation!!!}$$
the ``image'' of $cy_n$ except that the truncation is one level higher than we have considered before.  Since $\fin_n$ is a groupoid, $B_n$ is a groupoid.  Remember that $cy_n^{-1}(S)=\sum_{z:S^1}S=cy_nz$.  Let $(S,z,p):B_n$ be any element; we want to show that $(S,z,p)=_{B_n}(\bn n, \base, \refl\base)$ is not empty, so that $B_n$ is connected. Since $S^1$ is connected there is a $q:z=_{S^1}\base$ so $(cy_n(q)\,p,q,!):(S,z,p)=(\bn n,\base,\refl\base)$ (using that $cy_n(\base)\defequi\bn n$ to compose $p:S=cy_nz$ and $cy_n:cy_nz=cy_n\base$), $B_n$ is connected.  
Pointing $B_n$ in $(\bn n,!)$ we have a pointed connected groupoid, \ie a group, which we call the {\em cyclic group $C_n$ of order $n$}.  (in hindsight, $B_n=(BC_n)_\div$)

Note that the cyclic group of order $1$ is the trivial group, the cyclic group of order $2$ is equivalent to the permutation group $\Sigma_2$: there are exactly one nontrivial symmetry $f$ and $f^2$ is the identity.  When $m>2$ the cyclic group of order $m$ is a group that does not appear elsewhere in our current list.  In particular, the cyclic group of order $m$ has only $m$ different symmetries, whereas we will see that the group of permutations $\Sigma_m$ has $m!=1\cdot 2\cdot\dots\cdot m$ symmetries.
\end{example}
\begin{example}\label{ex:productofgroups}
  If you have two groups $G$ and $H$, their {\em product} $G\times H$ is given by taking the product of their classifying spaces:
$$G\times H\defequi\aut_{BG_\div\times BH_\div}((\pt_G,\pt_H))$$
(note that $B(G\times H)\oldequiv BG\times BH$ is pointed in $\pt_{G\times H}\oldequiv(\pt_G,\pt_H)$).  
For instance, $\Sigma_2\times\Sigma_2$ is called the {\em Klein group}.
\end{example}

\footnote{MANY MORE EXAMPLES.  We might tone down exercises like ``prove that $\typegroup$ is a groupoid'', even though we will want to use these results.  They take the geometry/fun out of the exposition.}
\begin{xca}
  \begin{enumerate}
  \item Compare the definitions of \cref{def:finiteset} and show that if $n:\NN$, then $\Sigma_n=\Sigma_{\bn{n} }$ %is equal to the permutation group on $n$ letters 
and (since $\fin_0=\fin_1=\bn 1$) that $\Sigma_{1}=\aut_{\bn{1} }(\triv)$.
%\item Display an element in $\bn{2} =_{\fin_2}\bn{2} $ different from $\refl{\bn{2} }$ in the group $\Sigma_{2}$ of permutations of two letters.  
\item Prove that the set $\bn{n} =_{\fin_n}\bn{n} $ is finite of cardinality $n!$.
\item Show that the $n$-fold cover of $S^1$ is equivalent to the first projection $\sum_{z:S^1}cy_n(z)\to S^1$, where $cy_n:S^1\to\fin_n$ is given by $cy_n(\base)\defequi\bn n$ and $cy_n(\Sloop):\bn n=\bn n$ is cyclic permutation (sending $n-1$ to $0$ and, for $0\leq i<n-1$, sending $i:\bn n$ to $i+1$).  Hint: for every $z:S^1$, $cy_nz:\fin_n$ is a finite set of cardinality $n$.  Decidability is not an issue, so you can appeal to our classification of the \coverings of the circle.
\item Show that, given a type $X$, the type of functions $BC_n\to X$ is equivalent to the type 
$$\sum_{f:S^1\to X}\prod_{z:S^1}f(z)=f(z^n)$$ of functions $f:S^1\to X$ such that the two ways around
$$\xymatrix{S^1\ar[d]_{(-)^n}\ar[dr]^f&\\S^1\ar[r]^f&X}$$
agree.  Hint: define the function $F_1:(BC_n\to X)\to (S^1\to X)$ by precomposition: $F_1(g)(z)=g(cy_nz,!)$ and observe that since $cy_n(z)=cy_n(z^n)$ we have a function $F:(BC_n\to X)\to \sum_{f:S^1\to X}\prod_{z:S^1}f(z)=f(z^n)$.
  \end{enumerate} 
\end{xca}

\begin{remark}
In \cref{lem:idtypesgiveabstractgroups} we will see that groups satisfy a set of laws justifying the name ``group''
%we may associate an abstract group $(a=_Aa,e,{-}^{-1},\cdot)$
and we will later show that groups are uniquely characterized by these laws.
\end{remark}
Some groups have the property that the order you perform the symmetries is immaterial.  The prime example is the group of integers $\ZZ\oldequiv \aut_{S^1}(\base)$  Any symmetry is of the form $\Sloop^n$ for some integer $n$, and if $\Sloop^m$, then $\Sloop^n\Sloop^m=\Sloop^{n+m}=\Sloop^{m+n}=\Sloop^m\Sloop^n$.

 Such cases are important enough to have their own name:
\begin{definition}\label{def:abgp}
  A group $G$ is \emph{abelian} if %all symmetries commute in the sense that 
the proposition
$$\mathbf{isAb}(G)\defequi\prod_{g,h:\pt_G=\pt_G}gh=hg$$
is true.  In other words, the type of abelian groups is 
$$\mathbf{Ab}\defequi \sum_{G:\typegroup}\mathbf{isAb}(G).$$
\end{definition}
\begin{xca}\label{exer:first examples}
  Show that permutation group $\Sigma_2$ is abelian, but that $\Sigma_3$ is not.  Show that if $G$ and $H$ are abelian groups, then so is their product $G\times H$.
\end{xca}
We can envision $g$ commuting with $h$ by the picture
$$\xymatrix{a\ar@{=}[r]^g_\to\ar@{=}[d]^\downarrow_h&a\ar@{=}[d]^\downarrow_h\\
a\ar@{=}[r]^g_\to&a}$$
and saying that going from (upper left hand corner) $a$ to (lower right hand corner) $a$ by either composition gives the same result.
\begin{remark}
  \label{rem:whatAREabeliangroups}
  The reference to $\mathbf{isAb}$ in the definition of abelian groups is avoidable using the ``one point union'' of pointed types $X\vee Y$ of \cref{def:wedge} (it is the sum of $X$ and $Y$ where the base points are identified); \cref{xca:whatAREabeliangroups} offers the alternative definition that a group $G$ is abelian if and only if the ``fold'' map $BG\vee BG\to BG$ factors over the inclusion $BG\vee BG\to BG\times BG$.
\end{remark}
\begin{remark}
  The condition $\isset(a=_Aa)$ in the definition of the type of groups is sometimes more of a nuisance, and deleting it gives the simpler concept of \aninftygp, see \cref{sec:inftygps}.
\end{remark}
\begin{xca}
   Let $\aut_A(a):\typegroup$ and let $b$ be an arbitrary element of $A$.  Prove that the groups $\aut_A(a)$ and $\aut_A(b)$ are equal.  Similarly for \inftygps when you get that far.
\end{xca}
\begin{remark}\label{rem:monoidandabsgplarge}
 In \cref{def:typegroup} the first $\sum$ in the definition of the type $\typegroup$ ranges over the entire universe $\UU$.  Hence, $\typegroup$ does not belong to $\UU$, but rather to the next universe as discussed briefly in \cref{sec:univax}.   This tendency that the ``type of all the types we are interested in'' is a ``large type'' is a regular feature of the theory and since it will not cause any trouble for us, we will not be consistent in pointing it out.
  \end{remark}

  \begin{xca}\label{xca:typegroupisgroupoid}
    Given two groups $G$ and $H$.  Prove that $G=H$ is a set.   Prove that the type of groups is a groupoid.  This means that, given a group $G$, the component of $\typegroup$, containing (and pointed at) $G$, is again a group, which we will call the \emph{group $\aut(G)$ of automorphisms} of $G$.
  \end{xca}

\section{The identity type as an abstract group }
\label{sec:identity-type-as-abstract}

Studying the identity type leads one to the definition of what a group should be:
Let $A$ be a type, and for the moment let $a=b$ be shorthand for $a=_Ab$ when $a,b:A$.  In \cref{sec:identity-types} we saw that
\begin{enumerate}
\item[R] {\bf Reflexivity.} For any $a:A$ there is an element
$$\refl a{}:a=a$$ 
%(by definition)
\item[S] {\bf Symmetry.} For any $a,b:A$ there is a an element $$\symm{}_{a,b}:(a=b)\to (b=a)$$ defined by $\symm{}_{a,a}(\refl a{})\defequi\refl a{}$
\item[T] {\bf Transitivity.} For any $a,b,c:A$ there is an element $$\trans{}_{a,b,c}:(b=c)\to((a=b)\to(a=c))$$ defined by $\trans{}_{a,a,a}(\refl a{})(\refl a{})\defequi \refl a{}$.
\end{enumerate}
%\footnote{\em\bf I have swapped the order of the input in trans so that it can fit.  I know you hate it and will force me to recant}

 To emulate classical notation, for fixed $a:A$,  %for the moment 
let's write
 \begin{enumerate}
% \item $G$ instead of $a=_Aa$,
 \item $e$ instead of $\refl a{}$
 \item $g^{-1}$ instead of $\symm_{a,a}(g)$, when $g:a=a$
 \item $g\cdot h$ instead of $\trans_{a,a,a}(g)(h)$ when $g,h:a=a$.
 \end{enumerate}
 What properties can we show about this without knowing anything about $A$ and $a$? For convenience, here is a list of the results we are aiming for: for all $g,g_1,g_2,g_3:a=a$ we will construct elements in all the following propositions
 \begin{enumerate}
 \item $g=g\cdot e$  \qquad(``right unit law'')\footnote{redundant (keep).  If you still want to reinsert the other redundant $g\cdot g^{-1}=e$ and $(g^{-1})^{-1}=g$ we have to do some renumbering.  
%Forgot which way you prefereed the equalities: from simple to complicated or the other way around?
}
 \item $g=e\cdot g$ \qquad(``left unit law'')
 \item $g_1\cdot(g_2\cdot g_3)=(g_1\cdot g_2)\cdot g_3$ \qquad(``associativity'')
 \item $e=g^{-1}\cdot g$ \qquad(``inverse'').
 %\item $g\cdot g^{-1}=e$ redundant (remove)
% \item $(g^{-1})^{-1}=g$ redundant (remove)
 \end{enumerate}
 

We do $g=e\cdot g$ in some detail (remember that ``$e$'' is shorthand for $\refl a{}$)
\begin{definition}\label{def:p1}
  Let $A$ be a type and $a, b:A$ and $g:a=b$ be elements.  Then $p_1(a,b,g):g=_{a=b}g\cdot e$ is the element defined by induction by saying that $p_1(a,a,e)$ is $\refl e:e=e\cdot e$.
\end{definition}
\begin{remark}
  This makes sense since we \emph{defined} $e\cdot e\defequi e$ (or, as it was originally formulated, $\trans_{a,a,a}(\refl a{})(\refl a{})\defequi \refl a{}$).  We'll say that we produce $p_1(a,b,g)$ by ``induction on $b$'', the case where $b$ is $a$ (and $g$ is $e$) is the start of the induction; the induction priciple for the identity type $a=b$ then finishes the construction.

As constructed, $p_1$ is actually an element in the type
$$\prod_{a:A}\prod_{b:A}\prod_{g:a=b}(g=g\cdot e)$$ -- it is constructed ``uniformly'' or ``naturally'' for all $a,b,g$: think of it as a function with $(a,b,g)$ as input and $p_1(a,b,g):g=g\cdot e$ as output.

We may add a little meat to the definition of $p_1$: in the definition of the identity type, for each $a:A$ let $P$ be the type family given by $P(b,g)\defequi (g=g\cdot e)$ for each $b:A$ and $g:a=b$.  
According to the definition of the identity type, in order to produce elements in $P(b,g)$ for arbitrary $b$ and $g$ it suffices to give an element in $P(a,e)\oldequiv (e=e\cdot e)$, but $e\cdot e\defequi e$ and so $\refl e:e=e$ will do.
\end{remark}
\begin{definition}\label{def:p3}
  Let $A$ be a type and $a,b,c,d:A$ and $g_3:a=b$, $g_2:b=c$ and $g_1:c=d$ elements.  Then $p_3(a,b,c,d,g_1,g_2,g_3):g_1\cdot(g_2\cdot g_3)=_{a=_Ad}(g_1\cdot g_2)\cdot g_3$ is the element defined by induction by saying that $p_3(a,a,a,a,e,e,e,e)$ is $\refl e:e\cdot(e\cdot e)=(e\cdot e)\cdot e$ [which makes sense since $e\cdot e\defequi e$].
\end{definition}
\begin{remark}
  This definition is somewhat more complicated than the first, in the sense that in order to unravel the induction to exactly the form accepted in the definition of the identity type, we need to apply the rule three times.  %((write out))
\end{remark}
\begin{definition}\label{def:p4}
  Let $A$ be a type and let $a,b:A$ and $g:a=b$ be elements.  Then $p_4(a,b,g):g^{-1}\cdot g=_{a=_Aa} e$ is the element defined by induction by saying that $p_4(a,a,e)$ is $\refl e:e=e\cdot e$ [which makes sense since $e^{-1}\defequi e$ and $e\cdot e\defequi e$].
\end{definition}

\begin{xca}\label{xca:p2}
    Define $p_2(a,b,g)$ %and $p_3(a,b,g)$
by exactly the same procedure, completing the list.
\end{xca}

\begin{remark}
   One may worry about many things when one sees the list ``right unit law, left unit law, associativity, inverse''.  For instance, for the particular case of $g$ being $e$, are the elements in $e=e\cdot e$ given in the left and right unit laws equal?  Since $a=a$ is a set, such worries become irrelevant: $e=e\cdot e$ is then a proposition, so any two elements are equal.
 \end{remark}

These properties of the identity type are bundled together in the concept of an abstract group, under the additional hypothesis that we are dealing with a set.

  \begin{definition}\label{def:abstractgroup}
    An \emph{abstract group} is a set $S$ together with
\begin{enumerate}
\item an element $e:S$,
\item a function taking a pair of elements $g_1,g_2:S$ to a third element which we call $g_1\cdot g_2:S$ such that
  \begin{enumerate}
  \item %$e$ is a ``neutral element'':
if $g:S$, then $g\cdot e=e\cdot g=g$ and
  \item %satisfying ``associativity'':
if $g_1,g_2,g_3:S$, then
$$g_1\cdot(g_2\cdot g_3)=(g_1\cdot g_2)\cdot g_3,$$
  \end{enumerate}
\item %inverses:
to every $g:S$ there is a $g^{-1}:S$ such that $%g\cdot g^{-1}=
e=g^{-1}\cdot g$.
\end{enumerate}
We refer to $e$ as the \emph{unit element}, $g_1\cdot g_2$ as the \emph{product of $g_1$ and $g_2$} and $g^{-1}$ as the \emph{inverse of $g$}.  The \emph{unit laws} will then be $g\cdot e=e\cdot g=g$, the \emph{associativity law} is $g_1\cdot(g_2\cdot g_3)=(g_1\cdot g_2)\cdot g_3$ and $%g\cdot g^{-1}=
g^{-1}\cdot g=e$ is referred to as the \emph{law of inverses}.  The set $S$ is called the \emph{underlying set} of the abstract group.
  \end{definition}

In conclusion we have proved that groups give rise to abstract groups:
  \begin{lemma}\label{lem:idtypesgiveabstractgroups}
    If $G$ is a group, then $\pt_G=\pt_G$ together with $e\defequi\refl{\pt_G}{}$, $g^{-1}\defequi\symm_{\pt_g,\pt_G}g$ and $g\cdot h\defequi\trans_{\pt_g,\pt_G,\pt_G}(g)(h)$
%$A$ is a groupoid %(alternatively called a ``$1$-type'') and $a:A$ is an element, then $a=_Aa$, together with $e\defequi\refl a{}$, $g^{-1}\defequi\symm_{a,a}g$ and $g\cdot h\defequi\trans_{a,a,a}(g)(h)$ 
define an abstract group
$$\abstr(G)\defequi (\pt_G=\pt_G,e,{-}^{-1},\cdot).$$
  \end{lemma}
  \begin{proof}
    The elements $p_1,\dots p_4$ of \cref{def:p1,def:p4,def:p3,xca:p2} show that all the relevant identity types (which are propositions since $A$ is a groupoid) are nonempty, as required.
  \end{proof}
  \begin{definition}\label{def:abstrG}
    Given a group $G$, the abstract group $\abstr(G)\defequi (a=_Aa,e,{-}^{-1},\cdot)$ of \cref{lem:idtypesgiveabstractgroups} is called the \emph{abstract group associated to $G$}.
  \end{definition}

  \begin{remark}
    It is sometimes handy to break up the rather long \cref{def:abstractgroup}  by saying that the right and left unit law together with associativity define a \emph{monoid}, and if we, in addition, have inverses satisfying the law of inverses, then we have an abstract group.
    % \end{remark}


    % \begin{remark}
      \label{rem:typemonoidabstrgp}
        Summing up in language a machine (and the occasional mad scientist) can handle, the \emph{type of monoids} is
$$\typemonoid\defequi \sum_{M:\UU}\sum_{e:M}\sum_{\mu{}:M\to M\to M}
\isset{(M)}\times\mathrm{Monoidlaws}(M,e,\mu)
$$
where
$$\mathrm{Monoidlaws}(M,e,\mu)\defequi\mathrm{Unitlaws}(M,e,\mu)\times\mathrm{Assoclaw}(M,\mu{})$$and
\begin{align*}
  \mathrm{Unitlaws}(M,e,\mu)\defequi\prod_{g:M}
&(g=\mu{}(g)(e))\times(g=\mu{}(e)(g)),\\
\mathrm{Assoclaw}(M,\mu{})\defequi\prod_{g_1,g_2,g_3:M}&\mu{}(g_1)(\mu{}(g_2)(g_3))=\mu{}(\mu{}(g_1)(g_2))(g_3).
\end{align*}
In the human language we used above, $\mu(g)(h)=g\cdot h$ and $\iota(g)=g^{-1}$ and $\mathrm{Unitlaws}$ and $\mathrm{Assoclaw}$ spell out to the machine that the unit behaves like a unit and that the multiplication is associative.
The
\emph{type of abstract groups} is
$$%\typeabsgp
\typegroup^\abstr\defequi
\sum_{(M,e,\mu):\typemonoid}\sum_{\iota\colon M\to M}\prod_{g:M}(\mu{}(\iota{}(g))(g)=e).$$
% where
% $$\mathrm{Grouplaws}(G,e,\mu,\iota)\defequi\mathrm{Monoidlaws}(G,e,\mu)\times \mathrm{Invlaws}(G,\iota{},\mu{},e)$$
% and
% $$\mathrm{Invlaws}(G,e,\mu{},\iota{})\defequi
% \prod_{g:G}(\mu{}(\iota{}(g))(g)=e)\times
% (\mu{}(g)(\iota{}(g))=e)\times
% (\iota{}(\iota{}(g))=g).$$
We will typically refer to a monoid as a triple $(M,e,\mu)$, omitting the names for the (true) $\isset$ and unit and associativity laws, and likewise, an abstract group will be referred to as a quadruple $(M,e,\mu,\iota)$.  The \emph{underlying set} of a group is defined by setting 
$$\mathrm{under}(M,e,\mu,\iota)=M.$$
\end{remark}
\begin{remark}
  That the concept of an abstract group synthesizes the idea of symmetries will be manifested shortly when we prove that 
$$\abstr:\typegroup\to\typegroup^\abstr$$
is an equivalence.
\end{remark}
\begin{remark}
  If $\mathcal G=(S,e,\mu,\iota)$ and $\mathcal G'=(S',e',\mu',\iota')$ are abstract groups, an element of the identity type $\mathcal G=\mathcal G'$ consists of quite a lot of information.  First and foremost, we need an identity $p:S=S'$ of sets, but from there on the information is a list of elements in propositions (this is more interesting for \inftygps).  An analysis shows that this list can be shortened to ``$e'=p(e)$ and $\mu'(p(s),p(t))=p(\mu(s,t))$''.  The most convenient way of obtaining such an identity is to dream up a function $f:S\to S'$ that preserve all the group laws (\ie what will shortly be called an abstract homomorphism) and then show that $f$ is an equivalence and  under univalence gives rise to an identity.
\end{remark}
\begin{xca}
  \label{xca:conj}
  Let $\mathcal G=(S,e,\mu,\iota)$ be an abstract group and let $g:S$.  If $s:S$, let $c^g(s)\defequi g\cdot s\cdot g^{-1}$.  Show that the resulting function $c^g:S\to S$ preserves all group laws (for instance $g\cdot(s\cdot s')\cdot g^{-1}=(g\cdot s\cdot g^{-1} )\cdot(g\cdot s\cdot g^{-1})$) and is an equivalence.  The resulting identity $c^g:\mathcal G=\mathcal G$ is called \emph{conjugation} by $g$\index{conjugation}.
\end{xca}

  \begin{remark}
Without the demand that the underlying type of an abstract group or monoid is a set, life would be more complicated.  For instance, for the case when $g$ is $e$, the unit law of \cref{def:abstractgroup} (or alternatively $\mathrm{Unitlaws}(G,\mu{},e)(e)$ in \cref{rem:typemonoidabstrgp}) would provide \emph{two} (potentially different) proofs that $e=e\cdot e$ and we would have to separately insist that they agree.  This problem vanishes in the setup we adopt below for \inftygps.
  \end{remark}

  \begin{xca}
    For an element $g$ in an abstract group $(G,e,\mu,\iota)$, prove that $e=g\cdot g^{-1}$ and $g=(g^{-1})^{-1}$ (for the machines among us: ``give an element in the proposition
$\prod_{g:\pt_G=\pt_G}
(e=\mu{}(g)(\iota{}(g)))\times
(g=\iota{}(\iota{}(g)))$'').
  \end{xca}
  \begin{xca}\label{xca:typemonoidisgroupoid}
    Prove that the types of monoids and abstract groups are groupoids.
  \end{xca}
  \begin{xca}
    \label{xca:cheapgroup}
    There is a leaner way of characterizing what an abstract group is: define a \emph{sheargroup} to be a set $S$ together with an element $e:S$, a function $S\times S\to S$ sending $(a,b):S\times S$ to $a*b:S$ and the following propositions where we use the shorthand $\bar a\defequi a*e$:
    \begin{enumerate}
    \item $e*a=a$,
    \item $a*a=e$ and
    \item $c*(b*a)=\overline{(c*\bar b)}*a$,
    \end{enumerate}
    for all $a,b,c:S$.
    Show that the type of abstract groups is equivalent to the type of sheargroups.  

Hint: setting $a\cdot b\defequi \bar b*a$ gives you an abstract group from a sheargroup and conversely, letting $a*b=b\cdot a^{-1}$ takes you back.  On your way you may need at some point to show that $\overline{\bar a}=a$: setting $c=\bar a$ and $b=a$ in the third formula will do the trick (after you have established that $\bar e=e$).  This exercise may be good to look back to in the many instances where the inverse inserted when ``multiplying from the right by $a$'' is forced by transport considerations. 
  \end{xca}




\section{Homomorphisms}
\label{sec:homomorphisms}


The notion of a group homomorphism from $G=\aut_A(a)$ to $H=\aut_B(b)$ is simple: it is an function $f:A\to B$ that ``sends $a$ to $b$'', \ie together with an element $p:a=_Bf(b)$:
\begin{definition}\label{def:grouphomomorphism}
  The type of \emph{group homomorphisms} from $G:\typegroup$ to $H:\typegroup$ is defined to be
$$\Hom(G,H)\defequi(BG\to_* BH),
%\footnote{would you be unhappy if I used $f:G\to_{\typegroup}H$ when it fits better with the typography?}%\sum_{f:A\to B}f(a)=_Bb.
$$
in other words, a homomorphism $f:\Hom(G,H)$ is a pair $(Bf_\div,p_f)$ where $Bf_\div:BG_\div\to BH_\div$ and $p_f:\pt_H=Bf_\div(\pt_G)$.
\end{definition}
When there is little danger of confusion we may drop the subscript $\div$ even when talking about the unpointed structure.
\begin{example}
  \begin{enumerate}
  \item   Consider two sets $S$ and $T$.  
Recall that $\Set_{(S)}\defequi\sum_{X:\Set}||S=X||$ is the component of the groupoid $\Set$ containing $S$, and when pointed at $S$ represents the permutation group $\Sigma_S$.  
The map $\Set_{(S)}\to\Set_{(S\coprod T)}$ sending $X$ to $X\coprod T$ induces a group homomorphism $\Sigma_S\to\Sigma_{S\coprod T}$.
Thought of as symmetries, this says that if you have a symmetry of $S$, then we get a symmetry on $S\coprod T$ (which doesn't do anything to $T$).  

Likewise, we get a map $\Set_{(S)}\to\Set_{(S\times T)}$ sending $X$ to $X\times T$ induces a group homomorphism $\Sigma_S\to\Sigma_{S\times T}$. 

In particular, we get homomorphisms $\Sigma_m\to\Sigma_{m+n}$ and $\Sigma_m\to\Sigma_{mn}$. \footnote{find a good description of the sign $\Sigma_n\to\Sigma_2$}
\item Let $G$ be a group.  Since there is a unique map from $BG$ to $\bn{1} $, we get a unique homomorphism from $G$ to the trivial group.  
Likewise, there is a unique morphism from the trivial group to $G$, sending the unique element of $\bn 1$ to $\pt_G$. 
\item If $G$ and $H$ are groups, the inclusions and projections between $B(G\times H)\oldequiv BG\times BH$ and $BG$ and $BH$ give rise to group homomorphisms between $G\times H$ and $G$ and $H$.  \footnote{Elaborate}
  \end{enumerate}
\end{example}
\begin{xca}
  Let $G$ be a group.  Show that $\Hom(\ZZ,G)=(\pt_G=_{BG}\pt_G)$.  %Show that ((wedges of circle vs multiplication))
\end{xca}
\begin{example}
  We will later show that if $G$ and $H$ are groups, then $\Hom(G,H)$ is equivalent to the {\bf set} of ``abstract group homomorphisms'' from $\abstr(G)$ to $\abstr(H)$, but it is instructive to see that $\Hom(G,H)$ is a set directly from the definition.

  Let us spell out the data needed to give an identity between two group homomorphisms $f,f':\Hom(G,H)$.  
We clearly must have a
$$J:Bf_\div=Bf_\div'\defequi\prod_{z:BG_\div}Bf_\div(z)=Bf'_\div(z),$$ and transport on $\pt_H=Bf_\div(?)$ shows that in addition we must have $!:J(\pt_G)\,p_f=_{\pt_H=Bf'_\div(\pt_G)}p_{f'}$:
$$\xymatrix{&\pt_H\ar@{=}[dl]_{p_f}^\gets\ar@{=}[dr]^{p_{f'}}_\to&\\
Bf_\div(\pt_G)\ar@{=}[rr]^{J(\pt_G)}_\to&&\,Bf_\div(\pt_G);}
$$
in other words, we have an equivalence
$$(f=f')\simeq \sum_{J:Bf_\div=Bf'_\div}J(\pt_G)\,p_f=p_{f'}.
$$
Now, $J$ being member of a $\prod$-type implies that if $z:BG$ and $\gamma:\pt_G=z$, then $!:J(z)Bf_\div(\gamma)=Bf'_\div\gamma J(\pt_G)$
$$\xymatrix{
Bf_\div(\pt_G)\ar@{=}[r]^{J(\pt_G)}_\to\ar@{=}[d]^\downarrow_{Bf_\div\gamma}&
Bf'_\div(\pt_G)\ar@{=}[d]^\downarrow_{Bf'_\div\gamma}\\
Bf_\div(z)\ar@{=}[r]^{J(z)}_\to&Bf'_\div(z)},
$$
so that -- since $BG$ is connected and $Bf_\div(z)=Bf'_\div(z)$ is a set -- $J(z)$ is determined by $J(\pt_G)$ which again is determined by $p_f$ and $p_{f'}$.  
This shows that $f=f'$ is a proposition and that $\Hom(G,H)$ is a set.

% We want to show that $f=f'$ is a proposition.  If $(J,!)$ and $(J',!)$ are elements in $\sum_{J:Bf_\div=Bf'_\div}J(\pt_G)\,p_f=p_{f'}$, then $!:J(\pt_G)=p_{f'}p_f^{-1}=J'(\pt_G)$.  Since $J(z)=J'(z)$ is a proposition for each $z:BG$ it suffices to display a map $(\pt_G=z)\to(J(z)=J'(z))$. 
\end{example}


\begin{xca}\label{xca:BGtotype}
  Let $G$ be a group and $A$ a connected groupoid.  Use the definitions and \cref{xca:freemaps} to show that the types
  \begin{enumerate}
  \item $BG_\div\to A$, 
  \item $\sum_{a:A}\sum_{f:BG_\div \to A}(f(\pt_G)=a)$, 
  \item $\sum_{a:A}(BG\to_*(A,a))$ and 
  \item $\sum_{a:A}\Hom(G,\aut_A(a))$
  \end{enumerate}
 are all equivalent.
\end{xca}

The definition of group homomorphisms in \cref{def:grouphomomorphism} should be contrasted with the usual -- and somewhat more cumbersome -- notion of a group homomorphism $f\colon G\to H$ of abstract groups where we must specify that in addition to preserving the neutral element ``$f(e_G)=e_H$'' it must preserve multiplication: ``$f(g)\cdot_H f(g')=f(g\cdot_G g')$ (where we have set the name of the group as a subscript to $e$ and $\cdot$).  In our setup this is simply true:

\begin{definition}\label{def:grouphomomaxioms}
Let $G$ and $H$ be groups and assume given a group homomorphism $f:G\to H$.  We now define something that we will call an ``abstract group homomorphism 
${\abstr}(f):\abstr(G)\to \abstr(H)$'', \ie a function of sets from $(\pt_G=\pt_G)$ to $(\pt_H=\pt_H)$ ``preserving'' the group structure, c.f~\cref{def:abstrG} for the definition of $\abstr(G)$ and \cref{def:abstrisfunctor} for a condensation of what the discussion below end up with concluding that ``preserves'' means.  

We take the time to develop this from first principles.
Recall that $f$ is nothing but a pointed function from $BG$ to $BH$; or in other words a map of (unpointed) types 
$$Bf_\div\colon BG_\div\to BH_\div$$ 
and an identity 
$$p_f: Bf_\div(\pt_G)=\pt_H.$$  
As in \cref{def:apd}%\footnote{I use $f(p)$ rather than the $\mathrm{ap}_f$-formalism which I think is alienating when compared with the classical setup}
, for $z:BG$ this gives rise to a map 
$$\ap{Bf_\div}:(\pt_G=z)\to (Bf_\div(\pt_G)=Bf_\div(z)),$$ 
defined by induction by declaring that $\ap{Bf_\div}(\refl{\pt_G})\defequi \refl{Bf_\div(\pt_G)}$.  
If $g:\pt_G=\pt_G$, then $\ap{Bf_\div}(g)$ is an element of $Bf_\div(\pt_G)=Bf_\div(\pt_G)$, while we want something in $\pt_H=\pt_H$.  However, this is not an obstacle since conjugation by $p_f: Bf_\div(\pt_G)=\pt_H$ gives rise to 
$$\mathrm{ad}_{p_f}:(Bf_\div(\pt_G)=Bf_\div(\pt_G))=(\pt_H=\pt_H)$$ (with $\mathrm{ad}_{p_f}(\refl{Bf_\div(\pt_G)})=\refl{\pt_H}$, as discussed in \cref{sec:heavy-transport}) and so 
$$\mathrm{ad}_{p_f}(\ap{Bf_\div}(g)):\pt_H=\pt_H.$$
This defines a function
$$f^\abstr\defequi \mathrm{ad}_{p_f}\ap{Bf_\div}:(\pt_G=\pt_G)\to(\pt_H=\pt_H)$$  
and we depict $f^\abstr(g)$ as the ``up, over and down'' identity of $\pt_H$:
$$\xymatrix{Bf_\div(\pt_G)\ar@{=}[r]^{\ap{Bf_\div}(g)}_\to\ar@{=}[d]^\downarrow_{p_f}&Bf_\div(\pt_G)\ar@{=}[d]^\downarrow_{p_f}\\
\pt_H\ar@{:}[r]^{f^\abstr(g)}_\to&\pt_H}.$$


% Since type-checking removes the ambiguity, we trust it will not lead to any confusion that we simplify the notation and use the symbol $f$ simultaneously for $Bf_\div$, writing $f\colon BG_\div\to BH_\div$ and for $\ap{Bf_\div}$, writing $f:(\pt_G=\pt_g)\to (f(\pt_G)=f(\pt_G))$, while we write
% $$f^\abstr\defequi \mathrm{ad}_p\ap{Bf_\div}:(\pt_G=\pt_G)\to(\pt_H=\pt_H), \qquad g\mapsto p\,f(g)\,p^{-1}$$  
% and depicting it as the ``up, over and down'' identity of $\pt_H$:
% $$\xymatrix{f(\pt_G)\ar@{=}[r]^{f(g)}_\to\ar@{=}[d]^\downarrow_p&f(\pt_G)\ar@{=}[d]^\downarrow_p\\
% \pt_H&\pt_H}.$$
% When time comes, even the superscript $\abstr$ will vanish.
% % (which is defensible, given that $p$ is part of the homomorphism $f$).

With the shorthand $$e_G\defequi\refl{\pt_G}:(\pt_G=\pt_G)\oldequiv\abstr(G)$$ and writing (to remind us where things happen)
$$g\cdot_Gg':(\pt_G=\pt_G)$$
 for the composite $g\,g'$ of $g$ and $g'$ (note that we use functional notation, so that composition is ``first $g'$ and then $g$'' as in the picture 
$\xymatrix{\pt_G\ar@{=}[r]^{g'}_\to&\pt_G\ar@{=}[r]^{g}_\to&\pt_G}$) %$\trans_{\pt_G,\pt_G\pt_G}(g,g'):$ 
and likewise with a subscript $H$, we have the following
  \begin{enumerate}
  \item $\refl{e_H}:f^\abstr(e_G)= e_H$ makes sense since
$$f^\abstr(e_G)\oldequiv\mathrm{ad}_{p_f}\ap{Bf_\div}(\refl{\pt_G})\oldequiv\mathrm{ap}_{p_f}\refl{f(\pt_G)}\oldequiv\refl{\pt_H}\oldequiv e_H
$$
% Since $\ap{Bf}(e_G)\oldequiv \ap{Bf}f(\refl{\pt_G})\oldequiv\refl{f(\pt_G)}$ and $e_H\oldequiv\refl{\pt_H}$, 
% $$f(e_G)\oldequiv\mathrm{ad}_p(\ap{Bf}(e_G))\oldequiv\mathrm{ad}_p(\refl{f(\pt_G)})\oldequiv \refl{\pt_H}\oldequiv e_H$$
      \item the proposition $f^\abstr(g\cdot_Gg')=f^\abstr(g)\cdot_Hf^\abstr(g')$ is inhabited by the composite
        \begin{align*}
          f^\abstr(g\cdot_Gg')&\defequi %\mathrm{ad}_{p_f}\ap{Bf}(g\cdot g)\\\oldequiv&
            \mathrm{ad}_{p_f}\ap{Bf}(g\, g')\\
          &=\mathrm{ad}_{p_f}(\ap{Bf}(g)\,\ap{Bf}(g)')\\
          %&= (\mathrm{ad}_{p_f}\ap{Bf}(g))\,{\mathrm{ad}_{p_f}\ap{Bf}(g)')\\
          &=\mathrm{ad}_{p_f}\ap{Bf}(g)\cdot_H\mathrm{ad}_{p_f}\ap{Bf}(g')\oldequiv f^\abstr(g)\cdot_Hf^\abstr(g'),
        \end{align*}
where we have used that both $\ap{Bf_\div}$ and $\mathrm{ad}_{p_f}$ take composites of identities to composites of identities.\footnote{give ref}

If you find it useful, you may consider the following picture:
$$\xymatrix{
Bf_\div(\pt_G)\ar@{=}[rr]^{\ap{Bf_\div}(g\, g')}_\to\ar@{=}[dr]^{\ap{Bf_\div}(g')}_\to\ar@{=}[dd]_{p_f}^\downarrow
&&Bf_\div(\pt_G)\ar@{=}[dd]_{p_f}^\downarrow\\
&Bf_\div(\pt_G)\ar@{=}[d]_{p_f}^\downarrow
\ar@{=}[ur]^{\ap{Bf_\div}(g)}_\to
&\\
\pt_H&\pt_H&\pt_H},$$
where $f^\abstr(g\cdot_Gg')$ is simply ``up, over and down'' while $f^\abstr(g)\cdot_Hf^\abstr(g')$ takes the detour via the $\pt_H$ in the middle.
  \end{enumerate}
\end{definition}
\begin{definition}\label{def:abstrisfunctor}
  If $\mathcal G\defequi(\mathcal G_{\Set},e_{\mathcal G},\mu_{\mathcal G},\iota_{\mathcal G})$ and $\mathcal H\defequi(\mathcal H_{\Set},e_{\mathcal H},\mu_{\mathcal H},\iota_{\mathcal H})$ are two abstract groups, then the set of homomorphisms from $\mathcal G$ to $\mathcal H$ is
 $$  \Hom^\abstr(\mathcal G,\mathcal H)%(G_{\Set},e_G,\mu_G,\iota_G),(H_{\Set},e_H,\mu_H,\iota_H))\\
=\sum_{f:\mathcal G_{\Set}\to \mathcal H_{\Set}}(f(e_{\mathcal G})=e_{\mathcal H})\times (f\mu_{\mathcal G}=\mu_{\mathcal H}f).
$$
Since $(f(e_{\mathcal G})=e_{\mathcal H})\times (f\mu_{\mathcal G}=\mu_{\mathcal H}f)$ is a proposition, a homomorphism of abstract group is uniquely determined by its underlying function of sets, and unless there is danger of confusion we may write $f$ instead of $(f,!)$. 

If $G$ and $H$ are groups, the function
$$\abstr:\Hom(G,H)\to\Hom^\abstr(\abstr(G),\abstr(H))$$
is the function $\abstr(f)=(f^\abstr,!)$ defined in \cref{def:grouphomomaxioms}.
\end{definition}
\begin{xca}
  Note that the inverses play no r\^ole in the definition of a homomorphism of abstract groups.  Prove that if $(f,!):\Hom^\abstr(\mathcal G,\mathcal H)$
%(G_{\Set},e_G,\mu_G,\iota_G),(H_{\Set},e_H,\mu_H,\iota_H))$ 
then $f(g^{-1})=(f(g))^{-1}$, making a separate axiom redundant.  
\end{xca}
\begin{example}
  \label{ex:conjhomo}
  Let $\mathcal G=(S,e,\mu,\iota)$ be an abstract group and let $g:S$.  In \cref{xca:conj} we defined $c^g:S\to S$ by setting $c^g(s)\defequi g\cdot s\cdot g^{-1}$ for $s:S$ and asked you to show that it ``preserves all group laws'', \ie represents a homomorphism
$$c^g:\Hom^\abstr(\mathcal G,\mathcal G)$$
called \emph{conjugation} by $g$\index{conjugation}.  
Actually, we asked more: namely that conjugation represents an identity (for which we used the same symbol) $c^g:\mathcal G=\mathcal G$.

If $\mathcal H$ is some other abstract group, transport along $c^g$ gives an identity
 $c^g_*:\Hom(\mathcal H,\mathcal G)=\Hom(\mathcal H,\mathcal G)$ which should be viewed as ``postcomposing with conjugation''.  (Naturally, similar considerations goes for elements in $\mathcal H$, giving rise to ``precomposition with conjugation''.)
\end{example}

\begin{xca}
Prove that composition of the functions on the underlying sets gives a composition of homomorphisms of abstract groups.

  Prove that if $f_0:\Hom(G_0,G_1)$ and $f_1:\Hom(G_1,G_2)$ then 
$$\abstr(f_1f_0)=\abstr(f_1)\abstr(f_0)$$ and that $\abstr(\id_G)=\id_{\abstr(G)}$.
\end{xca}



\section{\texorpdfstring{\inftygps}{∞-groups}}
\label{sec:inftygps}

Disregarding the set-condition we get the simpler notion of \inftygps:
\begin{definition}The type of $\infty$-groups is
  $$\typeinftygp\defequi \sum_{A:\UU}\sum_{a:A}\prod_{x:A}||x=_Aa||.$$
\end{definition}

\begin{remark}\label{rem:pointedtypes}
  Just as ``group'' is a synonym for ``pointed, connected groupoid'', ``$\infty$-group'' is synonym for ``pointed, connected type''.  As for groups, we suppress the propositional information from the notation and write $\aut_A(a)$ isntead of $(A,a,!)$ for an $\infty$-group.
% ;  the type of \emph{pointed types} being
% $$\pttype\defequi\sum_{A:\UU}A,$$
% and given two pointed types $(A,a)$ and $(B,b)$, the type of \emph{pointed maps} from $(A,a)$ to $(B,b)$ is
% $$((A,a)\to_*(B,b))\defequi\sum_{f\colon A\to B}f(a)=b.$$
\end{remark}


\footnote{Let $\typeset\defequi \sum_{A:\UU}\isset(A)$.}
\begin{definition}\label{def:classifyingspace}
  If $G\oldequiv\Aut_A(a):\typeinftygp$, then the underlying pointed type $BG\defequi (A,a)\colon\pttype$ is called the  \emph{classifying space} and $\pt_G\defequi a$ is the \emph{base point}.  
%We retain the same language also for ordinary groups in which case the classifying space is a groupoid (\ie a $1$-type).   %For \inftygps the definition is identical.
\end{definition}
\begin{definition}
  A homomorphism of $\infty$ groups is a pointed function of classifying spaces, \ie
  given two $\infty$-groups $G$ and $H$ we define
$$\Hom(G,H)\defequi(BG\to_*BH).$$
% If $G$ is \aninftygp, we let $\pt_G:BG$ (and sometimes simply $\pt$ if $G$ is clear from the context) be the distinguished point (so that $G\oldequiv\aut_{BG}(\pt_G)$).
\end{definition}





\section{$G$-sets}
\label{sec:gsets}

One of the goals of the next section is to prove that, in a precise sense, any abstract group corresponds to a group.  In doing that, we are invited to explore how abstract groups should be thought of as symmetries and introduce the notion of a $G$-set.  However, this takes a pleasant detour where we have to explore the most important feature of groups: they \emph{act} on things (giving rise to symmetries)!

Before we handle the more complex case of abstract groups, let us see what this looks like for groups.

\begin{definition}
  For $G$ a group (or \inftygp), a \emph{$G$-type} is a function
  $$X\colon BG\to\UU,$$
%($\BG_\div$ was defined to be the underlying type of $BG$)
and $X(\pt_G)$ is referred to as the \emph{underlying type}.
If the underlying type is a set, then $X$ is called a \emph{$G$-set}.

Otherwise said, the type of $G$-types is $\Type_G\defequi(BG\to\UU)$ and the type of $G$-sets is $\Set_G\defequi(BG\to\Set)$.
%$$\Type_G\defequi (BG\to\UU),\qquad \Set_G\defequi (BG\to\Set).$$
\end{definition}

\begin{remark}
  The reader will notice that the type of $G$-set is equivalent to the type of \coverings over $BG$. %\footnote{we should be careful with having too many official names for the same objects}
The reason we have allowed ourselves two names is that our focus is different: for a $G$-set $X:BG\to\Set$ we focus on the sets $X(z)$, whereas when talking about \coverings the first projection $\sum_{z:BG}X(z)\to BG$ takes center stage.  Each focus has its advantages.

Given a $G$-set, we may consider it as a $G$-type and will usually not make a notational distinction.
\end{remark}

\begin{example}\label{def:principaltorsor}
  If $G$ is a group (or \inftygp), then
$$\princ G:BG\to\UU,\qquad\princ G(z)\defequi\pathsp{\pt_G}(z)\defequi(\pt_G=z)$$ is a $G$-set (or $G$-type) called the \emph{principal $G$-torsor}.  
We've seen this family before in the guise of the (fibers of the) ``universal \covering'' of \cref{def:universalcover}!  
The term ``$G$-torsor'' will reappear several times and will mean nothing but a $G$-type in the component of $\princ G$ -- a ``twisted'' version of $\princ G$.

There is nothing sacred about starting the equality in $\pt_G$: if $y:BG$, then $\pathsp y:BG\to\UU$ is also a $G$-set (type) and if $q:y=y'$, then the preferred equality between $\pathsp y$ and $\pathsp{y'}$ sends $p:y=z$ to $pq^{-1}:y'=z$.  As a matter of fact, \cref{lem:BGbytorsor} will identify $BG$ and the type of $G$-torsors via the map sending $y:BG$ to $\pathsp y$ using the full transport structure of the identity type $\pathsp yz\defequi(y=z)$.
 
%The name torsor will be explained shortly.
\end{example}

\begin{example}\label{def:adjointrep}
  If $G$ is a group (or \inftygp), then
$$\Ad_G:BG\to\UU,\qquad\Ad_G(z)\defequi(z=z)$$ is a $G$-set (or $G$-type) called the \emph{adjoint $G$-set (or $G$-type)}.  The name ``adjoint'' comes from how transport works in this case; if $p:y=z$, then $\Ad_G(p):(y=y)=(z=z)$ is given by conjugation: 
$$\Ad_G(p)(q)=pqp^{-1}:z=z,$$ the picture
$$\xymatrix{y\ar@{=}[r]^p_\to\ar@{=}[d]_q^\downarrow&z\ar@{=}[d]^{\Ad_G(p)(q)}_\downarrow\\
y\ar@{=}[r]^p_\to&z}$$
is a mnemonic device illustrating that it couldn't have been different, and should be contrasted with the picture for $\princ G (p):(\pt_G=y)=(\pt_G=z)$:
$$\xymatrix{\pt_G\ar@{=}[r]^{\refl{\pt_G}}_\to\ar@{=}[d]_q^\downarrow&\pt_G\ar@{=}[d]^{\princ G(p)(q)}_\downarrow\\
y\ar@{=}[r]^p_\to&\,z.}$$  
Notice that by the induction principle for the circle,
$$\sum_{z:BG}\Ad_G(z)=\sum_{z:BG}(z=z)$$
is equivalent to the type of (unpointed!) maps $S^1\to BG$, known in other contexts as the \emph{free loop space} of $BG$, an apt name given that it is the type of ``all symmetries of $BG$.''  
The first projection $\sum_{z:BG}\Ad_G(z)\to BG$ correspond to the function $(S^1\to BG)\to BG$ given by evaluating at $\base$. 
\end{example}
\begin{example}
  \label{ex:HomHGasGset}
  Recall that a homomorphism $f:\Hom(H,G)$ consists of an unpointed map $F:BH_\div\to BG_\div$ together with a $p_f:\pt_G=F(\pt_H)$, so if, for $x:BH$ and $y:BG$, we define
$$\Hom(H,G)(x,y)\defequi\sum_{F:BH_\div\to BG_\div}(y=F(x))$$
we see that $\Hom(H,G)$ may be considered to be a $H\times G$-set
$$\Hom(H,G);BH\times BG\to\Set.$$
We will be particularly be interested in the restriction to $G$ giving a $G$-set for which we recycle the notation: $$\Hom(H,G)(y)\defequi\Hom(H,G)(\pt_H,y)$$ (and similarly for $H$).
\end{example}
\begin{xca}
  \label{xca:HomZGvsAdG}
  Prove that the $G$-set $\Hom(\ZZ,G):BG\to\Set$ is equivalent to $\Ad_G:BG\to\Set$.

Hint: Use the universal property of the circle to identify the family $\Hom(\ZZ,G)(y)$ with $\sum_{z:BG}\sum_{p:z=z}y=z$ and consider the map to $y=y$ sending $(y,p,q)$ to $q^{-1}p\,q$.
\end{xca}


\begin{example}\label{def:trivGset}
  If $G$ is a group (or \inftygp), and $X$ is a set (or type) then
$$\mathrm{triv}_GX(z)\defequi X$$ is a $G$-set (or $G$-type).  Examples of this sort (regardless of $X$) are called \emph{trivial $G$-sets (or $G$-types)}.
\end{example}

\begin{remark}
  \label{remark:GsetsareGsets}
  A $G$-type $X$ is often presented by focusing on the \emph{underlying type} $X(\pt_G)$  and providing it with a structure relating it to $G$ determining the entire function $X\colon BG\to\UU$.

More precisely, since $BG$ is connected, a $G$-type $X\colon BG\to\UU$ factors over the component $\UU_{(X(\pt_G))}\defequi\sum_{Y:\UU}||Y=X(\pt_G)||$ which contains the point $X(\pt_G)$.  Since $B\Sigma_{X(\pt_G)}\defequi(\UU_{(X(\pt_G))},X(\pt_G))$ the $G$-type $X$ can, without loss of information, be considered as a homomorphism 
$$G\to\Sigma_{X(\pt_G)}% \defequi(\UU_{(X(\pt_G))},X(\pt_G))
$$ from $G$ to the permutation group $\Sigma_{X(\pt_G)}$ of $X(\pt_G)$.

Conversely, if $X$ is any type \emph{and} we have a homomorphism $G\to\Sigma_X$ (in other words, a pointed map $BG\to B\Sigma_{X}%\defequi(\UU_{(X)},X(\pt_G)
$), then the composite
$$BG\to \UU_{(X)}\to \UU$$
is a $G$-type with $X$ exactly the value of $\pt_G$.

However, we must be careful not to focus too much on the underlying type.  
For instance, even though the underlying type of both $\Ad_G$ and $\princ G$ is $\pt_G=\pt_G$, in general  $\Ad_G$ and $\princ G$  are very different $G$-types.  
To drive this point home, compare the illustrations of transport along a $p:\pt_G=\pt_G$ for the two:
$$\xymatrix{\pt_G\ar@{=}[r]^p_\to\ar@{=}[d]_q^\downarrow&\pt_G\ar@{=}[d]^{\Ad_G(p)(q)}_\downarrow\\
\pt_G\ar@{=}[r]^p_\to&\pt_G},\qquad
\xymatrix{\pt_G\ar@{=}[r]^{\refl{\pt_G}}_\to\ar@{=}[d]_q^\downarrow&\pt_G\ar@{=}[d]^{\princ G(p)(q)}_\downarrow\\
\pt_G\ar@{=}[r]^p_\to&\pt_G}$$
A third $G$-type with underlying type $\pt_G=\pt_G$ is $\mathrm{triv}_G(\pt_G=\pt_G)$.
\end{remark}

\begin{xca}
  Prove that if $G$ is an abelian group, then $\Ad_G=\mathrm{triv}_G(\pt_G=\pt_G)$.
\end{xca}
\begin{xca}
  Use that $BG$ is connected to show that if $X$ is a $G$-set, then $X(z)$ is a set for all $z:BG$ (\ie $\prod_{z:BG}\isset(X(z))$).
\end{xca}

\sususe{Transitive $G$-sets}
\label{sec:transitiveGsets}
We end the section with some observations regarding so-called transitive $G$-sets which will be valuable when we move on to discussing subgroups.
Classically, an $\abstr(G)$-set $\mathcal X$ is said to be \emph{transitive} if for all $a,b:\mathcal X$ there is a $g:\mathcal X$ such that $a=g\cdot b$.  In our world this translates to
\begin{definition}
  \label{def:connectedGset}\label{def:transitiveGset}
  A $G$-set $X:BG\to\Set$ is \emph{transitive}\index{transitive $G$-set} if the proposition
$$\mathrm{isTrans}(X)\defequi\prod_{y:BG}\prod_{a,b:X(y)}||\sum_{g:y=y}a=g\cdot b||
$$
holds.
\end{definition}
Note that the mention of $y$ is redundant in the definition: by transport it is enough to demand $\prod_{a,b:X(\pt_G)}||\sum_{g:\pt_G=\pt_G}a=g\cdot b||$, and furthermore, if the proposition holds for one $b$ it holds for all others by composition.  

In other words, $X$ is transitive if and only if there is a $b:X(\pt_G)$ such that %for all $a:X(\pt_G)$ 
the map $-\cdot b:(\pt_G=\pt_G)\to X(\pt_G)$ is surjective.

In the other direction we have
\begin{lemma}
  \label{lem:conistrans}
  A $G$-set  is transitive if and only if the associated set bundle is connected.
\end{lemma}
\begin{proof}
  Consider a $G$-set $X:BG\to\Set$ and the associated \covering $f:\tilde X\to BG$ where $\tilde X\defequi\sum_{y:Y}X(y)$ and $f$ is the first projection.  Now, $\tilde X$ is connected if and only if there is a $y:BG$ and a $b:X(y)$ such that for all $z:BG$ and $a:X(z)$ there is a $g:z=y$ such that $a=X(g)b$.  Since $BG$ is connected, this is equivalent to asserting that there is a $b:X(\pt_G)$ such that for all $a:X(\pt_G)$ there is a $g:\pt_G=\pt_G$ such that $a=g\cdot b\defequi X(g)b$.
\end{proof}




\begin{lemma}
  \label{lem:evisinjwhentransitive}
  Let $X:BG\to\Set$ be a transitive $G$-set, $y:BG$ and $b:X(y)$.  Then the evaluation map
$$\mathrm{ev}:(X=X)\to X(y),\qquad \mathrm{ev}(f)\defequi f_y(b)$$
is injective.
\end{lemma}
\begin{proof}
  For $a:X(y)$, consider an $f:X=X$ with $f_y(b)=a$.  Let $z:BG$ and $c:X(z)$.
For any $g:y=z$ such that $X(g)b=c$ we have $f_z(c)=f_z(X(g)b)=X(g)f_y(b)=X(g)a$: the value does not depend on $f$.  Since we try to prove a proposition we are done. 
\end{proof}



\subsection{The classifying space is the type of torsors}
\label{sec:torsors}
\begin{definition}
  Given a group (or \inftygp) $G$, the type of {\em$G$-torsors} is
$$\typetorsor_G\defequi\sum_{X:\Type_G}||\princ G=X||,$$
where $\princ G$ is the principal $G$-torsor of \cref{def:principaltorsor}.
\end{definition}
\begin{remark}
  Note that if $G$ is a group (as opposed to \aninftygp), then $\princ G$ is a $G${\em-set}, and so for $G$-types $X$, the proposition $||\princ G=X||$ will be empty unless $X$ is a $G$-set too, and so in this case we could more simply have said $\typetorsor_G\defequi\sum_{X:\Set_G}||\princ G=X||.$  
Hence, for $G$ a group, the type of $G$-torsors is just another name for the component of the type of \coverings of $BG$ containing the universal \covering.

Observe that for a group $G$, $\typetorsor_G$ is a connected groupoid (admittedly in a higher universe) and so -- by specifying the base point $\princ G$ -- it represents a group!  Guess which one!
\end{remark}
\begin{definition}
  \label{def:BG2TorsG}
  Let $\pathsp{}\defequi\pathsp{}^G:BG\to(\typetorsor_G,\princ G)$ be the pointed map given by sending $z:BG$ to $\pathsp z$ and $\refl{\pathsp{\pt_G}}:\pathsp{\pt_G}=\princ G$. 
\end{definition}

\begin{example}\label{ex:pathsptransport}
  For $y,z:BG$ we make the induced map
$$\pathsp{}:(y=z)\to (\pathsp y=\pathsp z)
$$
explicit.  For $q:y=z$,  the transport $\pathsp q:\pathsp y=\pathsp z\defequi\prod_{x:BG}\pathsp y(x)=\pathsp z(x)$ is given by sending $p:\pathsp y(x)\defequi (y=x)$ to
$$\pathsp q(p)\defequi pq^{-1}:\pathsp z(x)\defequi(z=x),$$ 
or, in a picture 
$$\xymatrix{y\ar@{=}[r]^q_\to\ar@{=}[d]^{p}_\downarrow&z\ar@{=}[d]^{\pathsp q(p)}_\downarrow\\
x\ar@{=}[r]^{\refl x}_\to&\,x.}$$
\end{example}
\begin{lemma}\label{lem:pathsptransportiseq}
  For  $y,z:BG$ the induced map  (\ie transport) of identity types
$$\pathsp{}:(y=z)\to (\pathsp y=\pathsp z)
$$
is an equivalence.
\end{lemma}
\begin{proof}
 By the induction principle for the identity type,  
a function $f:\pathsp y\to \pathsp z$ is given by its value $f(\refl y):\pathsp zy\defequi (z=y)$ and in general, for $p:y= x$ we have by transport that $f(p)$ is the composite $p\,f(\refl y)=\pathsp{(f(\refl y))^{-1}}(p)$.  
In other words the map $\pathsp{}:(y=z)\to(\pathsp z=\pathsp y)$ is an equivalence.
\end{proof}


\begin{theorem}\label{lem:BGbytorsor}
  If $G$ is a group (or \inftygp), then the function
$$\pathsp{}:BG\to(\typetorsor_G\princ G),\qquad z\mapsto \pathsp z\defequi(x\mapsto(z=_{BG}x))$$
is an equivalence.
Univalence then provides us with an identity 
$$\bar{\pathsp{}}:G=(\typetorsor_G,\princ G)$$ of groups (or $\infty$-groups).\footnote{in the appropriate universe}.
\end{theorem}

\begin{proof}
  Since both $\typetorsor_G$ and $BG$ are connected, it suffices by \cref{lem:eqandcovofconntypes} to show that the induced map
$$\pathsp{}:(y=z)\to(\pathsp y= \pathsp z)
$$
is an equivalence, which is exactly the contents of \cref{lem:pathsptransportiseq}.  
%But this will follow by the very definition of the identity type!  

% For $q:y=z$,  the transport $T(q):\pathsp y=\pathsp z$ is given by  sending $p:y=x$ to
% $$T(q)(p)\defequi pq^{-1}:T(z)(x)\defequi(z=x),$$ 
% or, in a picture 
% $$\xymatrix{y\ar@{=}[r]^q_\to\ar@{=}[d]^{p}_\downarrow&z\ar@{=}[d]^{T(q)(p)}_\downarrow\\
% x\ar@{=}[r]^{\refl x}_\to&\,x.}$$

% %Hence $\bn 1\to T^{-1}(f)=\sum_{p:y=z}T(p)=f$ with value is an equivalence.
% %the same as giving an element in $T(z)(y)\defequi (z=y)$.
\end{proof}
\subsection{Any group is a subgroup of a permutation group}
\label{sec:groupssubperm}


This allows for a cute proof of what is often stated as ``any group is a permutation group'', which in our parlance translates to ``any symmetry is a symmetry of $\UU$'':\footnote{which reminds me of the following: my lecturer in cosmology once tried to publish a paper about rotating black holes, only to have it rejected because it turned out that it was his universe, not the black hole, that was rotating}
\begin{lemma}
  \label{lem:allgpsarepermutationgps}Let $G$ be a group.  Then remembering the base point in the factorization of the pricipal $G$-torsor $\princ G:BG\to\UU$ through the component $\UU_{(\pt_G=\pt_G)}\subseteq\UU$ (c.f.~\cref{remark:GsetsareGsets}) gives an injective homomorphism
$$\alpha_G:\Hom(G,\Sigma_{\pt_G=\pt_G})$$  
of $G$ into a permutation group.
\end{lemma}
\begin{proof}
  The type of injective group homomorphism from $G$ to a group $H$ is equivalent\footnote{((GIVE REF))} to the type of pointed \coverings of $BH$ by $BG$, so we need to show that $\alpha_G:BG\to\UU_{(\pt_G=\pt_G)}$ is a \covering.  
Under the identity $\bar{\pathsp{}}:BG=(\typetorsor_G,\princ G)$ of \cref{lem:BGbytorsor} the function $\alpha$ translates to the evaluation map
$$\xymatrix{
  \typetorsor_G\ar[rr]^-{\mathrm{ev}_{\pt_G}}\ar@{=}[d]&&\Sigma_{\pt_G=\pt_G}\ar@{=}[d]\\
  \underset{E:BG\to\Set}\sum\,\underset{x:BG}\prod||(\pt_G=x)=E(x)||\ar@{}[rr]^-{E\mapsto E(\pt_G)}&&
\,\underset{X:\Set}\sum||(\pt_G=\pt_G)=X||.
}$$
We must show that the preimages $\ev_{\pt_G}(X)$ for $X:\Sigma_{\pt_G=\pt_G}$ are sets.  
This preimage is equivalent to $\sum_{E:BG\to\Set}(X=E(\pt_G))$ (note the absence of a truncation).  We must show that if $(E,p),(F,q):\sum_{E:BG\to\Set}(X=E(\pt_G))$, then 
$$((E,p)=(F,q))=\prod_{x:BG}\sum_{\phi(x):E(x)=F(x)}\phi(\pt_G)=_{E(\pt_G)=F(\pt_G)}qp^{-1}$$ is a set.  
Note that if $r:\pt_G=x$, then $\phi(x)=F(r)qp^{-1}E(r)^{-1}$, or in a picture
$$\xymatrix{&E(pt_G)\ar@{=}[r]^{E(r)}_\to\ar@{=}[dd]_{\phi(pt_G)}^\downarrow&E(x)\ar@{=}[dd]_{\phi(x)}^\downarrow\\
  X\ar@{=}[ur]^p_\to\ar@{=}[dr]^q_\to&&\\
  &F(\pt_G)\ar@{=}[r]^{F(r)}_\to&F(x)},$$
and so $\phi(x)$ (which by nature is independent of such an $r:\pt_G=x$) is uniquely determined by $(E,p)$ and $(F,q)$.  Let us pin this down in our language.
If $\phi,\psi:((E,p)=(F,q))$, we must show that $\phi=\psi$ and since that for $x:BG$ both $E(x)$ and $F(x)$ are sets, it is enough to show that the proposition $\phi(x)=\psi(x)$ is not empty.  Let $f:(\pt_G=x)\to(\phi(x)=\psi(x))$ be given by letting $f(r)$ be the composite of the identities $\phi(x)=F(r)qp^{-1}E(r)^{-1}=\psi(x)$ given above.  
Since $BG$ is connected, $\pt_G=x$ is not empty, and we are done.
\end{proof}

\subsection{Homomorphisms and torsors}
\label{sec:homotor}
In view of the equivalence $\pathsp{}^G$ between $BG$ and $(\typetorsor_G,\princ G)$ of \cref{lem:BGbytorsor} one might ask what a group homomorphism  $f:\Hom(G,H)$ translates to on the level of torsors.  Off-hand, the answer is $(\pathsp{}^H)Bf(\pathsp{}^G)^{-1}$, but we can be more concrete than that.  We do know that for $x:BG$ the $G$-torsor $\pathsp x^G$ should be sent to $\pathsp {Bf(x)}^H$, but how do we express this for an arbitrary $G$-torsor?
\begin{definition}
  \label{def:restrictandinduce}
  Let $f:\Hom(G,H)$ be a group homomorphism.  If $Y:BH\to\Set$ is an $H$-set then the \emph{restriction} $f^*Y$ of $Y$ to $G$ is the $G$-set given by precomposition 
$$f^*Y\defequi Y\, f:BG\to\Set.$$  

If $X:BG\to\Set$ is a $G$-set and $y:BH$ define 
\footnote{OLD: $$f_*X(y)\defequi(\pt_H=y)\times_{\pt_G=\pt_G}X(\pt_G)$$ to be the set quotient of $(\pt_H=y)\times X(\pt_G)$ by the relation $(p,x)\sim(p\, f(q)^{-1},X(q)x)$ for all $q:\pt_G=pt_G$.  The \emph{induced} $H$-set 
$$f_*X:BH\to\Set$$ has value at $y:BH$ the set $f_*X(y)$.}
the induced $H$-type $f_*BH\to\UU$ by
$$f_*X(y)\defequi\sum_{x:BG}(Bf x =y)\times X(x).$$
\end{definition}

For $X$ being the principal $G$-torsor  $\princ G$, the contraction of $\sum_{x:BG}(\pt_G=x)$ induces an equivalence
$$\eta_y:f_*\princ G(y)=\sum_{x:BG}(Bfx=y)\times(\pt_G=x)\simeq (Bfx=y)\oldequiv\princ{Bf(x)}^H(y).$$  The resulting identity $\bar\eta:f_*\princ G=\princ{Bf(x)}^H$ shows that for every $G$-torsor $X$ the $H$-type $f_*X$ is an $H$-torsor, defining a map 
$$f_*:\typetorsor_G\to\typetorsor_H.$$
Summing up
\footnote{OLD:
When $X=\princ G$ we can get a good picture of $f_*X$:  composition gives a map
$$\eta:(\pt_H=y)\times (\pt_G=\pt_G)\to (\pt_H=y),\quad \eta(p,q)=p\,f(q)$$
and the fact that $\eta(p,q)=\eta(p\,f(r)^{-1},r\,q)$ for all $r:\pt_G=\pt_G$ tells us that $\eta$ defines a map $\eta_f(y):f_*\princ G(y)\to\princ H(y)$.  
The map $\iota:\princ H(y)\to f_*\princ G(y)$ defined by $\iota(p)=[p,e_G]$ is an inverse: $\iota\eta_f(y)$ sends $[p,q]$ to $[pf(q)^{-1},e_G]$, which is equal to $[p,q]$; and $\eta_f(y)\iota$ sends $p$ to $p\,f(e_G)^{-1}$, which is equal to $p$.  Hence $\eta_f(y)$ is an equivalence.}
\begin{lemma}
  \label{lem:inducedtorsor}
   Let $f:\Hom(G,H)$ be a group homomorphism.  
If $X$ is a $G$-torsor, then $f_*X$ is an $H$-torsor and the identity 
$\bar{\eta}:f_*\pathsp x^G=\pathsp{Bf(x)}^H$
\footnote{OLD: $\bar{\eta}_f:f_*\pathsp x^G=\pathsp{Bf(x)}^H$ associated to the equivalence $\eta_f$} shows that  
$$\xymatrix{BG\ar[r]^{Bf}\ar[d]^{\pathsp{}^G}&BH\ar[d]^{\pathsp{}^H}\\
\typetorsor_G\ar[r]^{f_*}&\typetorsor_H}$$ 
commutes.
\end{lemma}
\footnote{OLS: \begin{proof}
  If $||X=\princ G||$, then $||f_*X=f_*\princ G||$ and $\eta_f:f_*\princ G=\princ H$, so $f_*$ takes $G$-torsors to $H$-torsors.
\end{proof}
}
\begin{remark}
  \label{rem:inducedGsetfromabstracthomomorphisms}
  Notice that our construction of the induced $G$-set works equally well for a homomorphism $\phi:\Hom^\abstr(\abstr(G),\abstr(H))$: if $X:BG\to\Set$ is a $G$-set, then we define the $H$-set $\phi_*X:BH\to\Set$ by 
$$\phi_*X(y)\defequi(\pt_H=y)\times_{\pt_G=\pt_G}X(\pt_G)$$ to be the set quotient of $(\pt_H=y)\times X(\pt_G)$ by the relation $(p,x)\sim(p\, \phi(q)^{-1},X(q)x)$ for all $q:\pt_G=pt_G$. Just as above, for $X$ the principal $G$-torsor we get an identity  $\eta_\phi:\phi_*\princ G=\princ H$ which, when evaluated at $y:BH$, corresponds under univalence to the equivalence 
$$(\pt_H=y)\times_{\pt_G=\pt_G}(\pt_G=\pt_G)\to (\pt_H=y)$$ 
sending $[p,q]:(\pt_H=y)\times_{\pt_G=\pt_G}(\pt_G=\pt_G)$ to $p\,\phi(q):(\pt_H=y)$.
\end{remark}


\section{Groups concrete or abstract% -- same gem, different wrapping
}
\label{sec:Gsetforabstract}

We use \cref{lem:BGbytorsor} as our inspiration for trying to construct a group from an abstract group.  We define totally analogously the type of torsors for an abstract group.  It will then be a relative simple matter to show that the processes of
\begin{enumerate}
\item forming the abstract group of a group and 
\item taking the group represented by the torsors of an abstract group
\end{enumerate}
 are inverse to each others.

Note that we have not considered an abstract analog of the concept of $\infty$-group, so all we do in this section is set-based.

\begin{definition}
\label{def:abstrGtorsors}
  If ${\agp G}=(S,e,\mu,\iota)$ is an abstract group, a \emph{$\agp G$-set}\index{Gset@$\agp G$-set (of abstract group} is a set $\mathcal X$ together with a homomorphism
$\agp G\to\abstr(\Sigma_{\mathcal X})$
from $\agp G$ to the (abstract) permutation group of $\mathcal X$:
$$Set_{\agp G}^\abstr\defequi \sum_{\mathcal X:\Set}\Hom_\abstr({\agp G},\abstr(\Sigma_{\mathcal X})).$$

The \emph{principal ${\agp G}$-torsor} $\princ {\agp G}^\abstr$ is the ${\agp G}$-set consisting of the underlying set $\mathrm{under}({\agp G})\defequi S$ together with the homomorphism ${\agp G}\to\abstr(\Sigma_{S})$ with underlying function of sets $S\mapsto (S=S)$ given by sending $g:S$ to $\mathrm{ua}(s\mapsto s\cdot g^{-1})$.

The type of \emph{${\agp G}$-torsors} is
$$\typetorsor_{\agp G}^\abstr\defequi\sum_{S:\Set_{\agp G}^\abstr}||\princ {\agp G}=S||.$$
\end{definition}
\begin{example}
  If $G$ is a group we can unravel the definition and see that an $\abstr(G)$-set consists of
  \begin{enumerate}
  \item a set $S$, 
  \item a function $f:(\pt_G=\pt_G)\to (S=S)$ 
  \item such that $f(e_G)=\refl{S}$ and for all $p,q:\pt_G=\pt_G$ we have that $f(p\, q)=f(p)\,f(q)$.
  \end{enumerate}

\end{example}


To help reading the coming proofs we introduce some notation that is redundant, but may aid the memory in cluttered situations:  Let $x,y,z$ be elements in some type, then
\begin{align*}
%  \pre:(x=y)\to ((y=z)=(x=z)),\qquad&\pre(q)(p)\defequi pq\\
  \preinv:(y=x)\to ((y=z)=(x=z)),\quad&\preinv(q)(p)\defequi\pathsp qp\defequi pq^{-1}\\
  \post:(y=z)\to ((x=y)=(x=z)),\quad&\post(p)(q)\defequi\post_pq\defequi pq\\
  %\adjoint:(x=y)\to((x=x)=(y=y)),\qquad&\adjoint(q)(p)\defequi\adjoint_qp\defequi qpq^{-1}
\end{align*}
We recognize $\preinv$ from \cref{lem:pathsptransportiseq} as the induced map of identity types $\pathsp{}\colon (y=z)\to(\pathsp y=\pathsp z)$ evaluated at $x$, while post-composition $\post$ is transport in the family $\pathsp x$, 


\begin{example}\label{ex:qG}
  If $G$ is a group, then for any $x:BG$ the principal $G$-torsor \emph{evaluated at $x$}, \ie the set $\princ Gx\defequi(\pt_G=x)$, has a natural structure of an $\abstr(G)$-set by means of 
$$\preinv:(\pt_G=\pt_G)\to ((\pt_G=x)=(\pt_G=x))$$ and the fact that $\preinv(e_G)\defequi\refl{\pt_G=x}$ and that for $p,q:\pt_G=\pt_G$ we have that  $\preinv(p\,q)=\preinv(p)\preinv(q)$ (\ie if $r\colon \pt_G=x$ we have that 
$$\preinv(p\, q)(r)=r(p\,q)^{-1}=r\,q^{-1}p^{-1}=\preinv(p)\preinv(q)(r)$$  -- demonstrating why we chose $\preinv$: without the inverse this would have gone badly wrong).  

That this $\abstr(G)$-set is an $\abstr(G)$-torsor then follows since $BG$ is connected (any $\pt_G=x$ will serve as a proof of $(\pt_G=x,\preinv,!)=\princ{\abstr(G)}^\abstr$).

Though it sounded like we made a choice ending up with $\preinv$; we really didn't -- it is precisely what happens when you abstract the homomorphism $G\to\Sigma_{\princ G(x)}$: 
you get the function of identity types 
$$(\pt_G=\pt_G)\to (\princ G(x)=\princ G(x))$$ 
which by the very definition of transport for $\princ G$ is $\preinv$. 
\end{example}

\begin{definition}
  If ${\agp G}$ is an abstract group, then the \emph{concrete group $\concr({\agp G})$ associated with ${\agp G}$} is the group (given by the pointed connected groupoid) $(\typetorsor_{\agp G}^\abstr,\princ {\agp G})$.
\end{definition}
We give the construction of \cref{ex:qG} a short name since it will occur in important places.
\begin{definition}
  Let $G$ be a group.  Define the group homomorphism 
 $$q_G:G\to \concr(\abstr(G))$$ defined in terms of the pointed map by the same name
$$q_G:BG\to_* (\typetorsor^\abstr_{\abstr(G)},\princ {\abstr(G)}),\quad q_G(z)=(\princ G(z),\preinv,!).$$
\end{definition}

\begin{lemma}
  \label{lem:Groupsareidentitytypes}%Let ${\agp G}$ be an abstract group.  
For all groups $G$, the pointed function $q_G:G\to\concr(\abstr(G))$ 
is a  equivalence.
\end{lemma}
\begin{proof}
  To prove that $q_G$ is an equivalence it is, by \cref{lem:eqandcovofconntypes}, enough to show that if $x,y:BG$ then the induced map
$$q_G:(x=_{BG}y)\to (q_G(x)=q_G(y))%(\pt_G=x)=_\UU(\pt_G=y))
$$
is an equivalence.
  Now, $q_G(x)=q_G(y)$ is equivalent to the set 
\begin{align*}
  &((\pt_G=x),\preinv)=_{\abstr(G)\text{-set}}((\pt_G=y),\preinv)\\
=&\sum_{f:(\pt_G=x)=(\pt_G=y)}f\preinv=\preinv f
\end{align*}
 ($f\preinv=\preinv f$ is shorthand for $\prod_{q:\pt_G=x}\prod_{p:\pt_G=p}f(pq^{-1})=f(p)q^{-1}$ and the rest of the data is redundant at the level of symmetries) and under these identities $q_G$ is given by 
$$(\post,!):(x=y)\to \sum_{f:(\pt_G=x)=(\pt_G=y)}f\preinv=\preinv f.$$
Given an element
$(f,!):\sum_{f:(\pt_G=x)=(\pt_G=y)}f\preinv=\preinv f$, the preimage 
$(\post,!)^{-1}(f,!)$ is equivalent to the set
$\sum_{r:x=y}(f=\post_r)$.  But if $(r,!),(s,!): \sum_{r:x=y}(f=\post_r)$, then for all $p:\pt_G=x$ we get that $r\,p=f(p)=s\,p$, that is $r=s$, so that the preimage is in fact a proposition.  
To show that the preimage is contractible, it is enough to construct a function $(\pt_G=x)\to \sum_{r:x=y}(f=\post_r)$, and sending $p$ to $f(p)p^{-1}$ will do.
\end{proof}

% \begin{definition}
%   A $G$-torsor is a $G$-set which is isomorphic to the underlying $G$-set of $G$ (write out - avoid conflict of notation wrt $|G|$)
% \end{definition}

% $$, $\pre(q)(p)=pq$
% $\preinv:(y=x)\to ((y=z)=((x=z))$, $\preinv(q)(p)=pq^{-1}$
% $\post:(y=z)\to ((x=y)=((x=z))$, $$
%\footnote{how deeply do we want to integrate univalence?}
\begin{example}
  \label{ex:abstrconcrG}
  Let ${\agp G}=(S,e,\mu,\iota)$ be an abstract group.  
Then the underlying set of $\abstr(\concr({\agp G}))$ is $\princ {\agp G}^\abstr=_{\typetorsor^\abstr_{\agp G}}\princ {\agp G}^\abstr$.  
Unraveling the definitions we see that this set is equivalent to
$$\sum_{p:S=S}\prod_{q,s:S}(p(s\,q^{-1})=p(s)\,q^{-1}).
$$  
Setting $s\defequi e$ and renaming $t\defequi q^{-1}$ in the last equation, we see that $p(t)=p(e)t$; that is $p$ is simply multiplication with an element $p(e):S$.  in other words, the function 
$$r_{\agp G}:S\to  \sum_{p:S=S}\prod_{q,s:S}(p(s\,q^{-1})=p(s)\,q^{-1}),\qquad r_{\agp G}(u)\defequi(u\cdot\,,!)
$$  
is an equivalence of sets, which we by univalence is converted into an identity.  
The abstract group structure of $\abstr(\concr({\agp G}))$ is given by it being the symmetries of $\princ {\agp G}^\abstr$; translated to $\sum_{p:S=S}\prod_{q,s:S}(p(s\,q^{-1})=p(s)\,q^{-1})$ this corresponds via the first projection to the symmetries of $S$. %that of $S=S$ of the first projection.  
This means that we need to know that if $u,v:S$ and consider the two symmetries $u\cdot,v\cdot:S=S$, then their composite (the operation on the symmetry on $S$) is equal to $(u\cdot v)\cdot:S=S$ (the abstract group operation), but this is true by associativity ($u\cdot(v\cdot s)=(u\cdot v)\cdot s$).  That $r_{\agp G}$ also sends $e:S$ to $\refl S$ is clear.
Hence our identity $r_{\agp G}$ underlies an identity of abstract groups
$$r_{\agp G}:{\agp G}=_{\typegroup^\abstr}\abstr(\concr({\agp G})).$$
\end{example}

This shows that every abstract group encodes the symmetries of something essentially unique.  Summing up the information we get
\begin{theorem}
  \label{lem:Groupsareidentitytypes}Let ${\agp G}$ be an abstract group.  
Then
$$\abstr:\typegroup\to\typegroup^\abstr$$ is an equivalence% : \ie the type of groups and the type of abstract groups are equal
.
\end{theorem}
%\begin{proof}
 %  First consider $q_G$.  To prove that $q_G$ is an equivalence it is, by \cref{lem:eqandcovofconntypes}, enough to show that if $x,y:BG$ then the induced map
% $$q_G:(x=_{BG}y)\to (q_G(x)=q_G(y))%(\pt_G=x)=_\UU(\pt_G=y))
% $$
% is an equivalence.\footnote{something to be said for the homotopies vs. base point}  Now, $q_G(x)=q_G(y)$ is equivalent to the set 
% \begin{align*}
%   &((\pt_G=x),\preinv)=_{\abstr(G)-set}((\pt_G=y),\preinv)\\
% =&\sum_{f:(\pt_G=x)=(\pt_G=y)}f\preinv=\preinv f
% \end{align*}
%  ($f\preinv=\preinv f$ is shorthand for $\prod_{q:\pt_G=x}\prod_{p:\pt_G=p}f(pq^{-1})=f(p)q^{-1}$ and the rest of the data is redundant at the level of symmetries) and under these identities $q_G$ is given by 
% $$(\post,!):(x=y)\to \sum_{f:(\pt_G=x)=(\pt_G=y)}f\preinv=\preinv f.$$
% Given an element
% $(f,!):\sum_{f:(\pt_G=x)=(\pt_G=y)}f\preinv=\preinv f$, the preimage 
% $(\post,!)^{-1}(f,!)$ is equivalent to the set
% $\sum_{r:x=y}(f=\post_r)$.  But if $(r,!),(s,!): \sum_{r:x=y}(f=\post_r)$, then for all $p:\pt_G=x$ we get that $r\,p=f(p)=s\,p$, that is $r=s$, so that the preimage is in fact a proposition.  To show that the preimage is contractible, it is enough to construct a function $(\pt_G=x)\to \sum_{r:x=y}(f=\post_r)$, and sending $p$ to $f(p)p^{-1}$ will do.
%\end{proof}

\section{Homomorphisms, abstract vs.~concrete}
\label{sec:homabsisconcr}

Now that we know that the type of groups is equal to the type of abstract groups, it is natural to ask if the notion of group homomorphisms also coincide.  

They do, and we provide two independent and somewhat different arguments.  Translating from group homomorphisms to abstract group homomorphisms is easy: if $G$ and $H$ are groups, then we defined 
$$\abstr:\Hom(G,H)\to\Hom^\abstr(\abstr(G),\abstr(H))$$
in \cref{def:abstrisfunctor} as the function which takes a homomorphism, aka a pointed map $f=(Bf_\div,p_f):BG\to_*BH$ to the induced map of identity types 
$$f^\abstr\defequi \mathrm{ad}_{p_f}\ap{Bf_\div}:(\pt_G=\pt_G)\to(\pt_H=\pt_H)$$
 together with the proofs that this is an abstract group homomorphism from $\abstr(G)$ to $\abstr(H)$, c.f~\cref{def:grouphomomaxioms}.


Going back is somewhat more involved, and it is here we consider two approaches.
The first is a compact argument showing directly how to reconstruct a pointed map $Bf:BG\to_*BH$ from an abstract group homomorphism from $\abstr(G)$ to $\abstr(H)$, the second translates back and forth via our equivalence between abstract and concrete groups.



The statement we are after is


\begin{lemma}
  \label{lem:homomabstrconcr}
  If $G$ and $H$ are groups, then 
$$\abstr:\Hom(G,H)\to\Hom^\abstr(\abstr(G),\abstr(H))$$
is an equivalence.
\end{lemma}
and the next two subsections offer two proofs.



\sususe{``Delooping'' a group homomorphism}
\label{sec:thierrysdelooping}
We now explore the first approach.\footnote{Note to self: make into a lemma and say ``group'' etc}

\begin{proof}
  Let $(X,a)$ and $(Y,b)$ be two pointed connected $1$-types.
We suppose given a group morphism
$$f:a = a\rightarrow b = b$$
and we explain how to build a map $g:X \rightarrow Y$ with
a path $p:b = g(a)$ such that $p f(\omega) = g(\omega) p$
for all $\omega:a = a$. (So $g$ is a ``delooping'' of $f$.)

\medskip

Let us assume the problem solved. The map $g:X\rightarrow Y$
will then send any path $\alpha:a = x$ to a path $g(\alpha):g(a) = g(x)$
and so we get a family of paths $p(\alpha) = g(\alpha) p$ in $b = g(x)$ such that
$$p(\alpha\omega) = g(\alpha)g(\omega)p
  = g(\alpha)pf(\omega) = p(\alpha)f(\omega)$$
for all $\omega:a = a$ and $\alpha : a = x$.

\medskip

This suggests to introduce the following family
$$
C(x)~:=~ \sum_{y:Y}\sum_{p:(a=x)\rightarrow (b = y)}~~~\prod_{\omega:a=a}\prod_{\alpha:a=x}~
 p(\alpha\omega) = p(\alpha)f(\omega)
$$

 An element of $C(x)$ has three components, the last component being
 a proof since $Y$ is a $1$-type.

 The type $C(a)$ has a simpler description. An element of $C(a)$ is
 a pair $y,p$ such that $p(\alpha\omega) = p(\alpha)f(\omega)$ for
 $\alpha$ and $\omega$ in $a=a$. Since $f$ is a group morphism, this condition
 can be simplified to $p(\omega) = p(1_a)f(\omega)$, and the map $p$
 is completely determined by $p(1_a)$.
 Thus $C(a)$ is equal to $\sum_{y:Y}b = y$ and is contractible.

 \medskip

 It follows that we have
 $$
 \prod_{x:X}~ a = x\rightarrow\iscontr~ C(x)
 $$
and so, since $\iscontr~C(x)$ is a proposition
 $$
 \prod_{x:X}~ \Trunc{a = x}\rightarrow\iscontr~ C(x)
 $$
 Since $X$ is connected, we have
 $\prod_{x:X}\iscontr~ C(x)$
 and so, in particular, we have an element of $\prod_{x:X}C(x)$.

 We get in this way a map $g:X\rightarrow Y$
 together with a map $p:(a=x)\rightarrow (b = g(x))$ such that
 $p (\alpha\omega) = p(\alpha) f(\omega)$
 for all $\alpha$ in $a=x$ and $\omega$ in $a=a$.
We have, for $\alpha:a=x$
$$\prod_{x':X}\prod_{\lambda:x=x'}~p(\lambda\alpha) = g(\lambda)p(\alpha)$$
since this holds for $\lambda = 1_x$.
In particular, $p(\omega) = g(\omega)p(1_a)$.

We also have $p(\omega) = p(1_a)f(\omega)$, hence 
$p(1_a)g(\alpha) =  f(\alpha)p(1_a)$
for all $\alpha:a=a$ and we have found a delooping of $f$.
\end{proof}


\sususe{The concrete vs. abstract homomorphisms via torsors.}
\label{sec:absconctorsor}

The second approach to \cref{lem:homomabstrconcr} is as follows:


\begin{proof}
%\newcommand{\we}{\overset\sim\to}
  The equivalence of $\pathsp{}^G:BG\we(\typetorsor_G,\princ G)$ of \cref{lem:BGbytorsor} gives an equivalence
$$\pathsp{}:\Hom(G,H)\oldequiv (BG\to_*BH)\we((\typetorsor_G,\princ G)\to_*(\typetorsor_H,\princ H))
$$
Consider the map
$$A:((\typetorsor_G,\princ G)\to_*(\typetorsor_H,\princ H)\to \Hom^\abstr(\abstr(G),\abstr(H))$$
given by letting $A(f,p)$ be the composite 
$$\xymatrix{(\pt_G=\pt_G)\ar@{=}[d]^{\pathsp{}^G}_\downarrow&&&
  (\pt_H=\pt_H)\ar@{=}[d]^{\pathsp{}^H}_\downarrow\\
  (\princ G=\princ G)\ar[r]^-f&
  (f\princ G=f\princ G)\ar@{=}[rr]^-{q\mapsto p^{-1}qp}_\to&&
  (\princ H=\princ H)
}$$
(together with the proof that this is an abstract group homomorphism).  We see that we have shown the desired result if we prove instead $A$ is an equivalence.  The reason to complicate $\abstr$ this way is that it gets easier to write out a homotopy inverse.

If $(\phi,!):\Hom^\abstr(\abstr(G),\abstr(H))$ and $X:BG\to\Set$ is a $G$-torsor, recall the induced $H$-torsor $\phi_*X$ from \cref{rem:inducedGsetfromabstracthomomorphisms} and the identity $\eta_\phi:\phi_*\princ G=\princ H$. 
 %  let
%$\phi_*X:BH\to\Set$ be the $H$-set given by sending $z:BH$ to the set 
%$$\phi_*X(z)\defequi(\pt_H=z)\times_{\pt_G=\pt_G}X(\pt_G). \footnote{here I need set-quotients: $(\pt_H=z)\times_{\pt_G=\pt_G}X(\pt_G)$ is the coequalizer of the two maps $(\pt_H=z)\times X(\pt_G)\gets (\pt_H=z)\times{(\pt_G=\pt_G)}\times X(\pt_G)$ sending $(p,q,x)$ to $(p\,f(q),x)$ and $(p,X(q) x)$. I use $[p,x]$ to denote an element of this type}$$
%For $X$ the principal $G$-torsor we get an identity  $p_\phi:\phi_*\princ G=\princ H$ which, when evaluated at $z:BH$, corresponds under univalence to the equivalence 
%$$(\pt_H=z)\times_{\pt_G=\pt_G}(\pt_G=\pt_G)\to (\pt_H=z)$$ 
%sending $[p,q]:(\pt_H=z)\times_{\pt_G=\pt_G}(\pt_G=\pt_G)$ to $p\,\phi(q):(\pt_H=z)$.
Let 
$$B: \Hom^\abstr(\abstr(G),\abstr(H))\to ((\typetorsor_G,\princ G)\to_*(\typetorsor_H,\princ H)$$
be given by $B(\phi,!)=(\phi_*X,\eta_\phi)$

We show that $A$ and $B$ are inverse equivalences.  Given an abstract group homomorphism $(\phi,!):\Hom^\abstr(\abstr(G),\abstr(H))$, then $AB(\phi,!)$ has as underlying set map
$$\xymatrix{(\pt_G=\pt_G)\ar@{=}[d]^{\pathsp{}^G}_\downarrow&&&
  (\pt_H=\pt_H)\ar@{=}[d]^{\pathsp{}^H}_\downarrow\\
  (\princ G=\princ G)\ar[r]^-{\phi_*}&
  (\phi_*\princ G=\phi_*\princ G)\ar@{=}[rr]^-{q\mapsto \eta_\phi^{-1}q\eta_\phi}_\to&&
  (\princ H=\princ H),
}$$
and if we start with a $g:(\pt_G=\pt_G)$, then $\pathsp{}^G$ sends it to $\pathsp g^G\oldequiv\preinv (g)$.  Furthermore, $\phi_*\preinv (g)$ is $[\id,\preinv (g)]$ which is sent to $\preinv (\phi(g))$ in $\princ H=\princ H$ which corresponds to $\phi(g):(\pt_H=\pt_H)$ under $\pathsp{}^H$.  In other words, $AB(\phi,!)=(\phi,!)$.  The composite $BA$ is similar.
\end{proof}

\section{$G$-sets vs $\abstr(G)$-sets}
\label{sec:Gsetsabstrconcr}

Given a group $G$ it should by now come as no surprise that the type of $G$-sets is equivalent to the type of $\abstr(G)$-sets.

Recall from \cref{def:abstrGtorsors} that the type of $\abstr(G)$-set is
$$Set_{\abstr(G)}^\abstr\defequi \sum_{\mathcal X:\Set}\Hom_\abstr({\abstr(G)},\abstr(\Sigma_{\mathcal X})).$$
According to \cref{lem:homomabstrconcr}
$$\abstr:\Hom(G,\Sigma_{\mathcal X})\to\Hom^\abstr(\abstr(G),\abstr(\Sigma_{\mathcal X}))$$
is an equivalence, where the group $\Sigma_{\mathcal X}$ (as a pointed connected groupoid) is the component of type $\Set$, pointed at $\mathcal X$.  The component information is moot since we're talking about pointed maps from $BG$ and we see that $\Hom(G,\Sigma_{\mathcal X})$ is equivalent to $\sum_{F:BG_\div\to\Set}(\mathcal X=F(\pt_G))$.  Finally, 
$$\mathrm{pr}:\sum_{\mathcal X}\sum_{F:BG_\div\to\Set}(\mathcal X=F(\pt_G))\we 
%\sum_{F:BG_\div\to\Set}\sum_{\mathcal X}(\mathcal X=F(\pt_G))\we 
(BG_\div\to\Set),\quad \mathrm{pr}(\mathcal X,F,p)\defequi F
$$
is an equivalence (since $\sum_{\mathcal X}(\mathcal X=F(\pt_G))$ is contractible).  
Backtracking these equivalences we see that we have established
\begin{lemma}
  \label{lem:actionsconcreteandabstract}
  Let $G$ be a group.  Then the map
  $$\ev_{\pt_G}:\Set_G\to\Set^\abstr_{\abstr(G)},\qquad \ev_{\pt_G}(X)\defequi(X(\pt_G),a_X)
$$
is an equivalence, where the action $a_X:\Hom^\abstr(\abstr(G), \abstr(\Sigma_{X(\pt_G)}))$ is given by transport $X^=:(\pt_G=\pt_G)\to (X(\pt_G)=X(\pt_G))$.
\end{lemma}
If $X$ is a $G$-set, $g:\pt_G=\pt_G$ and $x:X(\pt_G)$, we seek forgiveness for writing $g\cdot x:X(\pt_G)$ instead of $\cast(a_X(g))(x)$.\footnote{and I ask forgiveness for strongly disliking the use of ``$\cast$'' as a name for some tacitly understood map!}

\begin{example}
  \label{ex:abstrandconj}
  Let $H$ and $G$ be groups.  Recall that the set of homomorphisms from $H$ to $G$ is a $G$-set in a natural way:
$$\Hom(H,G):BG\to\Set,\quad \Hom(H,G)(y)\defequi \sum_{F:BH_\div\to BG_\div}(y=F(\pt_H)).$$

What abstract $\abstr(G)$-set does this correspond to?
In particular, under the equivalence $\abstr:\Hom(H,G)\to\Hom^\abstr(\abstr(H),\abstr(G))$, what is the corresponding action of $\abstr(G)$ on the abstract homomorphisms?
%$\Hom^\abstr(\abstr(H),\abstr(G))$?  

The answer is that $g:\pt_G=\pt_G$ acts on $\Hom^\abstr(\abstr(H),\abstr(G))$ by postcomposing with conjugation $c^g$ by $g$ as defined in \cref{ex:conjhomo}.  

Let us spell this out in some detail:
If $(F,p):\Hom(H,G)(\pt_G)\defequi
 \sum_{F:BH_\div\to BG_\div}(\pt_G=F(\pt_H))$ and $g:\pt_G=\pt_G$, then $g\cdot(F,p)\defequi(F,p\,g^{-1})$.  If we show that the action of $g$ sends $\abstr(F,p)$ to $c^g\circ\abstr(F,p)$ we are done.

Recall that $\abstr(F,p)$ consists of the composite 
$$\xymatrix{(\pt_H=\pt_H)\ar[r]^-{F^=}&(F(\pt_H)=F(\pt_G))\ar[rr]^-{t\mapsto p^{-1}t\,p}&&(\pt_G=\pt_G)},$$ 
(\ie $\abstr(F,p)$ applied to $q:\pt_H=\pt_H $ is  $p^{-1}F^=(q)\,p$)  together with the proof that this is an abstract group homomorphism.  
We see that $\abstr(F,p\,g^{-1})$ is given by conjugation:
$q\mapsto(p\,g^{-1})^{-1}F^=(q)\,(p\,g^{-1})=g\,(p^{-1}F^=(q)\,p)\,g^{-1}$, or in other words $c^g\circ\abstr(F,p)$.
\end{example}
For reference we list the conclusion of this example as a lemma''
\begin{lemma}\label{lem:abstrandconj}
  If $H$ and $G$ are groups, then the equivalence of \cref{lem:actionsconcreteandabstract} sends the $G$-set $\Hom(H,G)$ to the $\abstr(G)$-set $\Hom^\abstr(\abstr(H),\abstr(G))$ with action given by postcomposing with conjugation by elements of $\abstr(G)$.
\end{lemma}

If $f:\Hom(G,G')$ is a homomorphism, then precomposition with $Bf:BG\to BG'$ defines a map $$f^*:(G'\text{-}\Set)\to(G\text{-}\Set).$$
We will have the occasion to use the following result which essentially says that if $f:\Hom(G,G')$ is a ``surjective homomorphism'', then $f^*$ imbeds the type of $G'$-sets as some of the components of the type of $G$-sets.
\begin{lemma}
  \label{lem:epifullyfaithful}
  Let $G$ and $G'$ be groups and let $f:\Hom(G,G')$ be a homomorphism.  If the induced map $f:(\pt_G=\pt_G)\to(\pt_{G'}=\pt_{G'})$ is surjective (c.f.~\cref{def:injection}), then the map $f^*:(G'\text{-}\Set)\to(G\text{-}\Set)$ (induced by precomposition with $Bf:BG\to BG'$) is ``fully faithful'' in the sense that if $X,Y$ are $G'$-sets, then
$$f^*:(X=Y)\to(f^*X=f^*Y)
$$
is an equivalence.
\end{lemma}
\begin{proof}
  Evaluation at $\pt_G$  yields an injective map 
$$\mathrm{ev}_{\pt_G}:(f^*X=f^*Y)\to(X(f(\pt_G)=Y(f(\pt_G)))$$ and the composite 
$$\mathrm{ev}_{\pt_G}f^*=\mathrm{ev}_{f(\pt_G)}:(X=Y)\to(X(f(\pt_G)=Y(f(\pt_G)))$$
 is the likewise injective, so $f^*:(X=Y)\to(f^*X=f^*Y)$ is injective. 

For surjectivity, let $F':f^*X=f^*Y$ and write, for typographical convenience, $a:X(f(\pt_G)=Y(f(\pt_G))$ for $\mathrm{ev}_{\pt_G}F'\defequi F'_{\pt_G}$.  
By the equivalence between $G$-sets and $\abstr(G)$-sets, $F'$ is uniquely pinned down by $a$ and the requirement that for all $g'=f(g)$ with $g:\pt_G=\pt_G$ the diagram 
$$\xymatrix{X(f(\pt_G))\ar@{=}[r]^{X({g'})}\ar@{=}[d]_{a}&
  X(f(\pt_G))\ar@{=}[d]_{a}\\
  Y(f(\pt_G))\ar@{=}[r]^{Y({g'})}&Y(f(\pt_G))}
$$
commutes.  Likewise, (using transport along the identity $p_f:\pt_{G'}=f(\pt_G)$) an $F:X=Y$ in the preimage of $a$ is pinned down by the commutativity of the same diagram, but with $g':f(\pt_G)=f(\pt_G)$ arbitrary (an a priori more severe requirement, again reflecting injectivity).   However, when $f:(\pt_G=\pt_G)\to(\pt_{G'}=\pt_{G'})$ is surjective these requirements coincide, showing that $f^*$ is an equivalence.


% Fix for the moment an  $a:X(f(\pt_G)=Y(f(\pt_G))$

% Now, by transport along the identity $p_f:\pt_{G'}=f(\pt_G)$ and the equivalence between $G'$-sets and $\abstr(G')$-sets, an identity $F':X=Y$ of $G'$-sets is uniquely pinned down by an identity $F'_{f(\pt_G)}:X(f(\pt_G)=Y(f(\pt_G))$ together with the proposition that for all $g':f(\pt_G)=f(\pt_G)$ the diagram $$\xymatrix{X(f(\pt_G))\ar@{=}[r]^{X_{g'}}\ar@{=}[d]_{F'_{f(\pt_G)}}&
%   X(f(\pt_G))\ar@{=}[d]_{F'_{f(\pt_G)}}\\
%   Y(f(\pt_G))\ar@{=}[r]^{Y_{g'}}&Y(f(\pt_G))}
% $$
% commutes.  Likewise, an identity $F:f^*X=f^*Y$ is given by exactly the same data, except that the diagram is only required to commute for $g'=f(g)$ for all $g:\pt_G=\pt_G$.  But when $f:(\pt_G=\pt_G)\to(\pt_{G'}=\pt_{G'})$ these requirements coincide.


% ; $F:X=Y$ is in the preimage of $a:X(f(\pt_G)=Y(f(\pt_G))$ if and only if $a=F_{f(\pt_G)}$ and for all $g':f(\pt_G)=f(\pt_G)$ the diagram
% $$\xymatrix{X(f(\pt_G))\ar@{=}[r]^{X_{g'}}\ar@{=}[d]_{F_{f(\pt_G)}}&
%   X(f(\pt_G))\ar@{=}[d]_{F_{f(\pt_G)}}\\
%   Y(f(\pt_G))\ar@{=}[r]^{Y_{g'}}&Y(f(\pt_G))}
% $$
% commutes.  However, since $f$ is surjective there is a $g:\pt_G=\pt_G$ so that $g'=f(g)$.  Therefore, anything in $f^*X=f^*Y$ which is in the preimage of $a$ is in the image of $f^*:X=Y$ and we have shown that $f^*$ is also a surjection.
\end{proof}



\section{Sums of groups}
\label{sec:coprod}
We have seen how the group of integers $\ZZ=(S^1,\base)$ synthesizes the notion of one symmetry with no relations: every symmetry of the circle is of the form $\Sloop^n$ for some unique $n$.  Also, given any group $G=\aut_A(a)$, the set $a=a$ of symmetries of $a$ corresponds to the set of homomorphisms $\ZZ\to G$, \ie to pointed functions $(S^1,\base)\to_*(A,a)$ by evaluation at $\Sloop$.  What happens if we want to study more than one symmetry at the time?  

For instance, is there a group $\ZZ\vee%\boxplus
\ZZ$ so that for any group $G=\aut_A(a)$ a homomorphism $\ZZ\vee%\boxplus
\ZZ\to G$ corresponds to {\bf two} symmetries of $a$?  
At the very least, $\ZZ\vee\ZZ$ itself would have to have two symmetries and these two can't have any relation, since in a general group $G=\aut_A(a)$ there is a priori no telling what the relation between the symmetries of $a$ might be.  
Now, \emph{one} symmetry is given by a pointed function $(S^1,\base)\to_*(A,a)$ and so a \emph{pair} of symmetries is given by a function $f:S^1+S^1\to A$ with the property that $f$ sends each of the base points of the circles to $a$.  But $S^1+S^1$ is not connected, and so not a group.  To fix this we take the clue from the requirement that both the base points were to be sent to a common base point and \emph{define} $S^1\vee S^1$ to be what we get from $S^1+S^1$ when we \emph{insert an identity} between the two basepoints.
$$\xymatrix{\base\ar@(ul,dl)[]|{\Sloop}\ar@{.>}[rr]^{\text{identify!}}&&\base\ar@(ur,dr)[]|{\Sloop}}
$$
The amazing thing is that this works -- an enormous simplification of the classical construction of the ``free products'' or ``amalgamated sum'' of groups.  We need to show that the ``wedge'' $S^1\vee S^1$ is indeed a group, and this proof simultaneously unpacks the classical description.

% \begin{definition}
%   \label{def:wedge}
%   Let $(A_1,a_1)$ and $(A_2,a_2)$ be pointed types.  Their wedge is the pointed type $(A_1\vee A_2,a_{12})$ given as a higher inductive type\footnote{how/where discussed?} by
%   \begin{enumerate}
%   \item functions $i_1:A_1\to A_1\vee A_2$ and $i_2:A_2\to A_1\vee A_2$
%   \item an element $a_{12}: A_1\vee A_2$ (where we point the type),
%   \item identities $g_1:i_1a_1=a_{12}$ and $g_2:i_2a_2=a_{12}$.
%   \end{enumerate}
%   The function 
% $$i^g_1:(a_1=_{A_1}a_1)\to(a_{12}=_{A_1\vee A_2}a_{12})$$ 
% is defined by $i^g_1(p)\defequi g_1i_1(p)g_1^{-1}$, and likewise $i_2^g(q)\defequi g_2i_2(q)g_2^{-1}$.
% \end{definition}
% ((PICTURE))

% \begin{lemma}
%   \label{lem:wedgeofgpoidisgpoid}
%   Let $\aut_{A_1}(a_1)$ and $\aut_{A_2}(a_2)$ be decidable groups, then the wedge sum $\aut_{A_1\vee A_2}(a_{12})$ is a decidable group.
% \end{lemma}
% \begin{proof}
% That ${A_1\vee A_2}$ is connected follows by transitivity of identity, passing through the identities $g_1$ and $g_2$ in the wedge if necessary.

% We must prove that the wedge is a groupoid, \ie that all identity types are sets, which we do by giving an explicit description of the universal \covering.  The idea is that an identity in $a_{12}=x$ can be factored into a string of identities, each lying solely in $A_1$ or in $A_2$.  We define a family of sets consisting of exactly such strings of identities --  it is a set since $A_1$ and $A_2$ are groupoids -- and prove that it is equivalent to the family $P(x)\defequi(a_{12}=_{A_1\vee A_2}x)$ which consequently must be a family of sets.

%   We use the notation of \cref{def:wedge} freely, and for ease of notation, let $a_{2k+i}\defequi a_i$ for $i=1,2$, $k:\NN$.
% Define families of sets
% $$C_i:A_i\to\Set,\qquad i=1,2$$
% by 
% $$C_i(x)\defequi(a_i=_{A_i}x)\times\sum_{n:\NN}\prod_{1\leq k\leq n}\sum_{p_k:a_{i+k}=% _{A_{i+k}}
%   a_{i+k}}(p_k\neq\refl {a_{i+k}})$$
% when $x:A_i$.  Note that $p_k\neq\refl{a_{i+k}}$ makes sense and is a proposition since our groups are decidable; we leave it out when naming elements.  Also, set
% $$C(a_{12})\defequi \sum_{i:\bn 1}(a_i=a_i)\times\sum_{n:\NN}\prod_{1\leq k\leq n}\sum_{p_k:a_{i+k}=% _{A_{i+k}}
%   a_{i+k}}(p_k\neq\refl {a_{i+k}})
% $$
% Define $C(g_i):C_i(a_1)\to C(a_{12})$ by
% $$C(g_i)(p_0,n,p_1\dots,p_n)=
% \begin{cases}
%   (1-i,p_1,n-1,p_2\dots,p_n)& \text{ if }p_0=\refl{a_i}\\
%   (i,\refl{a_i},p_0,n+1,p_1\dots,p_n)& \text{ if }p_0\neq\refl{a_1}.
% \end{cases}
% $$
% $C_{12}$ is ((obviously or write out)) an equivalence, and so the triple $(C_1,C_2,C_{12})$ defines a family
% $$C:A_1\vee A_2\to\Set.$$
% We will show that $C$ is equivalent to $P\defequi \pathsp{a_{12}}$, which is given by $P(x)=(a_{12}=x)$, and so that the identity types of the wedge are sets.

% One way is the ``inclusion''; more precisely, 
% $$\alpha:\prod_{x:A_1\vee A_2}(P(x)\to C(x))$$ is given by letting identities be considered as strings of length zero: $\alpha_i(i_ia)(p)=(0,p):C_i(a)$.  This is well defined since $\alpha_2(i_2a_2)(gpg^{-1})=C_{12}\alpha_1(i_1a_1)(p)$ ((is this how you'd say this?  Feel free to fix.  Remember that $C(x)$ is a set)).
% The other way, 
% $$\beta:\prod_{x:A_1\vee A_2}(C(x)\to P(x)),$$ is given by composing the identities, using the glue $g_1$ and $g_2$ to make their ends meet: $\beta_1(n,p_0,\dots,p_n,!)\defequi i_1(p_0)g^{-1}i_2^g(p_1)i_1^g(p_2) \dots i_{n+1}^g(p_n)$. % (ending in $\dots gi_1(p_n)$ if $n$ is even and $\dots g^{-1}i_2(p_n)g$ if $n$ is odd) 
% and likewise for $\beta_2$ and the glue ((write)).

% That $\beta\alpha(p)=p$ follows by path induction: it is enough to prove it for 
% $p\defequi\refl{}$ ((here the assymmetry of our definition makes saying this slightly awkward since the basepoint is in $i_1A_1$; fix)).  That $\alpha\beta(n,p_0,\dots,p_n)=(n,p_0,\dots,p_n)$ follows by induction on $n$ ((write)).
% \end{proof}

We start by giving a definition of the wedge construction which is important for pointed types in general and then prove that the wedge of two groups is a group whose symmertries are arbitrary ``words'' in the original symmetries.

\begin{definition}
  \label{def:wedge}
  Let $(A_1,a_1)$ and $(A_2,a_2)$ be pointed types.  The \emph{wedge}\index{wedge of pointed types} is the pointed type $(A_1\vee A_2,a_{12})$ given as a higher inductive type\footnote{how/where discussed?} by
  \begin{enumerate}
  \item functions $i_1:A_1\to A_1\vee A_2$ and $i_2:A_2\to A_1\vee A_2$
  \item an identity $g:i_1a_1=i_2a_2$.
  \end{enumerate}
We point this type at $a_{12}\defequi i_1a_1$.
  The function 
$$i^g_2:(a_2=_{A_2}a_2)\to(a_{12}=_{A_1\vee A_2}a_{12})$$ 
is defined by $i^g_2(p)\defequi g^{-1}i_2(p)g$, whereas (for notational consistency only) we set $i_1^g\defequi i_1:(a_1=_{A_1}a_1)\to(a_{12}=_{A_1\vee A_2}a_{12})$.
Simplifying by writing $i:A_1+A_2\to A_1\vee A_2$ for the function given by $i_1$ and $i_2$ (with basepoints systematically left out of the notation), the induction priciple is
$$\prod_{C:(A_1\vee A_2)\to\UU}\sum_{s:\prod_{a:A_1+A_2}Ci(a)}%\sum_{s_2:\prod_{a:A_2}Ci_2(a)}
((s(a_1)=C(g^{-1})s(a_2))\,\to\,\prod_{x:(A_1\vee A_2)}C(x)).$$
\end{definition}


Unraveling the induction principle we see that if $B$ is a pointed type, then a  pointed function $f:A_1\vee A_2\to_* B$ is given by providing pointed functions $f_1:A_1\to_* B$ and $f_2:A_2\to_* B$  -- the identity $f_1(a_1)=f_2(a_2)$ which seems to be missing is provided by the requirement of the functions being pointed.  For the record
\begin{lemma}
  \label{lem:univvee}
  If $B$ is a pointed type, then the function 
  $$i^*(A_1\vee A_2\to_*B)\to(A_1\to_*B)\times(A_2\to_*B),\qquad i^*(f)=(fi_1,fi_2)
$$
is an equivalence.
\end{lemma}

Here is a picture of $i_2^g(p)$: it is the symmetry of the base point $a_{12}\defequi i_1a_1$ you get by \emph{first} moving to $i_2a_2$ with $g$, \emph{then} travel around with $p$ ($i_2p$, really) and finally go home to the basepoint with the inverse of $g$.
% $$\xymatrix{i_1a_1\ar@/^/[rr]^{g}&&i_2a_2\,\,\,\ar@/^/[ll]^{g^{-1}}\ar@(ur,dr)[]^{p}}
% $$

% $$\xy (-20,20)*+{A};(0,20)*+{B}
% **\crv{}
% \endxy$$
% $$% \xy (-20,20)*+{i_1a_1\,\,};(0,20)*+{}
% % **\crv{}\endxy
% % \xy (0,20)*+{i_2a_2};(0,20)*+{}
% % **\crv{(20,30)&(0,40)&(-20,30)}
% % \endxy
% \xy (-20,20)*+{a_{12}\,\,};(-20,20)*+{}
% **\crv{(10,20)&(-20,35)&(0,45)&(20,35)&(15,20)&(-10,20)}
% %?>*\dir{>}
% ?(.38)*{} *!LD!/^3pt/{>}
% ?(.95)*{} *!LD!/^-15pt/{g^{-1}}
% ?(.03)*{} *!LD!/^-5pt/{g}
% ?(.55)*{} *!LD!/^-7pt/{i_2p}
% \endxy
% $$
% $$\xy (0,20)*+{A};(60,0)*+{B}
% **\crv{(20,20)&(30,20)&(50,-20)&(60,-10)}
%  ?<*\dir{<} ?>*\dir{>}
%  ?(.65)*{\oplus} *!LD!/^-5pt/{x}
%  ?(.65)/12pt/*{\oplus} *!LD!/^-5pt/{x’}
%  ?(.28)*=0{\otimes}-/40pt/*+{Q}="q"
%  +/100pt/*+{P};"q" **\dir{-}
% \endxy
% $$
$$
\xy (-20,20)*+{};(-20,20)*+{}
**\crv{(15,20)&(18,20)&(-10,35)&(10,45)&(25,30)&(20,19)&(0,20)}
%?>*\dir{>}
?(0)*{} *!LD!/^-20pt/{i_1A_1}
?(.45)*{} *!LD!/^2pt/{>}
?(.95)*{} *!LD!/^-15pt/{g^{-1}}
?(.03)*{} *!LD!/^-5pt/{g}
?(.55)*{} *!LD!/^-7pt/{i_2p}
?(.65)*{} *!LD!/^-30pt/{i_2A_2}
?(.87)*{} *!LD!/^-12pt/{i_2a_2}
?(.86)*{} *!LD!/^-2pt/{\bullet}
?(1)*{} *!LD!/^-2pt/{\bullet}
?(1)*{} *!LD!/^-12pt/{a_{12}}
\endxy
$$

We now prove that wedges of decidable groups are decidable groups.   The idea is that an identity in $a_{12}=x$ can be factored into a string of identities, each lying solely in $A_1$ or in $A_2$.  We define a family of sets consisting of exactly such strings of identities --  it is a set since $A_1$ and $A_2$ are groupoids -- and prove that it is equivalent to the family $P(x)\defequi(a_{12}=_{A_1\vee A_2}x)$ which consequently must be a family of sets.
We need to be able to determine whether a symmetry is reflexivity or not, but once we know that, the symmetries of the base point in the wedge are then given by ``words $p_0p_1\dots p_n$'' where the $p_j$ alternate between being symmetries in the first or the second group, and none of the $p_j$ for positive $j$ are allowed to be reflexivity% : effectively a symmetry in the wedge can be decomposed into composites of symmetries in each of the groups
.  Note that there order of the $p_j$s is not negotiable: if I shuffle them I get a new symmetry.
\begin{lemma}
  \label{lem:wedgeofgpoidisgpoid}
  Let $\aut_{A_1}(a_1)$ and $\aut_{A_2}(a_2)$ be decidable groups, then the wedge sum $\aut_{A_1\vee A_2}(a_{12})$ is a decidable group.  

Let $C_1$ be the set of strings $(p_0,n,p_1,\dots,p_n)$ with $n:\NN$ and, for $0\leq j\leq n$ 
\begin{itemize}
\item $p_{j}:a_1=a_1$ for even $j$ 
\item $p_{j}:a_2=a_2$ for odd $j$ and 
\item $p_j$ is not reflexivity for $j$ positive (makes sense and is a proposition since our groups are decidable).
\end{itemize}
  Then the function given by composition in $a_{12}=a_{12}$
$$\beta:C_1\to(a_{12}=a_{12}),\qquad\beta(p_0,n,p_1,\dots p_n)\defequi i_1^gp_0i_2^gp_1i_1^gp_2\dots i_?^gp_n$$ 
(where $i_?^gp_n$  is $i_1^gp_n$ or $i_2^gp_n$ according to whether $n$ is even or odd) is an equivalence.
\end{lemma}
\begin{proof}
That the wedge is connected follows by transitivity of identity, if necessary passing through the identity $g:i_1a_1=i_2a_2$ in the wedge.

We must prove that the wedge is a groupoid, \ie that all identity types are sets, which we do by giving an explicit description of the universal \covering. 

 We use the notation of \cref{def:wedge} freely, and for ease of notation, let $a_{2k+i}\defequi a_i$ and $i_{2k+i}^g\defequi i_i^g$ for $i=1,2$, $k:\NN$.  % Let $i_1:A_1\to A_1\vee A_2$ and $i_2:A_2\to A_1\vee A_2$ be the two inclusions, let $g:i_1\pt_{A_1}=i_2\pt_{A_2}$ be the imposed identity in the (non-symmetric formulation of the) wedge sum based in $\pt_{A_1\vee A_2}\defequi i_1\pt_{A_1}$.  For ease of notation, let $a_{2k+i}$ denote $\pt_{A_i}$ for $i=1,2$, $k:\NN$.
Define families of sets
$$C_i:A_i\to\Set,\qquad i=1,2$$
by 
$$C_i(x)\defequi(a_i=_{A_i}x)\times\sum_{n:\NN}\prod_{1\leq k\leq n}\sum_{p_k:a_{i+k}=% _{A_{i+k}}
  a_{i+k}}(p_k\neq\refl {a_{i+k}})$$
when $x:A_i$.  Note that $p_k\neq\refl{a_{i+k}}$  is a proposition; we leave it out when naming elements. Hence, an element in $C_1(a)$ is a tuple
$(p_0,n,p_1,\dots,p_n)$ where $p_0:a_1=_{A_1}a$, $p_1:a_2=_{A_2}a_2$, $p_2:a_1=_{A_1}a_1$, and so on -- alternating between symmetries of $a_1$ and $a_2$, and where $p_0$ is the only identity allowed to be $\refl{}$. Define $C_{12}:C_1(a_1)\to C_2(a_2)$ by
$$C_{12}(p_0,n,p_1\dots,p_n)=
\begin{cases}
  (\refl{a_2}0,)&\text{ if }p_0=\refl{a_1}, n=0,\\
  (p_1,n-1,p_2\dots,p_n)& \text{ if }p_0=\refl{a_1},n\neq0,\\
  (\refl{a_2},n+1,p_0,\dots,p_n)& \text{ if }p_0\neq\refl{a_1}.
\end{cases}
$$
It is perhaps instructive to see a table of the values $C_{12}(p_0,n,p_1,\dots,p_n)$ for $n<3$:
\begin{center}
  \begin{tabular}{r|c cc}
    &$(p_0,0)$&$(p_0,1,p_1)$&$(p_0,2,p_1,p_2)$\\
    \hline
    $p_0=\refl{a_1}$&$(\refl{a_2},0)$&$(p_1,0)$&$(p_1,1,p_2)$\\
    $p_0\neq\refl{a_1}$&$(\refl{a_2},1,p_0)$&$(\refl{a_2},2,p_0,p_1)$&$(\refl{a_2},3,p_0,p_1,p_2)$
  \end{tabular}
\end{center}
Since $C_{12}$ is an equivalence, the triple $(C_1,C_2,C_{12})$ defines a family
$$C:A_1\vee A_2\to\Set.$$
In particular, $C(a_{12})\defequi C_1(a_1)$.
For $x:A_1$ we let $i^C_1:C_1(x)\to C(i_1(x))$ be the induced equivalence, and likewise for $i^C_2$.
We will show that $C$ is equivalent to $P\defequi \pathsp{a_{12}}$, where $P(x)\defequi(a_{12}=x)$, and so that the identity types in the wedge are equal to the sets provided by $C$.

One direction is by transport in $C$; more precisely, 
$$\alpha:\prod_{x:A_1\vee A_2}(P(x)\to C(x))$$ is given by transport with $\alpha(a_{12})(\refl{a_{12}})\defequi(\refl{a_{1}},0):C(a_{12})$.  %This is well defined since $\alpha_2(i_2\pt_{A_2})(gpg^{-1})=C_{12}\alpha_1(i_1\pt_{A_1})(p)$ ((is this how you'd say this?)).
The other way, 
$$\beta:\prod_{x:A_1\vee A_2}(C(x)\to P(x))$$ is given by composing identities, using the glue $g$ to make their ends meet: 
$$\beta(i_1a)(p_0,n,p_1,\dots,p_n)\defequi i_1(p_0)i_2^g(p_1)i_3^g(p_2) \dots i_{n+1}^g(p_n)$$ 
(here the definition $\dots i_3^g\defequi i_1^g\defequi i_1$ proves handy since we don't need to distinguish the odd and even cases)  % (ending in $\dots gi_1(p_n)$ if $n$ is even and $\dots g^{-1}i_2(p_n)g$ if $n$ is odd) 
and likewise 
$$\beta(i_2a)(p_0,n,p_1,\dots,p_n)\defequi i_2(p_0)g\,i_1^g(p_1)i_2^g(p_2) \dots i_{n}^g(p_n)$$ and compatibility with the glue $C_{12}$ is clear since the composite $\refl{x}p$ is equal to $p$.
%$\beta_1(p_0,n,p_1,\dots,p_n,!)\defequi i_1(p_0)g^{-1}i_2(p_1)gi_1(p_2) g^{-1}\dots $ (ending in $\dots gi_1(p_n)$ if $n$ is even and $\dots g^{-1}i_2(p_n)g$ if $n$ is odd) and likewise for $\beta_2$ and the glue ((write)).

For notational convenience, we hide the $x$ in $\alpha(x)(p)$ and $\beta(x)(p)$ from now on.

That $\beta\alpha(p)=p$ follows by path induction: it is enough to prove it for $x=a_{12}$ and
$p\defequi\refl{a_{12}}$:
$$\beta\alpha(\refl{a_{12}})=\beta(\refl{a_1},0)=i_1^g\refl{a_1}=\refl{a_{12}}.$$  

That $\alpha\beta(p_0,n,p_1\dots,p_n)=(p_0,n,p_1,\dots,p_n)$ follows by induction on $n$ and $p_0$.  For $n=0$ it is enough to consider  $x=a_{12}$ and $p_0=\refl{a_1}$, and then 
$\alpha\beta(\refl{a_1},0)\defequi\alpha(\refl{a_{12}})\defequi(\refl{a_1},0)$.  In general, (for $n>0$) 
\begin{align*}
  \alpha\beta(p_0,n,p_1\dots,p_n)
=&\trp{C,i_1(p_0)i_2^g(p_1)i_1^g(p_2) \dots i_{n+1}^g(p_n)}(\refl{a_1,0})\\
=&\trp{C,i_1(p_0)}\dots\trp{C,i_{n+1}^g(p_n)}(\refl{a_1,0}).
\end{align*}
  The induction step is as follows: let $0< k\leq n$, then 
\begin{align*}
  &\trp{C,i_k^gp_{k-1}}i^C_{k-1}(p_k,n-k-1,p_{k+1},\dots,p_n)\\
  =&\trp{C,i_k^gp_{k-1}}i^C_k(\refl{a_{k-1}},n-k,p_k,\dots,p_n)\\
  =&i^C_k\trp{C_k,p_{k-1}}(\refl{a_{k-1}},n-k,p_k,\dots,p_n)\\
  =&(p_{k-1},n-k,p_k,\dots,p_n).
\end{align*}
((please see whether this makes sense to anybody but yvt))
\end{proof}

\begin{definition}
  \label{def:sumofgroup}
  If $G_1=\aut_{A_1}(a_1)$ and $G_1=\aut_{A_1}(a_1)$ are groups, then their \emph{sum}\index{sum of groups} is defined as
  $$G_1\vee G_2\defequi \aut_{A_1\vee A_2}(a_{12}).$$ The homomorphisms $i_1:G_1\to G_1\vee G_2$ and $i_2:G_2\to G_1\vee G_2$ induced from the structure maps  $i_1:A_1\to A_1\vee A_2$ and  $i_2:A_2\to A_1\vee A_2$ are also referred to as structure maps.
\end{definition}
\begin{lemma}
  \label{lem:sumofgroupsISsum} If $G_1$, $G_2$ and $G$ are groups, then the function
  $$\Hom(G_1\vee G_2,G)\to\Hom(G_1,G)\times\Hom(G_2,G)$$ 
given by restriction along the structure maps is an equivalence.
\end{lemma}
\begin{proof}
  ((write))
\end{proof}
Specializing, we return to our initial motivation and see that mapping out of a wedge of two circles \emph{exactly} captures the information of two independent symmetries:
\begin{corollary}
  \label{cor:ZplusZuniv}
  If $G$ is a group, then the functions
  $$\Hom(\ZZ\vee\ZZ,G)\to \Hom(\ZZ,G)\times\Hom(\ZZ,G)\to G\times G$$
  are equivalences.((fix language))
\end{corollary}
\begin{xca}
This leads to the following characterization of abelian groups formulated purely in terms of pointed connected groupoids (no reference to the identity types).
  \label{xca:whatAREabeliangroups}
  A group $G$ is abelian if and only if the canonical map 
$$+:G\vee G\to G$$ 
(given via \cref{lem:sumofgroupsISsum} by $G\oldequiv G$) extends over the inclusion 
$$i:G\vee G\to G\times G$$ 
(given by the inclusions $\mathrm{in}_1,\mathrm{in}_2:G\to G\times G$).\footnote{I haven't written out a formalization myself}
\end{xca}



%\section{structure of identity types}
%\section{automorphism 1-group = fundamental group (hint at higher groups)}
%\section{homomorphisms induced by functions (early)}
\section{``more examples: symmetric groups, integers, cyclic groups and modular arithmetic''}
\section{``group actions, orbits and fixed points''}

\section{Subgroups}
\label{sec:subgroups}
In our discussion of the group $\ZZ=\aut_{S^1}(\base)$ of integers in we discovered that the ``subsymmetries'' formed a very organized structure.  For each natural number $n$ we obtained a set of subsymmetry the entire identity type $\base=\base$, namely the set of all the iterates $(\Sloop^{n})^m$ where $m$ varies over the integers.  When $n$ was positive this was realized as the $n$-fold \covering of $S^1$ , when $n=0$ this was given by the universal \covering.  

For other groups the ``subsymmetries'' form more involved structures.  One thing is that our concept of a subtype of $B$ is merely the first projection $\sum_{b:B}P(b)\to B$, where the $P$ is a family of propositions.  Another thing is that for group the ``sub'' refers to the associated abstract groups, so that ``$BH$ is a subgroup of $BG$'' should \emph{not} mean that ``$BH$ is a subtype of $BG$'', but that we have a group homomorphism $f:\Hom(H,G)$ so that the induced function $(\pt_H=\pt_H)\to(\pt_G=\pt_G)$ is an ``inclusion of a subset''. 

Now, as we've seen, that   $f:(\pt_H=\pt_H)\to(\pt_G=\pt_G)$ is an injection (preimages are propositions) is equivalent to the preimages of $Bf:BH\to BG$ being sets.  Hence we get the following neat formulation.
    \begin{definition}
      \label{def:subgroup}
      Let $G$ be a group.  
      The \emph{type of subgroups of $G$}\index{type!subgroup} is the type
      $$\typesubgroup_G\defequi\sum_{H:\typegroup}\sum_{f:\Hom(H,G)}\isset(Bf^{-1}(\pt_G)).$$
       A subgroup $(H,f,!)$ is
      \begin{enumerate}
      \item \emph{trivial}\index{trivial subgroup} if $BH$ is contractible
      \item \emph{proper}\index{proper subgroup} if $Bf$ is not an equivalence.
      \end{enumerate}
    \end{definition}
    \begin{remark}
      \label{rem:notationsubgroup}
      A note on notation is in order.  
If $(H,i,!)$ is a subgroup of a group $G$ tradition often permits us to relax the burden of notation; we may write ``a subgroup $i:H\subseteq G$'', or, if we don't need the name of $i:\Hom(H,G)$ in what follows, simply ``a subgroup $H\subseteq G$'' or ``a subgroup $H$ of $G$''.
    \end{remark}

\begin{lemma}
  \label{lem:setofsubgroups}
  If $G$ is a group, then the type $\typesubgroup_G$ of subgroups is a set.
\end{lemma}
\begin{proof}
An identity between two subgroups $i_H:H\subseteq G$ and $i_{H'}:H'\subseteq G$ is an identity $p:H'=_{\typegroup}H$ such that $i_{H'}=i_H\,p$ (a proposition since $\Hom(H',G)$ is a set).
  By univalence and \cref{lem:eqofconntypes}, the identity type $H=H'$ is equivalent to the set 
$$\sum_{f:\Hom(H,H')}\isEq(f^=:(\pt_H=\pt_H)\to(\pt_{H'}=\pt_{H'})).$$  
%If $(H,i_H,!)$ is a subgroup of $G$, then
Consequently, the identity type
$(H,i_H,!)=_{\typesubgroup_G}(H,i_H,!)$ is equivalent to the type of homomorphisms $f:\Hom(H,H)$ which are such that $!:i_H^==i_H^=f^=$ and such that $f^=$ is an equivalence (as we see in a moment this last requirement is redundant).  
Now, since $(H,i_H,!)$ is a subgroup, $i_H^=$ is an injection of sets, which forces $!:f^==\refl{\pt_H=\pt_H}$, which ultimately forces $f$ to be (identical to) the identity homomorphism. 
\end{proof}
Not only is the type of subgroups  of $G$ a set, it is in a natural way (equivalent to the value at $\pt_G$ of) a $G$-set which we denote by the same name
$$%\typesubgroup_G:BG\to\Set,\qquad 
\typesubgroup_G(y)\defequi \sum_{H:\typegroup}\sum_{f:\Hom(H,G)(y)}\isset(Bf^{-1}(\pt_G)),$$
where  as in \cref{ex:HomHGasGset}  
$$\Hom(H,G)(y)\defequi\sum_{F:BH_\div\to BG_\div}(y=F(\pt_H))$$
is the $G$-set of homomorphisms from $H$ to $G$.
% In this interpretation, $(H,F,p,!):\typesubgroup_G(\pt_G)$ represents a subgroup of $G$ (so that $p:\pt_G=F(\pt_H)$).  An identity $g:\pt_G=\pt_G$ acts on $\typesubgroup_G(\pt_G)$ by sending $(H,F,p,!)$ to $(H,F,p\,g^{-1},!)$.

\begin{definition}
  \label{def:conjactonsubgroups}
  If $G$ is a group, then the action of $G$ on the set of subgroups is called \emph{conjugation}. 


  \label{def:conjugate}
  If $(H,F,p,!):\typesubgroup_G(\pt_G)$ is a subgroup of $G$ and $g:\pt_G=\pt_G$, then the subgroups  $(H,F,p,!),(H,F,p\,g^{-1},!):\typesubgroup_G(\pt_G)$ are said to be \emph{conjugate}\index{conjugate}. 
\end{definition}
\begin{remark}
  \label{rem:whyconjugate}
  The term ``conjugation'' may seem confusing as the %(abstract) 
action of $g:\pt_G=\pt_G$ on a subgroup $(H,F,p,!):\typesubgroup_G(\pt_G)$ (where $p:x=F(\pt_H)$) is simply $(H,F,p\,g^{-1},!)$, which does not seem much like conjugation.  
However, as we saw in \cref{ex:abstrandconj}, under the equivalence $\abstr:\Hom(H,G)\we\Hom^\abstr(\abstr(H),\abstr(G))$, the corresponding action on $\Hom^\abstr(\abstr(H),\abstr(G))$ is exactly (postcomposition with) conjugation $c^g:\abstr(G)=\abstr(G)$.  
\footnote{The same phenomenon appeared in \cref{xca:HomZGvsAdG} where we gave an equivalence between the $G$-sets $\Hom(\ZZ,G)$ and $\Ad_G$ (where the action is very visibly by conjugation).}
% \end{remark}

% \begin{remark}
  \label{rem:conjactiononsubgroups}
 %  It is worthwhile to study this action a bit further.  Let $(H,f,!)$ be a subgroup of $G$ and let $g:\pt_G=\pt_G$.  We can trade $f$ for $\abstr(f):\Hom^\abstr(\abstr(G),\abstr(H))$ whose underlying set map is the injection $Bf^=:(\pt_H=\pt_H)\to(\pt_G=\pt_G)$. %We allow ourselves to write ``$f$'' instead of $Bf^=$. 
% As we saw in \cref{ex:abstrandconj}, the action of $g$ postcomposes with conjugation $c^g:(\pt_G=\pt_G)=(\pt_G=\pt_G)$.
% Hence, if we represent a subgroup as 
% $$(H,\phi,!):\sum_{H:\typegroup}\sum_{\phi:\Hom^\abstr(\abstr(H),\abstr(G))}\isprop(\phi^{-1}(e_G)),$$ then
% $g\cdot(H,\phi,!)=(H,c^g\,\phi,!)$.

% Now, if $g$ comes from $H$, say, $\i_H(h)$
\end{remark}
Summing up the remark:
\begin{lemma}
  \label{lem:conjugationabstractly}
  Under the equivalence of \cref{lem:actionsconcreteandabstract} between $G$-sets and $\abstr(G)$-sets, the $G$-set $\typesubgroup_G$ corresponds to the $\abstr(G)$-set
$$\sum_{H:\typegroup}\sum_{\phi:\Hom^\abstr(\abstr(H),\abstr(G))}\isprop(\phi^{-1}(e_G))$$ of abstract subgroups of $\abstr(G)$, with action $g\cdot(H,\phi,!)\defequi(H,c^g\,\phi,!)$ for $g:\abstr(G)$, where  $c^g:\abstr(G)=\abstr(G)$ is conjugation as defined in \cref{ex:conjhomo}.
\end{lemma}


\begin{remark}
  If you're familiar with the set-theoretic flavor of things, you know that it is important to distinguish between subgroups and injective group homomorphisms.  
Our use of ``subgroup'' can be defended as follows.  
It corresponds in set-theoretic language to saying that a subgroup is an injective homomorphism \emph{modulo} the relation forcing that precomposing with an isomorphism yields identical subgroups.  
Set-theory offers the luxury of having a representative in every equivalence class: namely the image of the injection, type theory does not.
\end{remark}

\sususe{The geometry of subgroups: some small examples}
\label{smallsubgpex}

As a teaser, and in order to get a geometric feel for the subgroups and their intricate interplay, it can be useful to have some fairly manageable examples to stare at.  
Some of the main tools for analyzing the geometry of subgroups are collected in \cref{sec:fingp} on finite groups, and we hope the reader will be intrigued by our mysterious claims and go on to study \cref{sec:fingp}.
That said, the examples we'll present are possible to muddle through by hand without any fancy machinery, but brute force is generally not an option and even for the present examples it is not something you want to show publicly.

When presenting the subgroups of a group $G$, three types are especially revealing: the set of subgroups $\typesubgroup_G(\pt_G)$, the \emph{groupoid of subgroups} $\typesubgroup(G)\defequi\sum_{y:BG}\typesubgroup_G(y)$ and what we for now call the ``set of normal subgroups'' $\prod_{y:BG}\typesubgroup_G(y)$.   Our local use of ``normal subgroup'' is equivalent to the official definition to come.  

The first projection $\typesubgroup(G)\to BG$ is referred to as the \emph{\covering of subgroups}.

\footnote{Write out and fix the concrete examples (cyclic groups and $\Sigma_3$) commented out}
% \begin{remark}
% In  \cref{cha:circle} we studied the subgroups of the group of integers $G=\ZZ$ through \coverings over the circle $S^1$ (which we showed was equivalent to $B\ZZ$).
% We discovered a subgroup $n\ZZ$ for each natural number $n:\NN$ and in the groupoid $\typesubgroup({\ZZ})$ these sit as elements in separate components.  Each of these components are contractible (because addition is commutative: $\ZZ$ is an abelian group).

% In general, a component $K$ of the groupoid $\sum_{y:BG}\typesubgroup_G(y)$ of subgroups of a group $G$ may be much more interesting. For one thing the, $K$ can contain many subgroups in the sense that the preimage of the first projection $K\to BG$ is a set that may have many different elements; each representing a subgroup.  However, this set of subgroup will be a \emph{conjugacy class} of subgroups: the different subgroups are related by the conjugation action of $G$.  

% If $G$ is abelian this action is trivial, and $\sum_{y:BG}\typesubgroup_G(y)$ consists of contractible components indexed over the subgroups of $G$.  Otherwise different subgroups may live in the same component of the groupoid of subgroups -- we'll see examples in a moment.

% In addition, the components will not in general be contractible, revealing the symmetries of the subgroups under the conjugation action.
% \end{remark}


% \begin{example}
%   The trivial group only has itself as a subgroup; the groupoid of subgroups and the set of normal subgroups are singletons.
% \end{example}
% \begin{example}
%   The cyclic group $C_p$ of prime order $p$ has only two subgroups, the trivial and the full subgroup itself and both are normal.  In fact, all subgroups of abelian groups are normal.  

% In general, the cyclic group $C_n$ of order $n$ has exactly one subgroup for each divisor $i$ of $n$.
% \end{example}


% \begin{example}
%   The group $C_2\times C_2$ has has no less than five subgroups: the trivial one, three subgroups that as groups (as opposed as \emph{sub}groups) are equivalent to $C_2$ and the full group $C_4$ itself.
% \end{example}
% \begin{remark}
%   The permutation group $\Sigma_3$ has four nontrivial proper subgroups.  Three conjugate subgroups isomorphic as groups to $C_2$ and one normal one which is as a group is isomorphic to $C_3$.  The component containing the copies of $C_2$ is equivalent to a circle.
% \end{remark}
\sususe{Kernels and cokernels}
\label{subsec:ker}
%\newcommand{\ker}{\mathrm{ker}}
If $\phi:\Hom^\abstr(\mathcal G,\mathcal G')$ is an abstract group homomorphism, the preimage $\phi^{-1}(e_G)$ is a classically called the kernel of $\phi$ and the cokernel is the quotient set of $\mathcal G'$ by the relation that if $g:\mathcal G$ and $g':\mathcal G'$, then $g'\sim g'\cdot\phi(g)$.  
In our setup with a group homomorphism 
$$f:\Hom(G,G')\defequi(BG\to_*BG'),$$ the kernel and cokernel are just two aspects of the preimage 
$$(Bf)^{-1}(\pt_{G'})\defequi\sum_{z:BG}(\pt_{G'}=Bf(z)):$$
 the cokernel is the set of components and the kernel is a preferred component.  This point of view makes it clear that the kernel is a subgroup whereas there is no particular reason for the cokernel to be more than a ($G'$-) set.
\newcommand{\coker}{\mathrm{coker}\,}
\newcommand{\image}{\mathrm{im}\,}
\begin{definition}
  \label{def:cokernel}
  Let $f:\Hom(G,G')\defequi(BG\to_*BG')$  be a homomorphism. %  The preimage $$f^{-1}(\pt_{G'})\defequi\sum_{z:BG}(\pt_{G'}=f(z))$$
% is a groupoid containing important information about $f$.
The \emph{cokernel}\index{cokernel} of $f$ is the $G'$-set
$$\coker f:BG'\to\Set,\qquad \coker f(z)\defequi  ||(Bf)^{-1}(z)||_0.\footnote{set trunctation} $$ 
The associated $\abstr(G')$-set $\coker f(\pt_{G'})$ is also referred to as the cokernel of $f$.  If $f:\Hom(G,G')$ is clear from the context and displays $G$ as a subgroup of $G'$, we often write $G'/G$ for the cokernel of $f$.  

\end{definition}
\begin{definition}
  \label{def:kernel}
Let $f:\Hom(G,G')$  be a homomorphism.
Consider the element 
$$\pt_{\ker f}\defequi(\pt_G,p_f):(Bf)^{-1}(\pt_{G'})$$ (where $p_f:\pt_{G'}=Bf(\pt_G)$ is the part of $f$ claiming it is a pointed map). 
Define the \emph{kernel}\index{kernel}  of $f$ to be the group defined by the pointed component  of $\pt_{\ker f}$ in $(Bf)^{-1}(\pt_{G'})$:
$$\ker f\defequi ((Bf)^{-1}(\pt_{G'})_{(\pt_{\ker f})},\pt_{\ker f}).
$$ 
Written out,
$B\ker f_\div\defequi \sum_{z:BG}\sum_{p:\pt_{G'}=f(z)}||\pt_{\ker f}=(z,p)||.$  

The first projection $B\ker f\to_* BG$ displays the kernel $\ker f$ as a subgroup of $G$ (\emph{sub}group since the preimages are equivalent to the sets $\sum_{p:\pt_{G'}=Bf(z)}||\pt_{\ker f}=(z,p)||$).  

A subgroup is said to be \emph{normal}\index{normal} if it is the kernel of a surjective homomorphism.\footnote{clarify the relation between the surjective homomorphism and the subgroup}
\end{definition}

\begin{definition}
\label{def:image}\label{def:surjective}
The \emph{image}\index{image} of $f$ is the subgroup of $G'$ given by $B\image f\defequi \sum_{z:BG'}\coker f(z)$ pointed at $\pt_{\image f}\defequi (\pt_{G'},|\pt_G,p_f|)$ together with the first projection $B\image f\to BG'$ (plus the fact that $\coker f(z)$ is a set).  

The \emph{induced homomorphism} $\tilde f:\Hom(G,\image f)$ is given by sending $x:BG$ to 
$$B\tilde f(x)\defequi (Bf(x),|x,\refl{Bf(x)}|).$$ 

We say that the homomorphism $f$ is \emph{surjective}\index{surjective! group homomorphism} if $(Bf)^{-1}(\pt_{G'})$ is connected.
%, or equivalently if $\image f\to G'$ is an equivalence.
\end{definition}
% \begin{remark}
%   \label{rem:cokerasGset}
%   If $f:\Hom(G,G')$ we notice that the abstract group $\abstr(G')$ acts on $\coker(f)\defequi||f^{-1}(\pt_{G'})||_0$, making the cokernel an $\abstr(G')$-set.  If we prefer to talk about a $G'$-set, we consider the cokernel as the set-family $$BG'\to\Set,\qquad z\mapsto   ||f^{-1}(z)||_0.$$  
% We will see this used most frequently when considerint inclusions of subgroups: if $H$ is a subgroup of $G$, then $G/H$ is a $G$-set.
% \end{remark}
In view of \cref{ex:charsurinj} below, the families  
$$\mathrm{issurj},\mathrm{isinj}:\Hom(G,G')\to\Prop
$$
of propositions that a given homomorphism is surjective or injective have several useful interpretations.
\begin{xca}
  \label{ex:charsurinj}
  Let $f:\Hom(G,G')$ Prove that
  \begin{enumerate}
  \item the following are equivalent
    \begin{enumerate}
    \item $f$ is a surjective homomorphism (\cref{def:surjective}),
    \item the cokernel of $f$ is contractible,
    \item the first projection $B\image f\to BG$ is an equivalence
    \item the induced map of sets 
$f^=:(\pt_G=\pt_G)\to(\pt_{G'}=\pt_{G'})$ is a surjection
    \end{enumerate}
  \item the following are equivalent
    \begin{enumerate}
    \item $f$ is an injective homomorphism (\ie the induced map of sets 
$f^=:(\pt_G=\pt_G)\to(\pt_{G'}\to\pt_{G'})$
is an injection)
\item the kernel of $f$ is trivial
\item $Bf:BG\to BG'$ is a \covering.
\item the induced map $B\tilde f:BG\to B\image f$ is an equivalence.
    \end{enumerate}
  \end{enumerate}
\end{xca}



Note that if $f:\Hom(G,G')$, then the composite of the induced homomorphism $\tilde f:\Hom(G,\image f)$ with the subgroup inclusion (first projection) of $\image f$ in $G'$ is $f$ by definition.  We will refer to this as the \emph{factorization of $f$ through its image}.

\begin{lemma}
  \label{lem:kerandcoker}
  \label{lem:countinggps}
  Let $f:\Hom(G,G')$ be a group homomorphism.  The induced homomorphism $\tilde f:\Hom(G,\image f)$ is a surjective homomorphims and $f$ factors as $\tilde f$ followed by the inclusion of the image of $f$ in $G'$.  The induced map $(B\tilde f)^{-1}(\pt_{\image f})\to (Bf)^{-1}(\pt_{G'})$ induces an equivalence $B\ker\tilde f\simeq B\ker f$.
\end{lemma}
\begin{proof}
Only the last part needs further comment, but follows since since for $x:BG$ the first projection from $\pt_{\image f}=(Bf,|x,\refl{Bfx}|)$ to $\pt_{G'}=f(x)$ is an equivalence (the fibers are true propositions).
  \footnote{\color{blue}  
 Also show counting results for the finite group part somewhere.} 
\end{proof}

Finally, the image factorization would have been useless were it not for the fact that it is unique:
\begin{lemma}
  \label{lem:uniquenessofimagefactorizationforgroups}
  Let $G,H,G'$ be groups, let $h:\Hom(G,H)$ and $j:\Hom(H,G')$ be homomorphisms and let $!:f=j\,h$.  If $h$ is surjective there is a unique homomorphism $t:\Hom(H,\image f)$ so that $\tilde f=t\, h$ and $j$ is $t$ composed with the first projection from $\image f$ to $ G'$.
\end{lemma}
\begin{proof}
  We've used that we're operating with groupoids to simplify the statement, but a similar statement follows generally by essentially the proof below if you keep track of the element in $f=j\,h$.  To simplify we drop the ``$B$''s from the notation, writing ``$f$'' instead of ``$Bf$''.  

That $h$ is a surjective homomorphism amounts to saying that for $y:BH$, then the set truncation $||h^{-1}(y)||_0$ of the preimage is contractible, and so the first projection $\mathrm{pr_1}:\sum_{y:BH}|h^{-1}(y)||_0\to BH$ is an equivalence.

For $y:BH$, consider the map 
$$T_y:h^{-1}(y)\to f^{-1}(j y),\qquad T_y(x,p)\defequi (x,!_xj(p))$$ where $x:BG$, $p:y=h(x)$ and $!_xj(p):j(y)=f(x)$ is the composite of $j(p):j(y)=j\,h(x)$ and $!:j\,h=f$ (as applied to $x$).  Performing set-truncation on $T_y$ and precomposing with the inverse of the first projection, we get a map
$$t:BH%\sum_{y:BH}||h^{-1}(y)||_0
\to\sum_{z:BG'}||f^{-1}(z)||_0\oldequiv B\image f,\qquad Bt(y)\defequi(jy,|T_y|(q_y))$$
where $q_y:||h^{-1}(y)||_0$ is the second projection of the inverse of the first projection.  The agreement of $t$ with $\tilde f$ and $j$ follows by construction.
\end{proof}

\begin{example}
  An example from linear algebra: let $A$ be any $n\times n$-matrix with nonzero determinant and with integer entries, considered as a homomorphism $A:\Hom(\ZZ^n,\ZZ^n)$.  Then the cokernel of $A$ is a finite set with cardinality the absolute value of the determinant of $A$.  You might want to picture this as a $|\det(A)|$-fold \covering of the $n$-fold torus $(S^1)^{\times n}$ by itself.
\end{example}


\sususe{Subgroups through $G$-sets}


Occasionally it is useful to define ``subgroups'' slightly differently.
As we've defined it a subgroup of a group $G$ of the form $(H,f,!)$ where $H$ is a group (pointed connected groupoid  $BH$), $f:BH\to_* BG$ is a pointed map whose fibers are sets (a pointed \covering).  There is really no need to specify that $H$ is a group: if $F:T\to BG$ is a \covering, then $T$ is automatically a groupoid.  

On the other hand,  the type of \coverings over $BG$ is equivalent to the type of $G$-sets: if $X:BG\to\Set$ is a $G$-set, then the covering is given by the first projection $\tilde X\to BG$ where $\tilde X\defequi\sum_{y:BG}X(y)$ and the inverse is obtained by considering the fibers of a \covering.  Furthermore, we saw in \cref{lem:conistrans} that $\tilde X$ being connected is equivalent to the condition $\mathrm{isTrans}(X)$ of \cref{def:connectedGset} claiming that the $G$-set $X$ is transitive. 

Hence, the type (set, really) $\typesubgroup_G$ of subgroups of $G$ is equivalent to the type of pointed connected \coverings over $BG$, which again is equivalent to the type $\typesubgroup_G'$ of transitive $G$-sets $X:BG\to\Set$ together with a point in $X(\pt_G)$.  

The family of sets $\typesubgroup_G(y)$ where we let the element $y:BG$ vary is by the same reasoning equivalent to the family $\typesubgroup_G'(y)$ which we for reference spell out in symbols.

\newcommand{\typenormal}{\mathbf{Nor}}
\begin{definition}
  Let $G$ be a group and $y:BG$, then the $G$-set of \emph{subgroups' of $G$} is
  $$\typesubgroup_G':BG\to\Set,\qquad\typesubgroup_G'(y)\defequi\sum_{X:BG\to\Set}\sum_{\pt_y:X(y)}\mathrm{isTrans}(X)$$
and the type of \emph{normal subgroups'} is the set of fixed points
$$\typenormal_G'\defequi\prod_{y:BG}\typesubgroup_G'(y).$$
%A \emph{normal subgroup'} of $G$ is an element in $\typenormal_G'$.
\end{definition}
Likewise, in symbols, the above described equivalence between the families $\typesubgroup_G$ and $\typesubgroup_G'$ is provided by the map 
$$E(y):\typesubgroup_G(y)\to\typesubgroup_G'(y),\qquad E(H,F,p_F,!)=(F^{-1}, (\pt_H,p_F),!)
$$
(where $H$ is a group, $F:BH_\div\to BG_\div$ is a map and $p_F:y=F(\pt_H)$ an identity in $BG$; and $F^{-1}:BG\to\Set$ is $G$-set given by the preimages of $F$ and $(\pt_H,p_F):F^{-1}(y)\defequi \sum_{x:BH}y=F(x)$ is the base point).  If $y$ is $\pt_G$ we follow our earlier convention of dropping it from the notation.


Since the families are equivalent we may use $\typesubgroup_G$ or $\typesubgroup_G'$ interchangeably.  
There is, however, a little explanation needed in order to see that the type $\typenormal_G$ of normal subgroups is equivalent to $\typenormal'_G$.
We do this by using the intermediate set of surjections from $G$:
\newcommand{\epi}{\mathrm{epi}}
\begin{definition}
  \label{def:typeepi}
  If $G$ is a group, then the \emph{set of surjections from $G$} is the set
$$\epi_G\defequi\sum_{G':\typegroup}\sum_{f:\Hom(G,G')}\mathrm{issurj}(f).$$
\end{definition}
Note that if $f:\Hom(G,G')$ is a surjective homomorphism and $e:G'=G''$ is an identity of groups, then $(G',f,!)$ and $(G'',f',!)$ are identitified via $e$, where $f':\Hom(G,G'')$ is the homomorphism given by the composite of $f$ and the homomorphism corresponding to $e$.

\begin{definition}
  \label{def:ker2}
  If $f:\Hom(G,G')$ is a homomorphism and $x,y:BG$, set $P^f_y(x)\defequi (f(y)=f(x))$.
  Define $$\ker':\epi_G\to\typenormal_G'$$
  by $\ker'(G',f,!)(y)\defequi(P^f_y,\refl{f(y)},!)$.
\end{definition}

\begin{lemma}
  \label{lem:diagfornormal}
  The diagram
  $$\xymatrix{
  &\typenormal_G\ar[r]^{\subseteq}&\typesubgroup_G\ar[dd]_{\simeq}^{E}\\
  \epi_G\ar[ur]^{\ker}\ar[dr]_{ker'}&&\\
  &\typenormal_G'\ar[r]^{\subseteq}&\typesubgroup_G'}
$$
commutes, where the top composite is the image factorization of the kernel and the bottom inclusion is the inclusion of fixed points.
\end{lemma}
\begin{proof}
  Following $(G',f,!):\epi_G$ around the top to $\typesubgroup_G'$ yields the transitive $G$-set sending $y:BG$ to the set $\pt_{G'}=f(y)$ together with the point $p_f:\pt_{G'}=f(\pt_G)$ while around the bottom we get the transitive $G$-set sending $y:BG$ to the set $f(\pt_G)=f(y)$ together with the point $\refl{f(\pt_G)}:f(\pt_G)=f(\pt_G)$.  Hence, precomposition by $p_f$ gives the identity proving that the diagram commutes. 
\end{proof}
We will prove that both $\ker$ and $\ker'$ in the diagram of \cref{lem:diagfornormal} are equivalences, leading to the desired conclusion that the equivalence $E:\typesubgroup_G\we\typesubgroup_G'$ takes the subset $\typenormal_G$ identically to $\typenormal_G'$.  Actually, by the uniqueness of the image factorization shown in \cref{lem:uniquenessofimagefactorizationforgroups} it is enough to show that $\ker'$ is an equivalence; we'll spell out the details.

We start with a small, but crucial observation.
\begin{lemma}
  \label{lem:evaliseqwhennormal}
  Let $N:\typenormal'_G$ be a normal subgroup' with $N(y)\oldequiv (X_y,\pt_y,!)$ for $y:BG$.  Then for any $y,z:BG$
  \begin{enumerate}
  \item the evaluation map
$$\mathrm{ev}_{yz}:(X_y=X_z)\to X_z(y),\qquad \mathrm{ev}_{yz}(f)=f_y(\pt_y)$$
is an equivalence and
  \item  the map $X:(y=z)\to(X_y=X_z)$ (given by induction via $X_{\refl y}\defequi\refl{X_y}$) is surjective.
  \end{enumerate}
\end{lemma}
\begin{proof}
To establish the first fact we need to do induction independently on $y:BG$ and $z:BG$ in $X_y(z)$ at the same time as we observe that it suffices (since $BG$ is connected) to show that $\mathrm{ev}_{yy}$ is an equivalence.

% Induction on the index gives rise to the map $X:(y=z)\to(X_y=X_z)$ ($X_{\refl y}\defequi\refl{X_y}$) and t
The composite 
$$\mathrm{ev}_{yy}X:(y=y)\to X_yy$$ is determined by $\mathrm{ev}_{yy}X(\refl y)\oldequiv \pt_y$. 
By transitivity of $X_y$ this composite is surjective, hence $\mathrm{ev}_{yy}$ is surjective too.  

On the other hand, in  \cref{lem:evisinjwhentransitive} we used the transitivity of $X_y$ to deduce that $\mathrm{ev}_{yy}$ was injective.  Consequently $\mathrm{ev}_{yy}$ is an equivalence.  But since $\mathrm{ev}_{yy}$ is an equivalence and $\mathrm{ev}_{yy}X$ is surjective we conclude that $X$ is surjective
\end{proof}
\begin{definition}
\label{def:normalquotient}
  Let $N:\typenormal'_G$ be a normal subgroup' with $N(y)\oldequiv (X_y,\pt_y,!)$ for $y:BG$.  The \emph{quotient group}\index{quotient group} $G/N$ is the group defined as the component of the groupoid of $G$-sets containing and pointed in $X_{\pt_G}$.  

The \emph{quotient homomorphism}\index{quotient homomorphism} is the homomorphism $q_N:\Hom(G,G/N)$  defined by $Bq_N(z)=X_z$ (strictly pointed).  By \cref{lem:evaliseqwhennormal} $q_N$ is surjective and we have defined a map
$$q:\typenormal_G'\to\epi_G,\qquad q(N)=(G/N,q_N,!).$$
\end{definition}

\begin{remark}
It is instructive to see how the quotient homomorphism $Bq_N:BG\to BG/N$ is defined in the torsor interpretation of $BG$.  If $Y\colon BG\to\UU$ is a $G$-type we can define the quotient as
$$
Y/N:BG\to\UU,\qquad Y/N(y)\defequi\sum_{z:BG}Y(z)\times X_z(y).
$$
We note that in the case $\princ G(y)\defequi (\pt_G=y)$ we get
that 
$
\princ G /N(y)\defequi\sum_{z:BG}(\pt_G=z)\times X_z(y)
$
is equivalent to $X_{\pt_G}$.  Consequently, if $Y$ is a $G$-torsor, then $Y/N$ is in the component of $X_{\pt_G}$ and we have
$$-/N:\typetorsor_G\defequi (G\text{-set})_{(\princ G)}\to (G\text{-set})_{(X_{\pt_G})}.
$$ Our quotient homomorphism $q_N:\Hom(G,G/N)$ is the composite of the equivalence $\pathsp{}^G:BG\we\typetorsor_G$ of \cref{lem:BGbytorsor} and the quotient map $-/N$.
\end{remark}
\begin{lemma}
  \label{lem:qeq}
  The map $\ker':\epi_G\to\typenormal_G'$ is an equivalence with inverse $q:\typenormal_G'\to\epi_G$.
\end{lemma}
\begin{proof}
  Assume $N:\typenormal_G'$ with $N(y)\defequi(X_y,\pt_y,!)$ for $y:BG$.  Then $\ker' q(N):BG\to\Set$ takes $y:BG$ to $(\ker' q(N))(y)\oldequiv(Y_y,\refl{X_y},!)$, where $Y_y(z)\defequi (X_y=X_z)$.  Noting that the equivalence $\mathrm{ev}_{yz}:(X_y=X_z)\we X_z(y)$ of \cref{lem:evaliseqwhennormal} has $\mathrm{ev}_{yy}(\refl{X_y})\defequi \pt_y$ we see that univalence gives us the desired identity $\ker' q(N)=N$.\footnote{fix so that it adhers to dogmatic language and naturality in $N$ is clear}

Conversely, consider a surjective homomorphism $f:\Hom(G,G')$.  
For $x:BG$ and $z:BG'$ let $Q^f_z(x)\defequi (z=f(x))$.  
Then the quotient group $G/\ker'(f)$ is the component  of the groupoid of $G$-sets containing and pointed at $Q^f_{f(pt_G)}$ and the quotient homomorphism $q_{\ker' f}:BG\to_* BG/\ker' f$ is given by  $q_{\ker' f}(y)\defequi Q^f_{f(y)}$ (strictly pointed -- \ie by $\refl{Q^f_{f(\pt_G)}}$). 
Observe that by using the identification of the basepoint $Q^f_{f(\pt_G)}$ of $BG/\ker'f$ with $Q^f_{\pt_{G'}}$ given by $p_f:\pt_{G'}=f(\pt_G)$ we have defined a homomorphism $Q^f:\Hom(G',G/\ker'f$) such that 
 $$\xymatrix{&BG\ar[dl]_f\ar[dr]^{q_{\ker'f}}&\\
 BG'\ar[rr]_{Q^f}&&BG/\ker'f
}$$
commutes.
We are done if we can show that $Q^f$ is an equivalence.
The preimage of the base point $Q^f_{f(\pt_G)}$ is
$$\sum_{z:BG'}\prod_{y:BG}(z=f(y))=(f(\pt_G)=f(y))$$ 
which by 
\cref{lem:epifullyfaithful} is equivalent to
$$\sum_{z:BG'}\prod_{v:BG'}(z=v)=(f(\pt_G)=v)$$
which by \cref{lem:pathsptransportiseq} is equivalent to the contractible type $\sum_{z:BG'}z=f(\pt_G)$.
\end{proof}

\begin{corollary}
  \label{cor:normalisnormal}
  The kernel $\ker:\epi_G\to \typenormal_G$ is an equivalence of sets.
\end{corollary}
\begin{proof}
  Since $\ker':\epi_G\to\typenormal_G'$ and $E:\typesubgroup_G\to\typesubgroup_G'$ are equivalences and the diagram in \cref{lem:diagfornormal} commutes, the kernel from $\epi_G$ to $\typesubgroup_G$ is an injection.  Hence, given a normal subgroup, the type of epimorphisms of which it is the kernel is contractible.
\end{proof}

With this much effort in proving that our two perspectives on the concept of normal subgroups are the same, it can be worthwhile to make the composite equivalence
$$\ker\,q:\typenormal_G'\we\typenormal_G$$
explicit.   Let $N:\typenormal'_G$ be a normal subgroup' with $N(y)\oldequiv (X_y,\pt_y,!)$ for $y:BG$ with $X_y:BG\to\Set$, $\pt_y:X_y(y)$ and $!:\mathrm{isTrans}(X_y)$. 
Then 
$$(B\ker q_N)_\div\defequi\sum_{z:BG}(X_z=X_{\pt_G})$$
pointed in $(\pt_G,\refl{X_{\pt_G}})$ and with $i_{\ker f}:B\ker q_N\to_*BG$ given by the first projection.  This can be simplified somewhat:

\begin{definition}
  \label{def:associatednormal}
  Let $N:\typenormal'_G$ be a normal subgroup' with $N(y)\oldequiv (X_y,\pt_y,!)$ for $y:BG$ with $X_y:BG\to\Set$, $\pt_y:X_y(y)$ and $!:\mathrm{isTrans}(X_y)$.  Define a subgroup $(\mathrm{ass}(N),i_N,!)$ of $G$, called the \emph{associated normal subgroup}, as follows:
  \begin{enumerate}
  \item the connected groupoid $B\mathrm{ass}(N)_\div\defequi\sum_{z:BG}X_{\pt_G}(z)$,
  \item together with the point $\pt_N\defequi(\pt_G,\pt_{\pt_G})$,
  \item the first projection $Bi:B\mathrm{ass}(N)\to_*BG$
  \item together with the assertion that the preimages of $Bi$ (which are equivalent to $X_{\pt_G}(z)$ for varying $z:BG$) are sets. 
  \end{enumerate}
\end{definition}


We obviously need to show that the ``normal'' in the name is warranted.
\begin{lemma}
  \label{lem:normalsarekernels}
The map
$$\mathrm{ev}:B\ker q_N\to B\mathrm{ass}(N),\qquad \mathrm{ev}(z,f)\defequi(z,f(\pt_z))$$ is an equivalence and furthermore commutes with the first projections to $BG$.  The pointed map $\mathrm{ev}_*\defequi(\mathrm{ev},\refl{(\pt_G,\pt_{\pt_G})}):B\ker q_N\to_* B\mathrm{ass}(N)$
(well defined since $\mathrm{ev}(\pt_G,\refl{X_{\pt_G}})\defequi (\pt_G,\pt_{\pt_G})$) induces an identification of the subgroups $\ker\,q(N)$ and $\mathrm{ass}(N)$.  

Provided with this information, the associated normal subgroup $\mathrm{ass}(N)$ \emph{is} a normal subgroup and we get a commuting diagram
$$\xymatrix{&\typenormal_G\\
\epi_G\ar[ur]^{\ker}_{\simeq}\ar[dr]^{\ker'}_\simeq&\\
&\,\typenormal_G'.\ar[uu]^{\mathrm{ass}}_\simeq}$$
  
  % The associated normal subgroup of $N:\typenormal_G'$ is a normal subgroup; more precisely $N(\pt_G)$ is the kernel of the quotient homomorphism $q_N:\Hom(G,G/N)$.
  % Let $N:\typenormal'_G$ be a normal subgroup' with $N(y)\defequi (X_y,\pt_y,!)$ for $y:BG$ with $X_y:BG\to\Set$, $\pt_y:X_y(y)$ and $!:\mathrm{isTrans}(X_y)$.  Define a subgroup $(\mathrm{ass}(N),i_N,!)$ of $G$ as follows:
  % \begin{enumerate}
  % \item the connected groupoid $B\mathrm{ass}(N)_\div\defequi\sum_{z:BG}X_{\pt_G}(z)$,
  % \item together with the point $\pt_N\defequi(\pt_G,\pt_{\pt_G})$,
  % \item the first projection $Bi:B\mathrm{ass}(N)\to_*BG$
  % \item together with the assertion that the preimages of $Bi$ (which are equivalent to $X_{\pt_G}(z)$ for varying $z:BG$) are sets. 
  % \end{enumerate}
  % Then $(\mathrm{ass}(N),i_N,!)$ is a normal subgroup of $G$ in the sense that it is the kernel of a homomorphism.
\end{lemma}
\begin{proof}
  % Let $N(y)\oldequiv (X_y,\pt_y,!)$ for $y:BG$.  Let $G/N$ be the group defined as the component of the groupoid of $G$-sets containing and pointed in $X_{\pt_G}$.  Let $f:\Hom(G,G/N)$ be the homomorphism defined by $Bf(z)=X_z$.
 %  Consider the kernel of the quotient homomorphism $q_N:\Hom(G,G/N)$ of \cref{def:normalquotient},
% $$(B\ker q_N)_\div\defequi\sum_{z:BG}(X_z=X_{\pt_G})$$
% pointed in $(\pt_G,\refl{X_{\pt_G}})$ and with $i_{\ker f}:B\ker q_N\to_*BG$ given by the first projection.  Consider t

% We claim that $\mathrm{ev}$ is an equivalence.
Compared with the proof of $\ker'$ being an equivalence (\cref{lem:qeq}) there are no new ingredients.
Since $B\mathrm{ass}(N)$ is connected it is enough to show that the preimage of $\mathrm{ev}^{-1}(\pt_G,\pt_{\pt_G})$ is contractible.  
Since $\mathrm{ev}$ agrees with the projections to $BG$, the preimage is equivalent to $\sum_{f:X_{\pt_G}=X_{\pt_G}}f(\pt_{\pt_G})=\pt_{\pt_G}$.  We recognize this as the preimage $\mathrm{ev}_{{\pt_G}{\pt_G}}^{-1}(\pt_{\pt_G})$
of the evaluation map 
$\mathrm{ev}_{{\pt_G}{\pt_G}}:(X_{\pt_G}=X_{\pt_G})\to X_{\pt_G}(\pt_G)$ which is an equivalence by \cref{lem:evaliseqwhennormal}.  
% the preimage $\mathrm{ev}_{X_{\pt_G}}^{-1}(\pt_{\pt_G})$
% of the evaluation map 
% $\mathrm{ev}_{X_{\pt_G}}:(X_{\pt_G}=X_{\pt_G})\to X_{\pt_G}(\pt_G)$.  
% In \cref{lem:evisinjwhentransitive} %{lem:conistrans} 
% we proved that (since $X_{\pt_G}$ is transitive) $\mathrm{ev}_{X_{\pt_G}}$ is an injection.  Hence the preimage is a proposition, but since it contains $(\refl{X_{\pt_G}},\refl{\pt_{\pt_G}})$ it is contractible. 

Evoking univalence we get an identification of subgroups between the kernel of $f$ and $(\mathrm{ass}(N),i_N,!)$.
\end{proof}


\begin{remark}
  Where did we use that $N$ was a fixed point of the $G$-set $\typesubgroup_G'$?  If $Y:BG\to\Set$ is a transitive $G$-set and $\pt_H:Y(\pt_G)$, then surely we could consider the group $W$ defined as the component of the groupoid of $G$-sets containing and pointed at $Y$.  The first problem is that we wouldn't know how to construct a homomorphism from $G$ to $W$ which we then could consider the kernel of.  The second problem is that we'd be stuck at the very end where we 
used \cref{lem:evaliseqwhennormal} to show that the evaluation map is an equivalencs; if we only had transitivity we could use \cref{lem:evisinjwhentransitive} to pin down injectivity, but surjectivity needed the extra induction freedom. 
\end{remark}

Summing up, using the various interpretations of subgroups, we get the following list of equivalent sets all interpreting what a normal subgroup is.  
%The explicit equivalences are left out of the statements.
\begin{lemma}
  \label{lem:characterizations of normal}
  Let $G$ be a group, then the following sets are equivalent
\begin{enumerate}
\item The set $\epi_G$ of surjective homomorphisms from $G$,
\item the set $\typenormal_G$ of kernels of surjections from $G$,
\item the set $\typenormal_G'$ of fixed points of the $G$-set $\typesubgroup_G'$,
\item the set of fixed points of the $G$-set $\typesubgroup_G$
\item the set of fixed points of the $G$-set of abstract subgroups of $\abstr(G)$ of \cref{lem:conjugationabstractly}.
\end{enumerate}
\end{lemma}


% The associated normal subgroup defines an equivalence from $\typenormal_G'$ to the type $\typenormal_G$ of kernels of surjective homomorphism. To see this we construct an inverse.

% \begin{definition}
%   \label{def:kerneltofixedpoint}
%   Let $f:\Hom(G,G')$ be a surjective homomorphism.  For $y,z:BG$ consider the set $X_y^f(z)\defequi (f(z)=f(y))$ and the element $\pt^f_y\defequi\refl{f(y)}:X_y(y)$.  The $G$-set $X_y:BG\to\Set$ is transitive since $f$ is surjective and so we have a map 
% $$\mathrm{ssa}:\typenormal_G\to\typenormal_G',\qquad \mathrm{ssa}(f)(y)\defequi(X_y^f,\pt_y^f,!).$$
% \end{definition}

% \begin{lemma}
%   \label{lem:characterizations of normal}
%   Let $G$ be a group, then the associated normal subgroup is an equivalence
%   $$\mathrm{ass}:\typenormal_G'\to\typenormal_G$$
% with inverse $\mathrm{ssa}$.  Summing up the following sets are equivalent
% \begin{enumerate}
% \item The set $\typenormal_G$ of kernels of surjections from $G$,
% \item the set $\typenormal_G'$ of fixed points of the $G$-set $\typesubgroup_G'$,
% \item the set of fixed points of the $G$-set $\typesubgroup_G$
% \item the set of fixed points of the $G$-set of abstract subgroups of $\abstr(G)$.
% \end{enumerate}
% \end{lemma}
% \begin{proof}
%   The last three entries are equivalent since they are the fixed points of equivalent $G$-sets, so we only need to comment on the first assertion.

% Let $f:\Hom(G,G')$ be a surjective homomorphism.  
% The kernel $N\oldequiv \ker f$ is then given by the first projection 
% $$\text{pr}:\sum_{z:BG}\pt_{G'}=Bf(z)\to_*BG.$$
% Then $\mathrm{ass\,ssa}(\ker f)$ defined to be
% $$
% (\sum_{z:BG}\prod_{y:BG}
% (\pathsp{f(\pt_G)}^{G'}{f(y)}=\pathsp{f(z)}^{G'}{f(y)},
% \text{pr}, (\pt_G,\refl{\pathsp{f(\pt_G)}^{G'}{f(\pt_G)}},!),$$
% where $\pathsp{a}^{G'}{b}\defequi (a=b)$ for $a,b:BG'$.
% The desired identification between $\ker f$ and $\mathrm{ass\,ssa}(\ker f)$ is then given by composing the identifications
% $\preinv(p_f):(\pt_{G'}=f(z))=(f(\pt_G)=f(z))$ and 
% $$\preinv:(f(\pt_G)=f(z))=
% \prod_{y:BG}(\pathsp{f(\pt_G)}^{G'}{f(y)}=\pathsp{f(z)}^{G'}{f(y)})$$
% \footnote{((find ref where this was demonstrated: slight modification since we only claim naturality in $G$)).}
% \end{proof}



% There are many valuable constructions to be extracted from the proof of \cref{lem:normalsarekernels}.
% \begin{definition}
%   \label{def:associatedquotient}
% \end{definition}
% \footnote{COMEBACK 190509 Do the converse and elevate constructions to definitions}



\sususe{The pullback}
\label{sec:pullback}

\begin{definition}
  \label{def:pullback}
  Let $B, C, D$ be types and let $f:B\to D$ and $g:C\to D$ be two maps.  
The \emph{pullback}\index{pullback} of $f$ and $g$ is the type 
$$\prod(f,g)\defequi\sum_{(b,c):B\times C}(f(b)=_Dg(c))$$
together with the two projections $\prod(f,g)\to B$ and $\prod(f,g)\to C$ sending $(b,c,p):\prod(f,g)$ to $b:B$ or $c:C$.  If $f$ and $g$ are clear from the context, we may write $B\times_DC$ instead of $\prod(f,g)$ and summarize the situation by the diagram
$$\xymatrix{B\times_DC\ar[r]\ar[d]&C\ar[d]^g\\B\ar[r]^f&\,D.}$$
\end{definition}
\begin{xca}
  \label{xca:univpropofpullback}
  Let $f:B\to D$ and $g:C\to D$ be two maps with common target.  If $A$ is a type show that 
  \begin{align*}
    (A\to B)\times_{(A\to D)}(A\to C)\to &(A\to B\times_DC)\\ 
(\beta,\gamma,p:f\beta=g\gamma)\,\mapsto\,&(a\mapsto (f(a),g(a),p(a):f\beta(a)=g\gamma(a)))
  \end{align*}
 is an equivalence.
\end{xca}

\begin{example}
  If $g:\bn 1\to D$ has value $d:D$ and $f:B\to D$ is any map, then $\prod(f,g)\oldequiv B\times_D\bn 1$ is equivalent to the preimage $f^{-1}(d)\defequi\sum_{b:B}d=f(b)$.
\end{example}
\begin{example}
  \label{ex:pullbackandgcd}
  Much group theory is hidden in the pullback.  For instance, the greatest common divisor $\gcd(a,b)$ of $a,b:\NN$ is another name for the number of components you get if you pull back the $a$-fold and the $b$-fold cover of the circle: as we will see in \cref{lem:iso2} we have a pullback
$$\xymatrix{S^1\times C_{\gcd(a,b)}\ar[d]\ar[r]& S^1\ar[d]^{(-)^b}\\
S^1\ar[r]^{(-)^a}&\,S^1}
$$ 
(where $C_n$ was the cyclic group of order $n$).
To get a geometric idea, think of the circle as the unit circle in the complex numbers so that the $a$-fold cover is simply taking the $a$-fold power.  With this setup, the pullback should consist of pairs $(z_1,z_2)$ of unit length complex numbers with the property that $z_1^a=z_2^b$.  Let $a=a'G$ and $b=b'G$ where $G=\gcd(a,b)$. Taking an arbitrary unit length complex number $z$, then the pair $(z^{b'},z^{a'})$ is in the pull back (since $a'b=ab'$).  But so is $(\zeta z^{b'},z^{a'})$, where $\zeta$ is any $G$-th root of unity.  Each of the $G$-choices of $\zeta$ contributes in this way to a component of the pullback.  In more detail: identifying the cyclic group $C_G$ of order $G$ with the group of $g$-th roots of unity, the top horizontal map $S^1\times C_G\to S^1$ sends $(z,\zeta)$ to $z^{a'}$ and the left vertical map sends $(z,\zeta)$ to the product $\zeta z^{b'}$.  

Also the least common multiple is hidden in the pullback; in the present example it is demonstrated that the map(s) accross the diagram makes each component of the pullback a copy of the subgroup $a'b\ZZ$ of $\ZZ$.
\end{example}


\begin{definition}
  \label{def:intersectionand unionofsets}
  Let $S$ be a set and consider two subsets $A$ and $B$ of $S$ given by two families of propositions (for $s:S$) $P(s)$ and $Q(s)$.  The \emph{intersection}\index{intersection! of sets} $A\cap B$ of the two subsets is given by the family of propositions $P(s)\times Q(s)$.  The \emph{union}\index{union of sets} $A\cup B$ is given by the set family of propositions $A(s)+B(s)$.  
\end{definition}
\begin{xca}
  \label{xca:intersectionpullbackofsets}
  Given two subsets $A$, $B$ of a set $S$, prove that
  \begin{enumerate}
  \item The pullback $A\times_SB$ maps by an equivalence to the intersection $A\cap B$,
  \item\label{xca:cardinalityintersectionunion} 
    If $S$ is finite, then the sum of the cardinalities of $A$ and $B$ is equal to the sum of the cardinalities of $A\cup B$ and $A\cap B$.
  \end{enumerate}
\end{xca}

\begin{definition}
  \label{def:intersectionofgroups}
  Let $f:\Hom(H,G)$ and $f':\Hom(H',G)$ be two homomorphisms with common target.  The \emph{pullback}\index{pullback!of groups} $H\times_GH'$ is the group obtained as the (pointed) component of 
$$\pt_{H\times_GH'}\defequi(\pt_H,\pt_{H'},p_{f'}p_f^{-1})$$ of the pullback $BH\times_{BG}BH'$ (where $p_f:\pt_G=f(\pt_H)$ is the name we chose for the data displaying $f$ as a pointed map, so that $p_{f'}p_f^{-1}:f(\pt_H)=f'(\pt_{H'})$).

If $(H,f,!)$ and $(H',f',!)$ are subgroups of $G$, then the pullback is called the \emph{intersection}\index{intersection! of subgroups} and if the context is clear denoted simply $H\cap H'$.
\end{definition}
\begin{example}
  If $a,b:\NN$ are natural number with least common multiple $L$, then $L\ZZ$ is the intesection $a\ZZ\cap b\ZZ$ of the subgroups $a\ZZ$ and $b\ZZ$ of $\ZZ$. 
\end{example}

\begin{xca}
  Prove that if $f:\Hom(H,G)$ and $f':\Hom(H',G)$ are homomorphisms, then the pointed version of \cref{xca:univpropofpullback} induces an equivalence
$$(\pt_{H}=\pt_{H})\times_{(\pt_{G}=\pt_{G})}(\pt_{H'}=\pt_{H'})
\simeq (\pt_{H\times_GH'}=\pt_{H\times_GH'})
$$
(hint: set $A\defequi S^1$, $B\defequi BH$, $C\defequi BH'$ and $D\defequi BG$).  Elevate this equivalence to a statement about abstract groups.
\end{xca}

\begin{xca}
  If $\mathcal G$ is an abstract group and $\mathcal H$ and $\mathcal K$ are abstract subgroups.  Give a definition of the intersection $\mathcal H\cap\mathcal K$ is the abstract subgroup of $\mathcal G$ agreeing with our definition for groups.
\end{xca}
\begin{lemma}
  \label{lem:whatSylow2needs}
  Let $f:\Hom(G,G')$ be a surjective homomorphism with kernel $N$ and let $H$ be a subgroup of $G$.  Then
  %\begin{enumerate}
  %\item 
$N\cap H$ is a normal subgroup of $H$
%  \item The 
and the induced homomorphism $H/N\cap H\to G'$ is injective.
  % \item If $H$ and $G'$ are finite with coprime cardinalities, then $H$ is a subgroup of $N$.
%  \end{enumerate}
  \begin{proof}
Let $i:\Hom(H,G)$ be the inclusion.  We will show that $N\cap H$ is the kernel of the composite $fi:\Hom(H,G')$.  

Now, $N$ is the kernel of the surjective homomorphism $f$, giving an equivalence between $BN_\div$ and the preimage 
$$(Bf)^{-1}(\pt_{G'})\defequi\sum_{y:BG}\pt_{G'}=Bfy.$$  
Writing out the definition of the pullback (and using that for each $x:BH$ the type $\sum_{y:BG}y=Bix$ is contractible), we get an equivalence between $BN\times_{BG}BH$ and 
$$B(fi)^{-1}(\pt_{G'})\defequi\sum_{x:BH}\pt_{G'}=B(fi)x,$$  
the preimage of $\pt_{G'}$ of the composite $B(fi):BH\to BG'$.
 By definition, the intersection $B(N\cap H)$ is a the pointed component of the pullback containing $(\pt_N,\pt_H)$.  Under the equivalence with $B(fi)^{-1}(\pt_{G'})$ the intersection corresponds to the component of $(\pt_H,Bf(p_i)\,p_f)% :\sum_{x:BH}\pt_{G'}=B(fi)x
 $.  
Since (by definition of the composite of pointed maps) $p_{fi}\defequi Bf(p_i)\,p_f$ we get that the intersection $N\cap H$ is identified with the kernel of the composite $fi:\Hom(H,G')$.
%    \item 

Finally, since $N\cap H$ is the kernel of the composite $fi:\Hom(H,G')$, under the equivalence of \cref{lem:countinggps}, $N\cap H$ is equivalent to the kernel of the induced surjective homomorphism $\widetilde {fi}:\Hom(H,\image (fi))$.  Otherwise said, the quotient group $H/(N\cap H)$ is another name for $\image (fi)$ which indeed injects into $G'$.
%    \end{enumerate}
  \end{proof}
\end{lemma}



\footnote{\color{blue}  \color{blue}THE REST OF THE CHAPTER IS IN FLUX AND I DID NOT GET TO FIX ALL I WANTED TO FIX BEFORE I HAD TO QUIT FOR NOW.  IT CONTAINS KNOWN NONSENSE \tiny Don't actually seem to need at present; hence put on hold
\begin{lemma}
  \label{lem:iso2}
  Let $G$ be a group with subgroups $(H,i_H,!)$ and $(N,i_N,!)$ where $N$ is normal. Let $i_{HN}:Hom(H \vee N,G)$ be the homomorphism from the sum $H\vee N$ of \cref{def:sumofgroup} to $G$ induced by $i_H$ and $i_N$.  Then 
the pullback $BN\times_{BG}BH$
is equivalent to $G/{HN}\times B(H\cap N)$.\footnote{display the equivalence}
\end{lemma}
\begin{proof}
  ((WRITE)) Can alternatively be phrased as the fiber of $B(H\ltimes N)\to BG$ once semi-direct product has been discussed.
\end{proof}
}%endcolor

\begin{lemma}
  \label{lem:thereisaconjugate}
  Let $G$ be a group, $X$ a $G$-set, $x:X$ and $H=(H,i,!)$ a subgroup of $G$.  If $y$ is an element in the orbit of $x$ s.t. $H\subseteq Stab_y$  (\ie $y$ is an $H$-fixed point), then there is a conjugate $H'=(H',i',!)$ of $H$ with $H'\subseteq Stab_x$.
\end{lemma}
\begin{proof}
  CLASSICAL: There is a $g:G$ s.t. $y=g\cdot x$ and for all $h:H$ we have $h\cdot y=y$.  Define $H'=g^{-1}Hg$.  If $h':H'$, then $h'=g^{-1}hg$ for a unique $h:H$ and
$$h'\cdot x=(g^{-1}hg)\cdot x = g^{-1}\cdot(h\cdot (g\cdot x))=g^{-1}\cdot(h\cdot y)=g^{-1}\cdot y = x.$$
\end{proof}



\begin{definition}
  \label{def:normalizer}
\footnote{TO BE MOVED TO or AFTER the chapter on symmetry (need Burnside -  a $G$-set splits into orbits - etc) has been covered.  
Also, some proofs are written in a pseudoclassical way just to remind me of the idea.  It is not a typo and will by typied}
An element of the $G$-orbit of a subgroup $(H,i_H,!)$ are called a \emph{conjugate} of $(H,i_H,!)$.   The stabilizer group of a subgroup $(H,i_H,!)$ is called the \emph{normalizer $N_G(H)$ of $H$ in $G$}.\index{normalizer}

If $(K,i_K,!)$ is another subgroup containing a conjugate of $(H,i_H,!)$, we say that $(H,i_H,!)$ is \emph{subconjugate} to $(K,i_K,!)$.
\end{definition}
Recall that we defined a normal subgroup as a kernel of a homomorphism (which we may assume is surjective by replacing the target with the image without changing the kernel); we can now give a second characterization:
\begin{lemma}
  \label{lem:normalisfixed}
  Let $G$ be a group.  A subgroup of $G$ is normal if and only if it is a fixed points under the conjugation action.
\end{lemma}
\begin{proof}
  Consider a surjective homomorphism $f:\Hom(G,G')$ and let $BN\defequi\sum_{z:BG}(\pt_{G'}=Bf(z))$ (pointed at $(\pt_G,p_f)$) represent its kernel, with the first projection to $BG$ representing the injection $i_N:\Hom(N,G)$ (with $\refl{\pt_G}$ the witness that $\pt_G$ is identical to the first projection of $(\pt_G,p_f)$).   Now, by the very representation of $N$, for every $g:\pt_G=\pt_G$ we get an equivalence $C^g:BN\we BN$ by setting $C^g(z,p)\defequi(z,p\,f(g)^{-1})$ with basepoint identity (from $\pt_N\defequi(\pt_G,p_f)$ to $C^g\pt_G\defequi (\pt_G,p_ff(g)^{-1})$) given by $g^{-1}:\pt_G=\pt_G$ and the fact 
$$\xymatrix{\pt_Q\ar@{=}[rr]^{p_f}_\to\ar@{=}[d]_{\refl{\pt_Q}}&&f(\pt_G)\\
\pt_Q\ar@{=}[r]^-{p_f}_-\to&f(\pt_G)&\,f(\pt_G).\ar@{=}[l]^\gets_{f(g)}\ar@{=}[u]^\uparrow_{f(g)}}
$$
Since $C^g$ followed by the first projection is exactly the first projection and also the base points match up (\ie $\refl{pt_G}\circ\mathrm{pr}_1(g^{-1},!)=_{\pt_G=\mathrm{pr}_1(\pt,p_f)}g^{-1}$) we get an identity $(N,Bi_N,\refl{\pt_G},!)=_{\typesubgroup_G}(N,Bi_N,g^{-1},!)$, showing that the normal subgroup is a fixed point. 

Conversely, let $(H,i,!)$ be any subgroup of $G$ and consider the pointed component $BW$ of the type of $G$-sets containing the cokernel $G/H$.  If $X$ is a $G$-torsor, then the orbit $X/H$ is a $G$-set in $BW$, \footnote{((explain how you transport the $G\times G$-action from $G$ or deloop))}
providing us with a pointed map $f:BG\to BW$.
By \cref{lem:aut-orbit} \footnote{((which has yet to be provided with a proof))} the identity type $G/H=G/H$ is 


% Let $(H,F,p,!):\typesubgroup_G$ be a fixed point, \ie for all $g:\pt_G=\pt_G$ there is an identity $C^g:H=H$ so that 
% $$\xymatrix{}
% $$
\end{proof}

\begin{lemma}
  Let $(H,i_H,!):\typesubgroup_G$ and let $N_G(H)$ be the normalizer subgroup of $H$ of \cref{ex:abstrandconj} (considered as a subgroup $(N_G(H),i_{N_G(H)},!)$ of $G$).  Then $H$ is a normal subgroup of $N_G(H)$.  ((come back and prove normal))
\end{lemma}
\begin{proof}
  Remember that $N_G(H)$ is the stabilizer subgroup of $(H,i_H,!)$ under the conjugation action, so to prove that $H$ is a subgroup of $N_GH$ we need to show that if $h:\pt_H=\pt_H$, then there is an identity between $(H,i_H,!)$ and $c^{i_H(h)}(H,i_H,!)$.   
Let $g\defequi i_H(h)$.  If $s:\pt_H=\pt_H$, then $c^gi_H(s)=g\,i_H(s)\,g^{-1}=i_H(h)\,i_H(s)\,i_H(h)^{-1}=i_H(h\,s\,h^{-1})=i_H(c^hs)$ (since $i_H$ is a homomorphism).  
Using the identity $c^h:H=H$ we have obtained an identity $(H,i_H,!)=(H,c^{i_H(h)}i_H,!)$.
\end{proof}

\section{Historical remarks}
\label{sec:grouphistory}

% Move in place

% \begin{remark}
%   Notice that the last statement  (``More precisely\dots'')  not only asserts that there \emph{exist} inverses, but that there actually is a (preferred and consistent) way to produce them.

% Classically this was in many instances unnecessay to say because there was a unique inverse, and the distinction is not mentioned in introductory texts.  However, then this very point had to be revisited later on.  In our proof relevant setting it is obvious that the ultimate statement will have to go beyond an assertion that inverses exist.
% \end{remark}

%%% Local Variables:
%%% mode: latex
%%% fill-column: 144
%%% TeX-master: "book"
%%% End:


%the below is the illustration used for the n-fold \covering in the deck trafo section.
% Move in place
% \begin{figure}
%   \centering
%   \begin{tikzpicture}
%     \node (A) at (2,2) {$\sqrt[n]X$};
%     \node (B) at (2,-2) {$\bn{n}$};
%     \draw[->] (A) -- node[auto] {$p$} (B);
%     \foreach \y in {-2,0,1,2}
%     { \begin{scope}[shift={(0,\y)}]
%         \foreach \x in {0,...,4}
%         { \node[fill,circle,inner sep=1pt] at (180+72*\x:1 and .3) {}; }
%         \foreach \x in {0,...,3}
%         { \draw[-stealth] (180+72*\x:1 and .3) arc(180+72*\x:252+72*\x:1 and .3); }
%       \end{scope} }
%     \begin{scope}[shift={(0,-2)}]
%       \draw[-stealth] (108:1 and .3) arc(108:180:1 and .3);
%     \end{scope}
%     \foreach \y in {1,2}
%     { \begin{scope}[shift={(0,\y)}]
%         \draw[-stealth] (108:1 and .3)
%         .. controls ++( 5:-.3) and ++(80:.2) .. (-.7,-.4)
%         .. controls ++(80:-.2) and ++(90:.2) .. (-1,-1);
%       \end{scope} }
%     \draw[-stealth] (108:1 and .3)
%     .. controls ++( 5:-.3) and ++(80:.2) .. (-.7,-.4);
%     \node (dz) at (-.7,-.7) {\footnotesize $\vdots$};
%     \begin{scope}[shift={(0,3)}]
%       \draw[-stealth] (-.7,-.4)
%       .. controls ++(80:-.2) and ++(90:.2) .. (-1,-1);
%     \node (da) at (-.7,0) {\footnotesize $\vdots$};
%     \end{scope}
%   \end{tikzpicture}
%   \caption{The $n$'th root of an endomorphism, with projection}
%   \label{fig:rootproj}
% \end{figure}
