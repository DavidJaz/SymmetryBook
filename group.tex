\label{ch:groups}



%\section{Now it starts}
The identity type is not just any type:  in the previous sections we have seen that the identity type $a=_Aa$ reflects the ``symmetries'' of an element $a$ in a type $A$.  
Symmetries have special properties; for instance you can rotate a square by $90^o$, and you can rotate it by $-90^o$, undoing the first rotation.
Symmetries can also be composed, and this composition respects certain rules that holds in all examples.  One way to study the concept of ``symmetries'', would be to isolate the common rules for all our examples, but also show, conversely, that anything satisfying these rules actually \emph{is} an example. 



%As an instance of a property that holds in \emph{some} examples but not in others, we have seen that sometimes the order in which we use our symmetries matters, and sometimes it does not, see \cref{ch:intro}.  Hence, the concept of a group should not have a rule allowing you to change the order arbitrarily.

With inspiration of geometric and algebraic origins, it became clear to mathematicians at the end of the 19'th century that the properties of such symmetries could be codified by saying that they form an abstract \emph{group}. 
In \cref{sec:identity-types} we saw that the identity type was ``reflexive, symmetric and transitive'' -- and an abstract group is just a set with such operations satisfying certain rules.

%This is the purpose of the mathematical term ``group''.

We attack the issue more concretely; instead of focusing on the abstract properties we promote the types exhibiting the symmetries, and the rules follow from the rules for identity types without needing us to worry about them.  However, we \emph{will} show that the two approaches give the same end result.  

In this chapter we lay the foundations and provide some basic examples of groups.  ((describe the contents of the chapter))

\section{The type of groups}
\label{sec:typegroup}

\begin{example}\label{ex:base=base}
  We defined the circle $S^1$ in \cref{def:circle} by declaring that it has a point $\base$ and an element $\Sloop:\base=_{S^1}\base$, and we proved in \cref{cor:S1groupoid} that $\base=_{S^1}\base$ is equivalent to the set $\zet$ (of integers), where $n\in\zet$ corresponds to the $n$-fold composition of $\Sloop$ (which works for both positive and negative $n$).  
We can think of this as describing the symmetries of $\base$: we have one ``generator'' $\Sloop$, and this can be applied any number of times, giving a new symmetry for each new number.  
Here, composition of loops corresponds to usual addition of integers.  Hence, the circle is a very cheap packaging of the ``{group}'' of integers, the declaration of $\base$ and $\Sloop$ not only gives the \emph{set} $\zet$ of integers, but at the same time the addition.
\end{example}
\begin{example}
  Recall the finite set $\bn{2} =\bool:\fin_2$ from \cref{def:finiteset}, containing two elements.   
According to \cref{xca:C2}, $\bn{2} =\bn{2} $ has exactly two elements, $\refl{\mathrm 2}$ and $\twist$, and doing $\twist$ twice gives you back $\refl{\bn{2} }$.  
We see that this is exactly all the symmetries you'd expect to have in a two point set: you can let everything be ($\refl{\bn{2} }$) or you could swap the two elements ($\twist$); and if you swap twice everything is let be.  
The type $\fin_2$ (of ``finite sets with two elements'') is our embodyment of these symmetries.  

Observe that (by the definition of $S^1$) there is an interesting function $S^1\to\fin_2$, sending $\base:S^1$ to $\bn{2} :\fin_2$ and $\Sloop$ to $\twist$.
\end{example}


The examples Klein and Lie were interested in were of a type making it admissible to say that a group is the identity type $a=_Aa$ for \emph{some} type $A$ and \emph{some} element $a:A$.
However, in elementary texts it is customary to restrict the notion of a group to the case when $a=_Aa$ is a \emph{set} as we will do, starting in \cref{sec:identity-type-as-abstract}.  This makes some proofs easier, since if are we given two elements $g,h:a=_Aa$, then the identity type $g=h$ is a proposition, \ie $g$ can be equal to $h$ in at most one way.  Hence questions relating to uniqueness will never be a problem.



See \cref{sec:grouphistory} for a brief summary of the early history of groups.
\begin{remark}
  The reader may wonder about the status of the identity type $a=_Aa'$ where $a,a':A$ are different elements.  One problem is of course that if $p,q:(a=_Aa')$ there is no obvious way of composing $p$ and $q$, and another is that $a=_Aa'$ does not have a distinguished element such as $\mathrm{refl{}_a}:a=_Aa$.
Given $f:a=_Aa'$ we can use transport along $f$ to compare $a=_Aa'$ with $a=_Aa$ (much as affine planes can be compared with the standard plane or a finite dimensional real vector space is isomorphic to some Euclidean space), but absent existence and choice of such an $f$ the identity types $a=_Aa'$ and $a=_Aa$ are different animals.  We will return to this example when we've defined torsors.
\end{remark}


\begin{remark}
  When considering the identity type $a=_Aa$, only the elements $x:A$ with $x$ equal to $a$ are relevant, and we are free to consider only \emph{connected} $A$, \ie where $x=_Aa$ is never empty (c.f.~\cref{def:connected}).  Also, our preference for $a=_Aa$ to be a set indicates that we should consider only the connected types $A$ that are \emph{groupoids}.
\end{remark}


With this established, we let the \emph{type} of groups be defined as follows:

\begin{definition}\label{def:typegroup}
%\footnote{we must define  $\isset$ and propositional truncation.  Alternatively we must define $\isonetype$ and $\conn$}
  A \emph{group} is a pointed connected groupoid; the \emph{type of groups} is the type 
%$$\typegroup=\sum_{A:\UU}A\times\isonetype(A)\times \conn_0A.$$
$$\typegroup\defequi\sum_{A:\UU}\sum_{a:A}\isset(a=_Aa)\times\prod_{x:A}||x=_Aa||$$
of pointed connected groupoids.
%We refer to an element of $\typegroup$ as a \emph{group}.  
A group $G=(A,a,p,q):\typegroup$ will be referred to simply as $$\aut_A(a).$$  The underlying pointed type $$BG\defequi(A,a)$$ is referred to as the \emph{classifying space of $G$}.  The element $\pt_G\defequi a$ will be referred to as the \emph{base point}. 
\end{definition}
Informally, we may also refer to the type $BG_\div\defequi A$ as the classifying space of $G$.
\begin{remark}\label{rem:aut}
There is no ambiguity in writing $\aut_A(a)$ instead of $(A,a,p,q)$: being a connected groupoid is asserted by 
$$\isset(a=_Aa)\times\prod_{x:A}||x=_Aa||$$ which is a proposition  (\cref{lem:props-are-props}) and so the witness $(p,q)$  is unique.  In this sense, once you know that the classifying space is a connected groupoid, $BG$ carries all the information about $G$: $$G\oldequiv\aut_{BG_\div}(pt_{G}).$$
\end{remark}
\end{definition}
\subsection{First examples}
\label{sec:firstgroupexamples}
   \begin{example}\label{excirclegroup}
   The circle $S^1$, which we defined in \cref{def:circle}, is a connected groupoid (\cref{lem:circleisconnected}, \cref{cor:S1groupoid}) and is pointed at $\base$. The identity type $\base=_{S^1}\base$ is equivalent to to the set of integers $\zet$ and composition corresponds to addition.  This justifies our definition of the \emph{group of integers} as 
$$\ZZ=\aut_{S^1}(\base).$$
It is noteworthy that along the way we gave several versions of the circle, each of which has its own merits, the version in \cref{def:S1toC}
$$C=(\sum_{X:\UU}\sum_{f:X=X}||(\zet,s)=(X,f)||, (\zet,s))$$
being a very convenient one.
 \end{example}

\begin{example}\label{ex:groups}
  % Since any pointed connected groupoid is a group, there is no shortage of examples, but perhaps i
  Apart from the circle, there are some important groups that come almost for free: namely the symmetries in the type of sets.
%It is worthwhile to consider some specially designed examples.
  \begin{enumerate}
  \item Recall that the set $\bn{1} =\true$ has the single element which we can call $*$. Then $\aut_{\bn{1} }(*)$ is a group called the \emph{trivial group}.
  \item If $n:\NN$, then the \emph{permutation group of $n$ letters} is 
$$\Sigma_n\defequi\aut_{\fin_n}(\bn{n} ),$$ 
where $\fin_n$ is the groupoid of sets of cardinality $n$ (c.f.~\ref{def:finiteset}).  Note that even though the sets $\bn{n} =_{\fin}\bn{n} $ and $\bn{n} =_{\fin_n}\bn{n} $ are equal, we must use the component $\fin_n$ rather than the entire groupoid $\fin$ of finite sets to keep the underlying pointed groupoid $B\Sigma_n=(\fin_n,\bn{n} )$ connected.
  \item More generally, if $S$ is a set, is there a pointed connected groupoid $(A,a)$ so that $a=_Aa$ models all the ``permutations'' $S=_{\Set}S$ of $S$?  Again, the only thing wrong with ``$\aut_{\Set}(S)$'' (apart from $\Set$ being large\footnote{how do we deal with that?}) is that $\Set$ is not connected. 
%}!\footnote{it's so simple -- so very simple -- that only a child can do it!}  To be precise, the component of $S$ is
%$$A\defequi\sum_{X\in\Set}||S=X||.$$  
%The connected groupoid $\sum_{X\in\Set}||S=X||$ is pointed at $S$ (and the fact that $S=S$ is nonempty since $\refl S:S=S$).    
% Then 
% $$(S=_AS)=(S=_{\Set}S)$$ 
% (in the identity type of a $\Sigma$-type both the first and the second projections must be equal, but for $A\oldequiv\sum_{X:\Set}||S=X||$ the second projection is a proposition).  
%
 So, 
the \emph{group of permutations of $S$} is defined to be $\Sigma_{S}=\aut_{\Set_{(S)}}(S)$.  

Likewise, if $X$ is any type, the \emph{group of automorphism} or \emph{permutations} of $X$ is defined to be 
$$\Sigma_X=\aut_{\UU_{(X)}}X,$$
 where $U_{(X)}$ is the component of $\UU$ containing $X$.
  \end{enumerate}
\end{example}
\begin{example}\label{ex:cyclicgroups}
  In \cref{thm:coveringsofS1} we studied the symmetries of the ``$m$-fold \covering'' of the circle for $m$ a positive integer, and showed that there were $m$-different symmetries, but that they were just the powers $f^n$ (for $n=0,1,\dots,m-1$) of one (nonunique) symmery $f$ and that $f^{m+k}=f^k$ for any integer $k$.  This very important symmetry pops up in many situation, and is called the \emph{cyclic group of order $m$}.  In other words, the cyclic group of order $m$ is the (pointed) component of the type of \coverings of the circle containing the $m$-fold \covering.  We analyzed this in \cref{thm:coveringsofS1} and found that this (pointed) component was equivalent to 
$$BC_n\defequi(\sum_{X:\Set}\sum_{p:X=X}||(X,p)=\zet/m||,(\zet/m,!)).$$

Here is another, and occasionally more convenient, way of obtaining the cyclic group of order $n$.  Consider the function $$cy_n:S^1\to\fin_n$$ with $cy_n(\base)\defequi\bn n$ and $cy_n(\Sloop):\bn n=\bn n$ the identity corresponding to the equivalence given by cyclic permutation, sending $n-1$ to $0$ and, for $0\leq i<n-1$, sending $i:\bn n$ to $i+1$.  Note that the identity $cy_n\Sloop$ is cyclic in the sense that the $n$-fold iterate $(cy_n\Sloop)^n$ is $\refl{\bn n}$.  Then the $n$-fold \covering can be seen as the first projection 
$$\sum_{z:S^1}cy_n(z)\to S^1.$$

Consider the type 
$$B_n\defequi\sum_{S:\fin_n}\pi_0(cy_n^{-1}(S)),\footnote{I've used set truncation!!!}$$
the ``image'' of $cy_n$ except that the truncation is one level higher than we have considered before.  Since $\fin_n$ is a groupoid, $B_n$ is a groupoid.  Remember that $cy_n^{-1}(S)=\sum_{z:S^1}S=cy_nz$.  Let $(S,z,p):B_n$ be any element; we want to show that $(S,z,p)=_{B_n}(\bn n, \base, \refl\base)$ is not empty, so that $B_n$ is connected. Since $S^1$ is connected there is a $q:z=_{S^1}\base$ so $(cy_n(q)\,p,q,!):(S,z,p)=(\bn n,\base,\refl\base)$ (using that $cy_n(\base)\defequi\bn n$ to compose $p:S=cy_nz$ and $cy_n:cy_nz=cy_n\base$), $B_n$ is connected.  
Pointing $B_n$ in $(\bn n,!)$ we have a pointed connected groupoid, \ie a group, which we call the {\em cyclic group $C_n$ of order $n$}.  (in hindsight, $B_n=(BC_n)_\div$)

Note that the cyclic group of order $1$ is the trivial group, the cyclic group of order $2$ is equivalent to the permutation group $\Sigma_2$: there are exactly one nontrivial symmetry $f$ and $f^2$ is the identity.  When $m>2$ the cyclic group of order $m$ is a group that does not appear elsewhere in our current list.  In particular, the cyclic group of order $m$ has only $m$ different symmetries, whereas we will see that the group of permutations $\Sigma_m$ has $m!=1\cdot 2\cdot\dots\cdot m$ symmetries.
\end{example}
\begin{example}\label{ex:productofgroups}
  If you have two groups $G$ and $H$, their {\em product} $G\times H$ is given by taking the product of their classifying spaces:
$$G\times H\defequi\aut_{BG_\div\times BH_\div}((\pt_G,\pt_H))$$
(note that $B(G\times H)\oldequiv BG\times BH$ is pointed in $\pt_{G\times H}\oldequiv(\pt_G,\pt_H)$).  
For instance, $\Sigma_2\times\Sigma_2$ is called the {\em Klein group}.
\end{example}

\footnote{MANY MORE EXAMPLES.  We might tone down exercises like ``prove that $\typegroup$ is a groupoid'', even though we will want to use these results.  They take the geometry/fun out of the exposition.}
\begin{xca}
  \begin{enumerate}
  \item Compare the definitions of \cref{def:finiteset} and show that if $n:\NN$, then $\Sigma_n=\Sigma_{\bn{n} }$ %is equal to the permutation group on $n$ letters 
and (since $\fin_0=\fin_1=\bn 1$) that $\Sigma_{1}=\aut_{\bn{1} }(\triv)$.
%\item Display an element in $\bn{2} =_{\fin_2}\bn{2} $ different from $\refl{\bn{2} }$ in the group $\Sigma_{2}$ of permutations of two letters.  
\item Prove that the set $\bn{n} =_{\fin_n}\bn{n} $ is finite of cardinality $n!$.
\item Show that the $n$-fold cover of $S^1$ is equivalent to the first projection $\sum_{z:S^1}cy_n(z)\to S^1$, where $cy_n:S^1\to\fin_n$ is given by $cy_n(\base)\defequi\bn n$ and $cy_n(\Sloop):\bn n=\bn n$ is cyclic permutation (sending $n-1$ to $0$ and, for $0\leq i<n-1$, sending $i:\bn n$ to $i+1$).  Hint: for every $z:S^1$, $cy_nz:\fin_n$ is a finite set of cardinality $n$.  Decidability is not an issue, so you can appeal to our classification of the \coverings of the circle.
\item Show that, given a type $X$, the type of functions $BC_n\to X$ is equivalent to the type 
$$\sum_{f:S^1\to X}\prod_{z:S^1}f(z)=f(z^n)$$ of functions $f:S^1\to X$ such that the two ways around
$$\xymatrix{S^1\ar[d]_{(-)^n}\ar[dr]^f&\\S^1\ar[r]^f&X}$$
agree.  Hint: define the function $F_1:(BC_n\to X)\to (S^1\to X)$ by precomposition: $F_1(g)(z)=g(cy_nz,!)$ and observe that since $cy_n(z)=cy_n(z^n)$ we have a function $F:(BC_n\to X)\to \sum_{f:S^1\to X}\prod_{z:S^1}f(z)=f(z^n)$.
  \end{enumerate} 
\end{xca}

\begin{remark}
In \cref{lem:idtypesgiveabstractgroups} we will see that groups satisfy a set of laws justifying the name ``group''
%we may associate an abstract group $(a=_Aa,e,{-}^{-1},\cdot)$
and we will later show that groups are uniquely characterized by these laws.
\end{remark}
Some groups have the property that the order you perform the symmetries is immaterial.  The prime example is the group of integers $\ZZ\oldequiv \aut_{S^1}(\base)$  Any symmetry is of the form $\Sloop^n$ for some integer $n$, and if $\Sloop^m$, then $\Sloop^n\Sloop^m=\Sloop^{n+m}=\Sloop^{m+n}=\Sloop^m\Sloop^n$.

 Such cases are important enough to have their own name:
\begin{definition}\label{def:abgp}
  A group $G$ is \emph{abelian} if %all symmetries commute in the sense that 
the proposition
$$\mathbf{isAb}(G)\defequi\prod_{g,h:\pt_G=\pt_G}gh=hg$$
is true.  In other words, the type of abelian groups is 
$$\mathbf{Ab}\defequi \sum_{G:\typegroup}\mathbf{isAb}(G).$$
\end{definition}
\begin{xca}\label{exer:first examples}
  Show that permutation group $\Sigma_2$ is abelian, but that $\Sigma_3$ is not.  Show that if $G$ and $H$ are abelian groups, then so is their product $G\times H$.
\end{xca}
We can envision $g$ commuting with $h$ by the picture
$$\xymatrix{a\ar@{=}[r]^g_\to\ar@{=}[d]^\downarrow_h&a\ar@{=}[d]^\downarrow_h\\
a\ar@{=}[r]^g_\to&a}$$
and saying that going from (upper left hand corner) $a$ to (lower right hand corner) $a$ by either composition gives the same result.

\begin{remark}
  The condition $\isset(a=_Aa)$ in the definition of the type of groups is sometimes more of a nuisance, and deleting it gives the simpler concept of \aninftygp, see \cref{sec:inftygps}.
\end{remark}
\begin{xca}
   Let $\aut_A(a):\typegroup$ and let $b$ be an arbitrary element of $A$.  Prove that the groups $\aut_A(a)$ and $\aut_A(b)$ are equal.  Similarly for \inftygps when you get that far.
\end{xca}
\begin{remark}\label{rem:monoidandabsgplarge}
 In \cref{def:typegroup} the first $\sum$ in the definition of the type $\typegroup$ ranges over the entire universe $\UU$.  Hence, $\typegroup$ does not belong to $\UU$, but rather to the next universe as discussed briefly in \cref{sec:univax}.   This tendency that the ``type of all the types we are interested in'' is a ``large type'' is a regular feature of the theory and since it will not cause any trouble for us, we will not be consistent in pointing it out.
  \end{remark}

  \begin{xca}\label{xca:typegroupisgroupoid}
    Given two groups $G$ and $H$.  Prove that $G=H$ is a set.   Prove that the type of groups is a groupoid.  This means that, given a group $G$, the component of $\typegroup$, containing (and pointed at) $G$, is again a group, which we will call the \emph{group $\aut(G)$ of automorphisms} of $G$.
  \end{xca}

\section{The identity type as an abstract group }
\label{sec:identity-type-as-abstract}

Studying the identity type leads one to the definition of what a group should be:
Let $A$ be a type, and for the moment let $a=b$ be shorthand for $a=_Ab$ when $a,b:A$.  In \cref{sec:identity-types} we saw that
\begin{enumerate}
\item[R] {\bf Reflexivity.} For any $a:A$ there is an element
$$\refl a{}:a=a$$ 
%(by definition)
\item[S] {\bf Symmetry.} For any $a,b:A$ there is a an element $$\symm{}_{a,b}:(a=b)\to (b=a)$$ defined by $\symm{}_{a,a}(\refl a{})\defequi\refl a{}$
\item[T] {\bf Transitivity.} For any $a,b,c:A$ there is an element $$\trans{}_{a,b,c}:(b=c)\to((a=b)\to(a=c))$$ defined by $\trans{}_{a,a,a}(\refl a{})(\refl a{})\defequi \refl a{}$.
\end{enumerate}
%\footnote{\em\bf I have swapped the order of the input in trans so that it can fit.  I know you hate it and will force me to recant}

 To emulate classical notation, for fixed $a:A$,  %for the moment 
let's write
 \begin{enumerate}
% \item $G$ instead of $a=_Aa$,
 \item $e$ instead of $\refl a{}$
 \item $g^{-1}$ instead of $\symm_{a,a}(g)$, when $g:a=a$
 \item $g\cdot h$ instead of $\trans_{a,a,a}(g)(h)$ when $g,h:a=a$.
 \end{enumerate}
 What properties can we show about this without knowing anything about $A$ and $a$? For convenience, here is a list of the results we are aiming for: for all $g,g_1,g_2,g_3:a=a$ we will construct elements in all the following propositions
 \begin{enumerate}
 \item $g=g\cdot e$  \qquad(``right unit law'')\footnote{redundant (keep).  If you still want to reinsert the other redundant $g\cdot g^{-1}=e$ and $(g^{-1})^{-1}=g$ we have to do some renumbering.  
%Forgot which way you prefereed the equalities: from simple to complicated or the other way around?
}
 \item $g=e\cdot g$ \qquad(``left unit law'')
 \item $g_1\cdot(g_2\cdot g_3)=(g_1\cdot g_2)\cdot g_3$ \qquad(``associativity'')
 \item $e=g^{-1}\cdot g$ \qquad(``inverse'').
 %\item $g\cdot g^{-1}=e$ redundant (remove)
% \item $(g^{-1})^{-1}=g$ redundant (remove)
 \end{enumerate}
 

We do $g=e\cdot g$ in some detail (remember that ``$e$'' is shorthand for $\refl a{}$)
\begin{definition}\label{def:p1}
  Let $A$ be a type and $a, b:A$ and $g:a=b$ be elements.  Then $p_1(a,b,g):g=_{a=b}g\cdot e$ is the element defined by induction by saying that $p_1(a,a,e)$ is $\refl e:e=e\cdot e$.
\end{definition}
\begin{remark}
  This makes sense since we \emph{defined} $e\cdot e\defequi e$ (or, as it was originally formulated, $\trans_{a,a,a}(\refl a{})(\refl a{})\defequi \refl a{}$).  We'll say that we produce $p_1(a,b,g)$ by ``induction on $b$'', the case where $b$ is $a$ (and $g$ is $e$) is the start of the induction; the induction priciple for the identity type $a=b$ then finishes the construction.

As constructed, $p_1$ is actually an element in the type
$$\prod_{a:A}\prod_{b:A}\prod_{g:a=b}(g=g\cdot e)$$ -- it is constructed ``uniformly'' or ``naturally'' for all $a,b,g$: think of it as a function with $(a,b,g)$ as input and $p_1(a,b,g):g=g\cdot e$ as output.

We may add a little meat to the definition of $p_1$: in the definition of the identity type, for each $a:A$ let $P$ be the type family given by $P(b,g)\defequi (g=g\cdot e)$ for each $b:A$ and $g:a=b$.  
According to the definition of the identity type, in order to produce elements in $P(b,g)$ for arbitrary $b$ and $g$ it suffices to give an element in $P(a,e)\oldequiv (e=e\cdot e)$, but $e\cdot e\defequi e$ and so $\refl e:e=e$ will do.
\end{remark}
\begin{definition}\label{def:p3}
  Let $A$ be a type and $a,b,c,d:A$ and $g_3:a=b$, $g_2:b=c$ and $g_1:c=d$ elements.  Then $p_3(a,b,c,d,g_1,g_2,g_3):g_1\cdot(g_2\cdot g_3)=_{a=_Ad}(g_1\cdot g_2)\cdot g_3$ is the element defined by induction by saying that $p_3(a,a,a,a,e,e,e,e)$ is $\refl e:e\cdot(e\cdot e)=(e\cdot e)\cdot e$ [which makes sense since $e\cdot e\defequi e$].
\end{definition}
\begin{remark}
  This definition is somewhat more complicated than the first, in the sense that in order to unravel the induction to exactly the form accepted in the definition of the identity type, we need to apply the rule three times.  %((write out))
\end{remark}
\begin{definition}\label{def:p4}
  Let $A$ be a type and let $a,b:A$ and $g:a=b$ be elements.  Then $p_4(a,b,g):g^{-1}\cdot g=_{a=_Aa} e$ is the element defined by induction by saying that $p_4(a,a,e)$ is $\refl e:e=e\cdot e$ [which makes sense since $e^{-1}\defequi e$ and $e\cdot e\defequi e$].
\end{definition}

\begin{xca}\label{xca:p2}
    Define $p_2(a,b,g)$ %and $p_3(a,b,g)$
by exactly the same procedure, completing the list.
\end{xca}

\begin{remark}
   One may worry about many things when one sees the list ``right unit law, left unit law, associativity, inverse''.  For instance, for the particular case of $g$ being $e$, are the elements in $e=e\cdot e$ given in the left and right unit laws equal?  Since $a=a$ is a set, such worries become irrelevant: $e=e\cdot e$ is then a proposition, so any two elements are equal.
 \end{remark}

These properties of the identity type are bundled together in the concept of an abstract group, under the additional hypothesis that we are dealing with a set.

  \begin{definition}\label{def:abstractgroup}
    An \emph{abstract group} is a set $S$ together with
\begin{enumerate}
\item an element $e:S$,
\item a function taking a pair of elements $g_1,g_2:S$ to a third element which we call $g_1\cdot g_2:S$ such that
  \begin{enumerate}
  \item %$e$ is a ``neutral element'':
if $g:S$, then $g\cdot e=e\cdot g=g$ and
  \item %satisfying ``associativity'':
if $g_1,g_2,g_3:S$, then
$$g_1\cdot(g_2\cdot g_3)=(g_1\cdot g_2)\cdot g_3,$$
  \end{enumerate}
\item %inverses:
to every $g:S$ there is a $g^{-1}:S$ such that $%g\cdot g^{-1}=
e=g^{-1}\cdot g$.
\end{enumerate}
We refer to $e$ as the \emph{unit element}, $g_1\cdot g_2$ as the \emph{product of $g_1$ and $g_2$} and $g^{-1}$ as the \emph{inverse of $g$}.  The \emph{unit laws} will then be $g\cdot e=e\cdot g=g$, the \emph{associativity law} is $g_1\cdot(g_2\cdot g_3)=(g_1\cdot g_2)\cdot g_3$ and $%g\cdot g^{-1}=
g^{-1}\cdot g=e$ is referred to as the \emph{law of inverses}.  The set $S$ is called the \emph{underlying set} of the abstract group.
  \end{definition}

In conclusion we have proved that groups give rise to abstract groups:
  \begin{lemma}\label{lem:idtypesgiveabstractgroups}
    If $G$ is a group, then $\pt_G=\pt_G$ together with $e\defequi\refl{\pt_G}{}$, $g^{-1}\defequi\symm_{\pt_g,\pt_G}g$ and $g\cdot h\defequi\trans_{\pt_g,\pt_G,\pt_G}(g)(h)$
%$A$ is a groupoid %(alternatively called a ``$1$-type'') and $a:A$ is an element, then $a=_Aa$, together with $e\defequi\refl a{}$, $g^{-1}\defequi\symm_{a,a}g$ and $g\cdot h\defequi\trans_{a,a,a}(g)(h)$ 
define an abstract group
$$\abstr(G)\defequi (\pt_G=\pt_G,e,{-}^{-1},\cdot).$$
  \end{lemma}
  \begin{proof}
    The elements $p_1,\dots p_4$ of \cref{def:p1,def:p4,def:p3,xca:p2} show that all the relevant identity types (which are propositions since $A$ is a groupoid) are nonempty, as required.
  \end{proof}
  \begin{definition}\label{def:abstrG}
    Given a group $G$, the abstract group $\abstr(G)\defequi (a=_Aa,e,{-}^{-1},\cdot)$ of \cref{lem:idtypesgiveabstractgroups} is called the \emph{abstract group associated to $G$}.
  \end{definition}

  \begin{remark}
    It is sometimes handy to break up the rather long \cref{def:abstractgroup}  by saying that the right and left unit law together with associativity define a \emph{monoid}, and if we, in addition, have inverses satisfying the law of inverses, then we have an abstract group.
    % \end{remark}


    % \begin{remark}
      \label{rem:typemonoidabstrgp}
        Summing up in language a machine (and the occasional mad scientist) can handle, the \emph{type of monoids} is
$$\typemonoid\defequi \sum_{M:\UU}\sum_{e:M}\sum_{\mu{}:M\to M\to M}
\isset{(M)}\times\mathrm{Monoidlaws}(M,e,\mu)
$$
where
$$\mathrm{Monoidlaws}(M,e,\mu)\defequi\mathrm{Unitlaws}(M,e,\mu)\times\mathrm{Assoclaw}(M,\mu{})$$and
\begin{align*}
  \mathrm{Unitlaws}(M,e,\mu)\defequi\prod_{g:M}
&(g=\mu{}(g)(e))\times(g=\mu{}(e)(g)),\\
\mathrm{Assoclaw}(M,\mu{})\defequi\prod_{g_1,g_2,g_3:M}&\mu{}(g_1)(\mu{}(g_2)(g_3))=\mu{}(\mu{}(g_1)(g_2))(g_3).
\end{align*}
In the human language we used above, $\mu(g)(h)=g\cdot h$ and $\iota(g)=g^{-1}$ and $\mathrm{Unitlaws}$ and $\mathrm{Assoclaw}$ spell out to the machine that the unit behaves like a unit and that the multiplication is associative.
The
\emph{type of abstract groups} is
$$%\typeabsgp
\typegroup^\abstr\defequi
\sum_{(M,e,\mu):\typemonoid}\sum_{\iota\colon M\to M}\prod_{g:M}(\mu{}(\iota{}(g))(g)=e).$$
% where
% $$\mathrm{Grouplaws}(G,e,\mu,\iota)\defequi\mathrm{Monoidlaws}(G,e,\mu)\times \mathrm{Invlaws}(G,\iota{},\mu{},e)$$
% and
% $$\mathrm{Invlaws}(G,e,\mu{},\iota{})\defequi
% \prod_{g:G}(\mu{}(\iota{}(g))(g)=e)\times
% (\mu{}(g)(\iota{}(g))=e)\times
% (\iota{}(\iota{}(g))=g).$$
We will typically refer to a monoid as a triple $(M,e,\mu)$, omitting the names for the (true) $\isset$ and unit and associativity laws, and likewise, an abstract group will be referred to as a quadruple $(M,e,\mu,\iota)$.  The \emph{underlying set} of a group is defined by setting 
$$\mathrm{under}(M,e,\mu,\iota)=M.$$
\end{remark}
\begin{remark}
  That the concept of an abstract group synthesizes the idea of symmetries will be manifested shortly when we prove that 
$$\abstr:\typegroup\to\typegroup^\abstr$$
is an equivalence.
\end{remark}

  \begin{remark}
Without the demand that the underlying type of an abstract group or monoid is a set, life would be more complicated.  For instance, for the case when $g$ is $e$, the unit law of \cref{def:abstractgroup} (or alternatively $\mathrm{Unitlaws}(G,\mu{},e)(e)$ in \cref{rem:typemonoidabstrgp}) would provide \emph{two} (potentially different) proofs that $e=e\cdot e$ and we would have to separately insist that they agree.  This problem vanishes in the setup we adopt below for \inftygps.
  \end{remark}

  \begin{xca}
    For an element $g$ in an abstract group $(G,e,\mu,\iota)$, prove that $e=g\cdot g^{-1}$ and $g=(g^{-1})^{-1}$ (for the machines among us: ``give an element in the proposition
$\prod_{g:\pt_G=\pt_G}
(e=\mu{}(g)(\iota{}(g)))\times
(g=\iota{}(\iota{}(g)))$'').
  \end{xca}
  \begin{xca}\label{xca:typemonoidisgroupoid}
    Prove that the types of monoids and abstract groups are groupoids.
  \end{xca}
  \begin{xca}
    \label{xca:cheapgroup}
    There is a leaner way of characterizing what an abstract group is: define a \emph{sheargroup} to be a set $S$ together with an element $e:S$, a function $S\times S\to S$ sending $(a,b):S\times S$ to $a*b:S$ and the following propositions where we use the shorthand $\bar a\defequi a*e$:
    \begin{enumerate}
    \item $e*a=a$,
    \item $a*a=e$ and
    \item $c*(b*a)=\overline{(c*\bar b)}*a$,
    \end{enumerate}
    for all $a,b,c:S$.
    Show that the type of abstract groups is equivalent to the type of sheargroups.  

Hint: setting $a\cdot b\defequi \bar b*a$ gives you an abstract group from a sheargroup and conversely, letting $a*b=b\cdot a^{-1}$ takes you back.  On your way you may need at some point to show that $\overline{\bar a}=a$: setting $c=\bar a$ and $b=a$ in the third formula will do the trick (after you have established that $\bar e=e$).  This exercise may be good to look back to in the many instances where the inverse inserted when ``multiplying from the right by $a$'' is forced by transport considerations. 
  \end{xca}




\section{Homomorphisms}
\label{sec:homomorphisms}


The notion of a group homomorphism from $G=\aut_A(a)$ to $H=\aut_B(b)$ is simple: it is an function $f:A\to B$ that ``sends $a$ to $b$'', \ie together with an element $p:a=_Bf(b)$:
\begin{definition}\label{def:grouphomomorphism}
  The type of \emph{group homomorphisms} from $G:\typegroup$ to $H:\typegroup$ is defined to be
$$\Hom(G,H)\defequi(BG\to_* BH).
\footnote{would you be unhappy if I used $f:G\to_{\typegroup}H$ when it fits better with the typography?}%\sum_{f:A\to B}f(a)=_Bb.
$$
\end{definition}
\begin{example}
  \begin{enumerate}
  \item   Consider two sets $S$ and $T$.  
Recall that $\Set_{(S)}\defequi\sum_{X:\Set}||S=X||$ is the component of the groupoid $\Set$ containing $S$, and when pointed at $S$ represents the permutation group $\Sigma_S$.  
The map $\Set_{(S)}\to\Set_{(S\coprod T)}$ sending $X$ to $X\coprod T$ induces a group homomorphism $\Sigma_S\to\Sigma_{S\coprod T}$.
Thought of as symmetries, this says that if you have a symmetry of $S$, then we get a symmetry on $S\coprod T$ (which doesn't do anything to $T$).  

Likewise, we get a map $\Set_{(S)}\to\Set_{(S\times T)}$ sending $X$ to $X\times T$ induces a group homomorphism $\Sigma_S\to\Sigma_{S\times T}$. 

In particular, we get homomorphisms $\Sigma_m\to\Sigma_{m+n}$ and $\Sigma_m\to\Sigma_{mn}$. \footnote{find a good description of the sign $\Sigma_n\to\Sigma_2$}
\item Let $G$ be a group.  Since there is a unique map from $BG$ to $\bn{1} $, we get a unique homomorphism from $G$ to the trivial group.  Likewise, there is a unique morphism from the trivial group to $G$, sending the unique element of $\bn 1$ to $\pt_G$. 
\item If $G$ and $H$ are groups, the inclusions and projections between $B(G\times H)\oldequiv BG\times BH$ and $BG$ and $BH$ give rise to group homomorphisms between $G\times H$ and $G$ and $H$.  \footnote{Elaborate}
  \end{enumerate}
\end{example}
\begin{xca}
  Let $G$ be a group.  Show that $\Hom(\ZZ,G)=(\pt_G=_{BG}\pt_G)$.  %Show that ((wedges of circle vs multiplication))
\end{xca}



\begin{xca}\label{xca:BGtotype}
  Let $G$ be a group and $A$ a connected groupoid.  Use the definitions and \cref{xca:freemaps} to show that the types
  \begin{enumerate}
  \item $BG_\div\to A$, 
  \item $\sum_{a:A}\sum_{f:BG_\div \to A}(f(\pt_G)=a)$, 
  \item $\sum_{a:A}(BG\to_*(A,a))$ and 
  \item $\sum_{a:A}\Hom(G,\aut_A(a))$
  \end{enumerate}
 are all equivalent.
\end{xca}

The definition of group homomorphisms in \cref{def:grouphomomorphism} should be contrasted with the usual -- and somewhat more cumbersome -- notion of a group homomorphism $f\colon G\to H$ of abstract groups where we must specify that in addition to preserving the neutral element ``$f(e_G)=e_H$'' it must preserve multiplication: ``$f(g)\cdot_H f(g')=f(g\cdot_G g')$ (where we have set the name of the group as a subscript to $e$ and $\cdot$).  In our setup this is simply true:

\begin{definition}\label{def:grouphomomaxioms}
Let $G$ and $H$ be groups and assume given a group homomorphism $f:G\to H$.  We now define something that we will call an ``abstract group homomorphism 
${\abstr}(f):\abstr(G)\to \abstr(H)$'', \ie a function of sets from $(\pt_G=\pt_G)$ to $(\pt_H=\pt_H)$ ``preserving'' the group structure, c.f~\cref{def:abstrG} for the definition of $\abstr(G)$ and \cref{def:abstrisfunctor} for a condensation of what the discussion below end up with concluding that ``preserves'' means.  

We take the time to develop this from first principles.
Recall that $f$ is nothing but a pointed function from $BG$ to $BH$; or in other words a map of (unpointed) types 
$$Bf_\div\colon BG_\div\to BH_\div$$ 
and an identity 
$$p_f: Bf_\div(\pt_G)=\pt_H.$$  
As in \cref{def:apd}%\footnote{I use $f(p)$ rather than the $\mathrm{ap}_f$-formalism which I think is alienating when compared with the classical setup}
, for $z:BG$ this gives rise to a map 
$$\ap{Bf_\div}:(\pt_G=z)\to (Bf_\div(\pt_G)=Bf_\div(z)),$$ 
defined by induction by declaring that $\ap{Bf_\div}(\refl{\pt_G})\defequi \refl{Bf_\div(\pt_G)}$.  
If $g:\pt_G=\pt_G$, then $\ap{Bf_\div}(g)$ is an element of $Bf_\div(\pt_G)=Bf_\div(\pt_G)$, while we want something in $\pt_H=\pt_H$.  However, this is not an obstacle since conjugation by $p_f: Bf_\div(\pt_G)=\pt_H$ gives rise to 
$$\mathrm{ad}_{p_f}:(Bf_\div(\pt_G)=Bf_\div(\pt_G))=(\pt_H=\pt_H)$$ (with $\mathrm{ad}_{p_f}(\refl{Bf_\div(\pt_G)})=\refl{\pt_H}$, as discussed in \cref{sec:heavy-transport}) and so 
$$\mathrm{ad}_{p_f}(\ap{Bf_\div}(g)):\pt_H=\pt_H.$$
This defines a function
$$f^\abstr\defequi \mathrm{ad}_{p_f}\ap{Bf_\div}:(\pt_G=\pt_G)\to(\pt_H=\pt_H)$$  
and we depict $f^\abstr(g)$ as the ``up, over and down'' identity of $\pt_H$:
$$\xymatrix{Bf_\div(\pt_G)\ar@{=}[r]^{\ap{Bf_\div}(g)}_\to\ar@{=}[d]^\downarrow_{p_f}&Bf_\div(\pt_G)\ar@{=}[d]^\downarrow_{p_f}\\
\pt_H\ar@{:}[r]^{f^\abstr(g)}_\to&\pt_H}.$$


% Since type-checking removes the ambiguity, we trust it will not lead to any confusion that we simplify the notation and use the symbol $f$ simultaneously for $Bf_\div$, writing $f\colon BG_\div\to BH_\div$ and for $\ap{Bf_\div}$, writing $f:(\pt_G=\pt_g)\to (f(\pt_G)=f(\pt_G))$, while we write
% $$f^\abstr\defequi \mathrm{ad}_p\ap{Bf_\div}:(\pt_G=\pt_G)\to(\pt_H=\pt_H), \qquad g\mapsto p\,f(g)\,p^{-1}$$  
% and depicting it as the ``up, over and down'' identity of $\pt_H$:
% $$\xymatrix{f(\pt_G)\ar@{=}[r]^{f(g)}_\to\ar@{=}[d]^\downarrow_p&f(\pt_G)\ar@{=}[d]^\downarrow_p\\
% \pt_H&\pt_H}.$$
% When time comes, even the superscript $\abstr$ will vanish.
% % (which is defensible, given that $p$ is part of the homomorphism $f$).

With the shorthand $$e_G\defequi\refl{\pt_G}:(\pt_G=\pt_G)\oldequiv\abstr(G)$$ and writing (to remind us where things happen)
$$g\cdot_Gg':(\pt_G=\pt_G)$$
 for the composite $g\,g'$ of $g$ and $g'$ (note that we use functional notation, so that composition is ``first $g'$ and then $g$'' as in the picture 
$\xymatrix{\pt_G\ar@{=}[r]^{g'}_\to&\pt_G\ar@{=}[r]^{g}_\to&\pt_G}$) %$\trans_{\pt_G,\pt_G\pt_G}(g,g'):$ 
and likewise with a subscript $H$, we have the following
  \begin{enumerate}
  \item $\refl{e_H}:f^\abstr(e_G)= e_H$ makes sense since
$$f^\abstr(e_G)\oldequiv\mathrm{ad}_{p_f}\ap{Bf_\div}(\refl{\pt_G})\oldequiv\mathrm{ap}_{p_f}\refl{f(\pt_G)}\oldequiv\refl{\pt_H}\oldequiv e_H
$$
% Since $\ap{Bf}(e_G)\oldequiv \ap{Bf}f(\refl{\pt_G})\oldequiv\refl{f(\pt_G)}$ and $e_H\oldequiv\refl{\pt_H}$, 
% $$f(e_G)\oldequiv\mathrm{ad}_p(\ap{Bf}(e_G))\oldequiv\mathrm{ad}_p(\refl{f(\pt_G)})\oldequiv \refl{\pt_H}\oldequiv e_H$$
      \item the proposition $f^\abstr(g\cdot_Gg')=f^\abstr(g)\cdot_Hf^\abstr(g')$ is inhabited by the composite
        \begin{align*}
          f^\abstr(g\cdot_Gg')&\defequi %\mathrm{ad}_{p_f}\ap{Bf}(g\cdot g)\\\oldequiv&
            \mathrm{ad}_{p_f}\ap{Bf}(g\, g')\\
          &=\mathrm{ad}_{p_f}(\ap{Bf}(g)\,\ap{Bf}(g)')\\
          %&= (\mathrm{ad}_{p_f}\ap{Bf}(g))\,{\mathrm{ad}_{p_f}\ap{Bf}(g)')\\
          &=\mathrm{ad}_{p_f}\ap{Bf}(g)\cdot_H\mathrm{ad}_{p_f}\ap{Bf}(g')\oldequiv f^\abstr(g)\cdot_Hf^\abstr(g'),
        \end{align*}
where we have used that both $\ap{Bf_\div}$ and $\mathrm{ad}_{p_f}$ take composites of identities to composites of identities.\footnote{give ref}

If you find it useful, you may consider the following picture:
$$\xymatrix{
Bf_\div(\pt_G)\ar@{=}[rr]^{\ap{Bf_\div}(g\, g')}_\to\ar@{=}[dr]^{\ap{Bf_\div}(g')}_\to\ar@{=}[dd]_{p_f}^\downarrow
&&Bf_\div(\pt_G)\ar@{=}[dd]_{p_f}^\downarrow\\
&Bf_\div(\pt_G)\ar@{=}[d]_{p_f}^\downarrow
\ar@{=}[ur]^{\ap{Bf_\div}(g)}_\to
&\\
\pt_H&\pt_H&\pt_H},$$
where $f^\abstr(g\cdot_Gg')$ is simply ``up, over and down'' while $f^\abstr(g)\cdot_Hf^\abstr(g')$ takes the detour via the $\pt_H$ in the middle.
  \end{enumerate}
\end{definition}
\begin{definition}\label{def:abstrisfunctor}
  If $\mathcal G\defequi(\mathcal G_{\Set},e_{\mathcal G},\mu_{\mathcal G},\iota_{\mathcal G})$ and $\mathcal H\defequi(\mathcal H_{\Set},e_{\mathcal H},\mu_{\mathcal H},\iota_{\mathcal H})$ are two abstract groups, then the set of homomorphisms from $\mathcal G$ to $\mathcal H$ is
 $$  \Hom^\abstr(\mathcal G,\mathcal H)%(G_{\Set},e_G,\mu_G,\iota_G),(H_{\Set},e_H,\mu_H,\iota_H))\\
=\sum_{f:\mathcal G_{\Set}\to \mathcal H_{\Set}}(f(e_{\mathcal G})=e_{\mathcal H})\times (f\mu_{\mathcal G}=\mu_{\mathcal H}f).
$$
Since $(f(e_{\mathcal G})=e_{\mathcal H})\times (f\mu_{\mathcal G}=\mu_{\mathcal H}f)$ is a proposition, a homomorphism of abstract group is uniquely determined by its underlying function of sets, and unless there is danger of confusion we may write $f$ instead of $(f,!)$. 

If $G$ and $H$ are groups, the function
$$\abstr:\Hom(G,H)\to\Hom^\abstr(\abstr(G),\abstr(H))$$
is the function $\abstr(f)=(f^\abstr,!)$ defined in \cref{def:grouphomomaxioms}.
\end{definition}
\begin{xca}
  Note that the inverses play no r\^ole in the definition of a homomorphism of abstract groups.  Prove that if $(f,!):\Hom^\abstr(\mathcal G,\mathcal H)$
%(G_{\Set},e_G,\mu_G,\iota_G),(H_{\Set},e_H,\mu_H,\iota_H))$ 
then $f(g^{-1})=(f(g))^{-1}$, making a separate axiom redundant.  
\end{xca}
\begin{xca}
Prove that composition of the functions on the underlying sets gives a composition of homomorphisms of abstract groups.

  Prove that if $f_0:\Hom(G_0,G_1)$ and $f_1:\Hom(G_1,G_2)$ then 
$$\abstr(f_1f_0)=\abstr(f_1)\abstr(f_0)$$ and that $\abstr(\id_G)=\id_{\abstr(G)}$.
\end{xca}



\section{\texorpdfstring{\inftygps}{∞-groups}}
\label{sec:inftygps}

Disregarding the set-condition we get the simpler notion of \inftygps:
\begin{definition}The type of $\infty$-groups is
  $$\typeinftygp\defequi \sum_{A:\UU}\sum_{a:A}\prod_{x:A}||x=_Aa||.$$
\end{definition}

\begin{remark}\label{rem:pointedtypes}
  Just as ``group'' is a synonym for ``pointed, connected groupoid'', ``$\infty$-group'' is synonym for ``pointed, connected type''.  As for groups, we suppress the propositional information from the notation and write $\aut_A(a)$ isntead of $(A,a,!)$ for an $\infty$-group.
% ;  the type of \emph{pointed types} being
% $$\pttype\defequi\sum_{A:\UU}A,$$
% and given two pointed types $(A,a)$ and $(B,b)$, the type of \emph{pointed maps} from $(A,a)$ to $(B,b)$ is
% $$((A,a)\to_*(B,b))\defequi\sum_{f\colon A\to B}f(a)=b.$$
\end{remark}


\footnote{Let $\typeset\defequi \sum_{A:\UU}\isset(A)$.}
\begin{definition}\label{def:classifyingspace}
  If $G\oldequiv\Aut_A(a):\typeinftygp$, then the underlying pointed type $BG\defequi (A,a)\colon\pttype$ is called the  \emph{classifying space} and $\pt_G\defequi a$ is the \emph{base point}.  
%We retain the same language also for ordinary groups in which case the classifying space is a groupoid (\ie a $1$-type).   %For \inftygps the definition is identical.
\end{definition}
\begin{definition}
  A homomorphism of $\infty$ groups is a pointed function of classifying spaces, \ie
  given two $\infty$-groups $G$ and $H$ we define
$$\Hom(G,H)\defequi(BG\to_*BH).$$
% If $G$ is \aninftygp, we let $\pt_G:BG$ (and sometimes simply $\pt$ if $G$ is clear from the context) be the distinguished point (so that $G\oldequiv\aut_{BG}(\pt_G)$).
\end{definition}





\section{$G$-sets}
\label{sec:gsets}

One of the goals of the next section is to prove that, in a precise sense, any abstract group corresponds to a group.  In doing that, we are invited to explore how abstract groups should be thought of as symmetries and introduce the notion of a $G$-set.  However, this takes a pleasant detour where we have to explore the most important feature of groups: they \emph{act} on things (giving rise to symmetries)!

Before we handle the more complex case of abstract groups, let us see what this looks like for groups.

\begin{definition}
  For $G$ a group (or \inftygp), a \emph{$G$-type} is a function
  $$X\colon BG\to\UU,$$
%($\BG_\div$ was defined to be the underlying type of $BG$)
and $X(\pt_G)$ is referred to as the \emph{underlying type}.
If the underlying type is a set, then $X$ is called a \emph{$G$-set}.

Otherwise said, the type of $G$-types is $\Type_G\defequi(BG\to\UU)$ and the type of $G$-sets is $\Set_G\defequi(BG\to\Set)$.
%$$\Type_G\defequi (BG\to\UU),\qquad \Set_G\defequi (BG\to\Set).$$
\end{definition}

\begin{remark}
  The reader will notice that the type of $G$-set is equivalent to the type of \coverings over $BG$.%\footnote{we should be careful with having too many official names for the same objects}
The reason we have allowed ourselves two names is that our focus is different: for a $G$-set $X:BG\to\Set$ we focus on the sets $X(z)$, whereas when talking about \coverings the first projection $\sum_{z:BG}X(z)\to BG$ takes center stage.  Each focus has its advantages.

Given a $G$-set, we may consider it as a $G$-type and will usually not make a notational distinction.
\end{remark}


\begin{example}\label{def:principaltorsor}
  If $G$ is a group (or \inftygp), then
$$\princ G:BG\to\UU,\qquad\princ G(z)\defequi\pathsp{\pt_G}(z)\defequi(\pt_G=z)$$ is a $G$-set (or $G$-type) called the \emph{principal $G$-torsor}.  
We've seen this family before in the guise of the (fibers of the) ``universal \covering'' of \cref{def:universalcover}!  
The term ``$G$-torsor'' will reappear several times and will mean nothing but a $G$-type in the component of $\princ G$ -- a ``twisted'' version of $\princ G$.

There is nothing sacred about starting the equality in $\pt_G$: if $y:BG$, then $\pathsp y:BG\to\UU$ is also a $G$-set (type) and if $q:y=y'$, then the preferred equality between $\pathsp y$ and $\pathsp{y'}$ sends $p:y=z$ to $pq^{-1}:y'=z$.  As a matter of fact, \cref{lem:BGbytorsor} will identify $BG$ and the type of $G$-torsors via the map sending $y:BG$ to $\pathsp y$ using the full transport structure of the identity type $\pathsp yz\defequi(y=z)$.
 
%The name torsor will be explained shortly.
\end{example}

\begin{example}\label{def:adjointrep}
  If $G$ is a group (or \inftygp), then
$$\Ad_G:BG\to\UU,\qquad\Ad_G(z)\defequi(z=z)$$ is a $G$-set (or $G$-type) called the \emph{adjoint $G$-set (or $G$-type)}.  The name ``adjoint'' comes from how transport works in this case; if $p:y=z$, then $\Ad_G(p):(y=y)=(z=z)$ is given by conjugation: 
$$\Ad_G(p)(q)=pqp^{-1}:z=z,$$ the picture
$$\xymatrix{y\ar@{=}[r]^p_\to\ar@{=}[d]_q^\downarrow&z\ar@{=}[d]^{\Ad_G(p)(q)}_\downarrow\\
y\ar@{=}[r]^p_\to&z}$$
is a mnemonic device illustrating that it couldn't have been different, and should be contrasted with the picture for $\princ G (p):(\pt_G=y)=(\pt_G=z)$:
$$\xymatrix{\pt_G\ar@{=}[r]^{\refl{\pt_G}}_\to\ar@{=}[d]_q^\downarrow&\pt_G\ar@{=}[d]^{\princ G(p)(q)}_\downarrow\\
y\ar@{=}[r]^p_\to&\,z.}$$  
Notice that by the induction principle for the circle,
$$\sum_{z:BG}\Ad_G(z)=\sum_{z:BG}(z=z)$$
is equivalent to the type of (unpointed!) maps $S^1\to BG$, known in other contexts as the \emph{free loop space} of $BG$, an apt name given that it is the type of ``all symmetries of $BG$.''  
The first projection $\sum_{z:BG}\Ad_G(z)\to BG$ correspond to the function $(S^1\to BG)\to BG$ given by evaluating at $\base$. 
\end{example}
\begin{example}\label{def:trivGset}
  If $G$ is a group (or \inftygp), and $X$ is a set (or type) then
$$\mathrm{triv}_GX(z)\defequi X$$ is a $G$-set (or $G$-type).  Examples of this sort (regardless of $X$) are called \emph{trivial $G$-sets (or $G$-types)}.
\end{example}

\begin{remark}
  \label{remark:GsetsareGsets}
  A $G$-type $X$ is often presented by focusing on the \emph{underlying type} $X(\pt_G)$  and providing it with a structure relating it to $G$ determining the entire function $X\colon BG\to\UU$.

More precisely, since $BG$ is connected, a $G$-type $X\colon BG\to\UU$ factors over the component $\UU_{(X(\pt_G))}\defequi\sum_{Y:\UU}||Y=X(\pt_G)||$ which contains the point $X(\pt_G)$.  Since $B\Sigma_{X(\pt_G)}\defequi(\UU_{(X(\pt_G))},X(\pt_G))$ the $G$-type $X$ can, without loss of information, be considered as a homomorphism 
$$G\to\Sigma_{X(\pt_G)}% \defequi(\UU_{(X(\pt_G))},X(\pt_G))
$$ from $G$ to the permutation group $\Sigma_{X(\pt_G)}$ of $X(\pt_G)$.

Conversely, if $X$ is any type \emph{and} we have a homomorphism $G\to\Sigma_X$ (in other words, a pointed map $BG\to B\Sigma_{X}%\defequi(\UU_{(X)},X(\pt_G)
$), then the composite
$$BG\to \UU_{(X)}\to \UU$$
is a $G$-type with $X$ exactly the value of $\pt_G$.

However, we must be careful not to focus too much on the underlying type.  
For instance, even though the underlying type of both $\Ad_G$ and $\princ G$ is $\pt_G=\pt_G$, in general  $\Ad_G$ and $\princ G$  are very different $G$-types.  
To drive this point home, compare the illustrations of transport along a $p:\pt_G=\pt_G$ for the two:
$$\xymatrix{\pt_G\ar@{=}[r]^p_\to\ar@{=}[d]_q^\downarrow&\pt_G\ar@{=}[d]^{\Ad_G(p)(q)}_\downarrow\\
\pt_G\ar@{=}[r]^p_\to&\pt_G},\qquad
\xymatrix{\pt_G\ar@{=}[r]^{\refl{\pt_G}}_\to\ar@{=}[d]_q^\downarrow&\pt_G\ar@{=}[d]^{\princ G(p)(q)}_\downarrow\\
\pt_G\ar@{=}[r]^p_\to&\pt_G}$$
A third $G$-type with underlying type $\pt_G=\pt_G$ is $\mathrm{triv}_G(\pt_G=\pt_G)$.
\end{remark}

\begin{xca}
  Prove that if $G$ is an abelian group, then $\Ad_G=\mathrm{triv}_G(\pt_G=\pt_G)$.
\end{xca}
\begin{xca}
  Use that $BG$ is connected to show that if $X$ is a $G$-set, then $X(z)$ is a set for all $z:BG$ (\ie $\prod_{z:BG}\isset(X(z))$).
\end{xca}
\subsection{The classifying space is the type of torsors}
\label{sec:torsors}
\begin{definition}
  Given a group (or \inftygp) $G$, the type of {\em$G$-torsors} is
$$\typetorsor_G\defequi\sum_{X:\Type_G}||\princ G=X||,$$
where $\princ G$ is the principal $G$-torsor of \cref{def:principaltorsor}.
\end{definition}
\begin{remark}
  Note that if $G$ is a group (as opposed to \aninftygp), then $\princ G$ is a $G${\em-set}, and so for $G$-types $X$, the proposition $||\princ G=X||$ will be empty unless $X$ is a $G$-set too, and so in this case we could more simply have said $\typetorsor_G\defequi\sum_{X:\Set_G}||\princ G=X||.$  
Hence, for $G$ a group, the type of $G$-torsors is just another name for the component of the type of \coverings of $BG$ containing the universal \covering.

Observe that for a group $G$, $\typetorsor_G$ is a connected groupoid (admittedly in a higher universe) and so -- by specifying the base point $\princ G$ -- it represents a group!  Guess which one!
\end{remark}
\begin{definition}
  \label{def:BG2TorsG}
  Let $\pathsp{}\defequi\pathsp{}^G:BG\to(\typetorsor_G,\princ G)$ be the pointed map given by sending $z:BG$ to $\pathsp z$ and $\refl{\pathsp{\pt_G}}:\pathsp{\pt_G}=\princ G$. 
\end{definition}

\begin{example}\label{ex:pathsptransport}
  For $y,z:BG$ we make the induced map
$$\pathsp{}:(y=z)\to (\pathsp y=\pathsp z)
$$
explicit.  For $q:y=z$,  the transport $\pathsp q:\pathsp y=\pathsp z\defequi\prod_{x:BG}\pathsp y(x)=\pathsp z(x)$ is given by sending $p:\pathsp y(x)\defequi (y=x)$ to
$$\pathsp q(p)\defequi pq^{-1}:\pathsp z(x)\defequi(z=x),$$ 
or, in a picture 
$$\xymatrix{y\ar@{=}[r]^q_\to\ar@{=}[d]^{p}_\downarrow&z\ar@{=}[d]^{\pathsp q(p)}_\downarrow\\
x\ar@{=}[r]^{\refl x}_\to&\,x.}$$
\end{example}
\begin{lemma}\label{lem:pathsptransportiseq}
  For  $y,z:BG$ the induced map  (\ie transport) of identity types
$$\pathsp{}:(y=z)\to (\pathsp y=\pathsp z)
$$
is an equivalence.
\end{lemma}
\begin{proof}
 By the induction principle for the identity type,  
a function $f:\pathsp y\to \pathsp z$ is given by its value $f(\refl y):\pathsp zy\defequi (z=y)$ and in general, for $p:y= x$ we have by transport that $f(p)$ is the composite $p\,f(\refl y)=\pathsp{(f(\refl y))^{-1}}(p)$.  
In other words the map $\pathsp{}:(y=z)\to(\pathsp z=\pathsp y)$ is an equivalence.
\end{proof}


\begin{theorem}\label{lem:BGbytorsor}
  If $G$ is a group (or \inftygp), then the function
$$\pathsp{}:BG\to(\typetorsor_G\princ G),\qquad z\mapsto \pathsp z\defequi(x\mapsto(z=_{BG}x))$$
is an equivalence.
Univalence then provides us with an identity 
$$\bar{\pathsp{}}:G=(\typetorsor_G,\princ G)$$ of groups (or $\infty$-groups).\footnote{in the appropriate universe}.
\end{theorem}

\begin{proof}
  Since both $\typetorsor_G$ and $BG$ are connected, it suffices by \cref{lem:eqandcovofconntypes} to show that the induced map
$$\pathsp{}:(y=z)\to(\pathsp y= \pathsp z)
$$
is an equivalence, which is exactly the contents of \cref{lem:pathsptransportiseq}.  
%But this will follow by the very definition of the identity type!  

% For $q:y=z$,  the transport $T(q):\pathsp y=\pathsp z$ is given by  sending $p:y=x$ to
% $$T(q)(p)\defequi pq^{-1}:T(z)(x)\defequi(z=x),$$ 
% or, in a picture 
% $$\xymatrix{y\ar@{=}[r]^q_\to\ar@{=}[d]^{p}_\downarrow&z\ar@{=}[d]^{T(q)(p)}_\downarrow\\
% x\ar@{=}[r]^{\refl x}_\to&\,x.}$$

% %Hence $\bn 1\to T^{-1}(f)=\sum_{p:y=z}T(p)=f$ with value is an equivalence.
% %the same as giving an element in $T(z)(y)\defequi (z=y)$.
\end{proof}
\subsection{Any group is a subgroup of a permutation group}
\label{sec:groupssubperm}


This allows for a cute proof of what is often stated as ``any group is a permutation group'', which in our parlance translates to ``any symmetry is a symmetry of $\UU$'':\footnote{which reminds me of the following: my lecturer in cosmology once tried to publish a paper about rotating black holes, only to have it rejected because it turned out that it was his universe, not the black hole, that was rotating}
\begin{lemma}
  \label{lem:allgpsarepermutationgps}Let $G$ be a group.  Then remembering the base point in the factorization of the pricipal $G$-torsor $\princ G:BG\to\UU$ through the component $\UU_{(\pt_G=\pt_G)}\subseteq\UU$ (c.f.~\cref{remark:GsetsareGsets}) gives an injective homomorphism
$$\alpha_G:\Hom(G,\Sigma_{\pt_G=\pt_G})$$  
of $G$ into a permutation group.
\end{lemma}
\begin{proof}
  The type of injective group homomorphism from $G$ to a group $H$ is equivalent\footnote{((GIVE REF))} to the type of pointed \coverings of $BH$ by $BG$, so we need to show that $\alpha_G:BG\to\UU_{(\pt_G=\pt_G)}$ is a \covering.  
Under the identity $\bar{\pathsp{}}:BG=(\typetorsor_G,\princ G)$ of \cref{lem:BGbytorsor} the function $\alpha$ translates to the evaluation map
$$\xymatrix{
  \typetorsor_G\ar[rr]^-{\mathrm{ev}_{\pt_G}}\ar@{=}[d]&&\Sigma_{\pt_G=\pt_G}\ar@{=}[d]\\
  \underset{E:BG\to\Set}\sum\,\underset{x:BG}\prod||(\pt_G=x)=E(x)||\ar@{}[rr]^-{E\mapsto E(\pt_G)}&&
\,\underset{X:\Set}\sum||(\pt_G=\pt_G)=X||.
}$$
We must show that the preimages $\ev_{\pt_G}(X)$ for $X:\Sigma_{\pt_G=\pt_G}$ are sets.  
This preimage is equivalent to $\sum_{E:BG\to\Set}(X=E(\pt_G))$ (note the absence of a truncation).  We must show that if $(E,p),(F,q):\sum_{E:BG\to\Set}(X=E(\pt_G))$, then 
$$((E,p)=(F,q))=\prod_{x:BG}\sum_{\phi(x):E(x)=F(x)}\phi(\pt_G)=_{E(\pt_G)=F(\pt_G)}qp^{-1}$$ is a set.  
Note that if $r:\pt_G=x$, then $\phi(x)=F(r)qp^{-1}E(r)^{-1}$, or in a picture
$$\xymatrix{&E(pt_G)\ar@{=}[r]^{E(r)}_\to\ar@{=}[dd]_{\phi(pt_G)}^\downarrow&E(x)\ar@{=}[dd]_{\phi(x)}^\downarrow\\
  X\ar@{=}[ur]^p_\to\ar@{=}[dr]^q_\to&&\\
  &F(\pt_G)\ar@{=}[r]^{F(r)}_\to&F(x)},$$
and so $\phi(x)$ (which by nature is independent of such an $r:\pt_G=x$) is uniquely determined by $(E,p)$ and $(F,q)$.  Let us pin this down in our language.
If $\phi,\psi:((E,p)=(F,q))$, we must show that $\phi=\psi$ and since that for $x:BG$ both $E(x)$ and $F(x)$ are sets, it is enough to show that the proposition $\phi(x)=\psi(x)$ is not empty.  Let $f:(\pt_G=x)\to(\phi(x)=\psi(x))$ be given by letting $f(r)$ be the composite of the identities $\phi(x)=F(r)qp^{-1}E(r)^{-1}=\psi(x)$ given above.  
Since $BG$ is connected, $\pt_G=x$ is not empty, and we are done.
\end{proof}

\subsection{Homomorphisms and torsors}
\label{sec:homotor}
In view of the equivalence $\pathsp{}^G$ between $BG$ and $(\typetorsor_G,\princ G)$ of \cref{lem:BGbytorsor} one might ask what a group homomorphism  $f:\Hom(G,H)$ translates to on the level of torsors.  Off-hand, the answer is $(\pathsp{}^H)Bf(\pathsp{}^G)^{-1}$, but we can be more concrete than that.  We do know that for $x:BG$ the $G$-torsor $\pathsp x^G$ should be sent to $\pathsp {Bf(x)}^H$, but how do we express this for an arbitrary $G$-torsor?
\begin{definition}
  \label{def:restrictandinduce}
  Let $f:\Hom(G,H)$ be a group homomorphism.  If $Y:BH\to\Set$ is an $H$-set then the \emph{restriction} $f^*Y$ of $Y$ to $G$ is the $G$-set given by precomposition 
$$f^*Y\defequi Y\, f:BG\to\Set.$$  

If $X:BG\to\Set$ is a $G$-set and $y:BH$ define 
$$f_*X(y)\defequi(\pt_H=y)\times_{\pt_G=\pt_G}X(\pt_G)$$ to be the set quotient of $(\pt_H=y)\times X(\pt_G)$ by the relation $(p,x)\sim(p\, f(q)^{-1},X(q)x)$ for all $q:\pt_G=pt_G$.  The \emph{induced} $H$-set 
$$f_*X:BH\to\Set$$ has value at $y:BH$ the set $f_*X(y)$.
\end{definition}
When $X=\princ G$ we can get a good picture of $f_*X$:  composition gives a map
$$\eta\defequi:(\pt_H=y)\times (\pt_G=\pt_G)\to (\pt_H=y),\quad \eta(p,q)=p\,f(q)$$
and the fact that $\eta(p,q)=\eta(p\,f(r)^{-1},r\,q)$ for all $r:\pt_G=\pt_G$ tells us that $\eta$ defines a map $\eta_f(y):f_*\princ G(y)\to\princ H(y)$.  
The map $\iota:\princ H(y)\to f_*\princ G(y)$ defined by $\iota(p)=[p,e_G]$ is an inverse: $\iota\eta_f(y)$ sends $[p,q]$ to $[pf(q)^{-1},e_G]$, which is equal to $[p,q]$; and $\eta_f(y)\iota$ sends $p$ to $p\,f(e_G)^{-1}$, which is equal to $p$.  Hence $\eta_f(y)$ is an equivalence.
\begin{lemma}
  \label{lem:inducedtorsor}
   Let $f:\Hom(G,H)$ be a group homomorphism.  
If $X$ is a $G$-torsor, then $f_*X$ is an $H$-torsor and the identity $\bar{\eta}_f:f_*\pathsp x^G=\pathsp{Bf(x)}$ associated to the equivalence $\eta_f$ shows that  
$$\xymatrix{BG\ar[r]^{Bf}\ar[d]^{\pathsp{}^G}&BH\ar[d]^{\pathsp{}^H}\\
\typetorsor_G\ar[r]^{f_*}&\typetorsor_H}$$ 
commutes.
\end{lemma}
\begin{proof}
  If $||X=\princ G||$, then $||f_*X=f_*\princ G||$ and $\eta_f:f_*\princ G=\princ H$, so $f_*$ takes $G$-torsors to $H$-torsors.
\end{proof}
\begin{remark}
  \label{rem:inducedGsetfromabstracthomomorphisms}
  Notice that our construction of the induced $G$-set works equally well for a homomorphism $\phi:\Hom^\abstr(\abstr(G),\abstr(H))$: if $X:BG\to\Set$ is a $G$-set, then we define the $H$-set $\phi_*X:BH\to\Set$ by 
$$\phi_*X(y)\defequi(\pt_H=y)\times_{\pt_G=\pt_G}X(\pt_G)$$ to be the set quotient of $(\pt_H=y)\times X(\pt_G)$ by the relation $(p,x)\sim(p\, \phi(q)^{-1},X(q)x)$ for all $q:\pt_G=pt_G$. Just as above, for $X$ the principal $G$-torsor we get an identity  $eta_\phi:\phi_*\princ G=\princ H$ which, when evaluated at $y:BH$, corresponds under univalence to the equivalence 
$$(\pt_H=y)\times_{\pt_G=\pt_G}(\pt_G=\pt_G)\to (\pt_H=y)$$ 
sending $[p,q]:(\pt_H=y)\times_{\pt_G=\pt_G}(\pt_G=\pt_G)$ to $p\,\phi(q):(\pt_H=y)$.
\end{remark}


\section{Groups concrete or abstract% -- same gem, different wrapping
}
\label{sec:Gsetforabstract}

We use \cref{lem:BGbytorsor} as our inspiration for trying to construct a group from an abstract group.  We define totally analogously the type of torsors for an abstract group.  It will then be a relative simple matter to show that the processes of
\begin{enumerate}
\item forming the abstract group of a group and 
\item taking the group represented by the torsors of an abstract group
\end{enumerate}
 are inverse to each others.

Note that we have not considered an abstract analog of the concept of $\infty$-group, so all we do in this section is set-based.

\begin{definition}
\label{def:abstrGtorsors}
  If ${\agp G}=(S,e,\mu,\iota)$ is an abstract group, a \emph{$\agp G$-set}\index{Gset@$\agp G$-set (of abstract group} is a set $\mathcal X$ together with a homomorphism
$\agp G\to\abstr(\Sigma_{\mathcal X})$
from $\agp G$ to the (abstract) permutation group of $\mathcal X$:
$$Set_{\agp G}^\abstr\defequi \sum_{\mathcal X:\Set}\Hom_\abstr({\agp G},\abstr(\Sigma_{\mathcal X})).$$

The \emph{principal ${\agp G}$-torsor} $\princ {\agp G}^\abstr$ is the ${\agp G}$-set consisting of the underlying set $\mathrm{under}({\agp G})\defequi S$ together with the homomorphism ${\agp G}\to\abstr(\Sigma_{S})$ with underlying function of sets $S\mapsto (S=S)$ given by sending $g:S$ to $\mathrm{ua}(s\mapsto s\cdot g^{-1})$.

The type of \emph{${\agp G}$-torsors} is
$$\typetorsor_{\agp G}^\abstr\defequi\sum_{S:\Set_{\agp G}^\abstr}||\princ {\agp G}=S||.$$
\end{definition}
\begin{example}
  If $G$ is a group we can unravel the definition and see that an $\abstr(G)$-set consists of
  \begin{enumerate}
  \item a set $S$, 
  \item a function $f:(\pt_G=\pt_G)\to (S=S)$ 
  \item such that $f(e_G)=\refl{S}$ and for all $p,q:\pt_G=\pt_G$ we have that $f(p\, q)=f(p)\,f(q)$.
  \end{enumerate}

\end{example}


To help reading the coming proofs we introduce some notation that is redundant, but may aid the memory in cluttered situations:  Let $x,y,z$ be elements in some type, then
\begin{align*}
%  \pre:(x=y)\to ((y=z)=(x=z)),\qquad&\pre(q)(p)\defequi pq\\
  \preinv:(y=x)\to ((y=z)=(x=z)),\quad&\preinv(q)(p)\defequi\pathsp qp\defequi pq^{-1}\\
  \post:(y=z)\to ((x=y)=(x=z)),\quad&\post(p)(q)\defequi\post_pq\defequi pq\\
  %\adjoint:(x=y)\to((x=x)=(y=y)),\qquad&\adjoint(q)(p)\defequi\adjoint_qp\defequi qpq^{-1}
\end{align*}
We recognize $\preinv$ from \cref{lem:pathsptransportiseq} as the induced map of identity types $\pathsp{}\colon (y=z)\to(\pathsp y=\pathsp z)$ evaluated at $x$, while post-composition $\post$ is transport in the family $\pathsp x$, 


\begin{example}\label{ex:qG}
  If $G$ is a group, then for any $x:BG$ the principal $G$-torsor \emph{evaluated at $x$}, \ie the set $\princ Gx\defequi(\pt_G=x)$, has a natural structure of an $\abstr(G)$-set by means of 
$$\preinv:(\pt_G=\pt_G)\to ((\pt_G=x)=(\pt_G=x))$$ and the fact that $\preinv(e_G)\defequi\refl{\pt_G=x}$ and that for $p,q:\pt_G=\pt_G$ we have that  $\preinv(p\,q)=\preinv(p)\preinv(q)$ (\ie if $r\colon \pt_G=x$ we have that 
$$\preinv(p\, q)(r)=r(p\,q)^{-1}=r\,q^{-1}p^{-1}=\preinv(p)\preinv(q)(r)$$  -- demonstrating why we chose $\preinv$: without the inverse this would have gone badly wrong).  

That this $\abstr(G)$-set is an $\abstr(G)$-torsor then follows since $BG$ is connected (any $\pt_G=x$ will serve as a proof of $(\pt_G=x,\preinv,!)=\princ{\abstr(G)}^\abstr$).

Though it sounded like we made a choice ending up with $\preinv$; we really didn't -- it is precisely what happens when you abstract the homomorphism $G\to\Sigma_{\princ G(x)}$: 
you get the function of identity types 
$$(\pt_G=\pt_G)\to (\princ G(x)=\princ G(x))$$ 
which by the very definition of transport for $\princ G$ is $\preinv$. 
\end{example}

\begin{definition}
  If ${\agp G}$ is an abstract group, then the \emph{concrete group $\concr({\agp G})$ associated with ${\agp G}$} is the group (given by the pointed connected groupoid) $(\typetorsor_{\agp G}^\abstr,\princ {\agp G})$.
\end{definition}
We give the construction of \cref{ex:qG} a short name since it will occur in important places.
\begin{definition}
  Let $G$ be a group.  Define the group homomorphism 
 $$q_G:G\to \concr(\abstr(G))$$ defined in terms of the pointed map by the same name
$$q_G:BG\to_* (\typetorsor^\abstr_{\abstr(G)},\princ {\abstr(G)}),\quad q_G(z)=(\princ G(z),\preinv,!).$$
\end{definition}

\begin{lemma}
  \label{lem:Groupsareidentitytypes}%Let ${\agp G}$ be an abstract group.  
For all groups $G$, the pointed function $q_G:G\to\concr(\abstr(G))$ 
is a  equivalence.
\end{lemma}
\begin{proof}
  To prove that $q_G$ is an equivalence it is, by \cref{lem:eqandcovofconntypes}, enough to show that if $x,y:BG$ then the induced map
$$q_G:(x=_{BG}y)\to (q_G(x)=q_G(y))%(\pt_G=x)=_\UU(\pt_G=y))
$$
is an equivalence.
  Now, $q_G(x)=q_G(y)$ is equivalent to the set 
\begin{align*}
  &((\pt_G=x),\preinv)=_{\abstr(G)\text{-set}}((\pt_G=y),\preinv)\\
=&\sum_{f:(\pt_G=x)=(\pt_G=y)}f\preinv=\preinv f
\end{align*}
 ($f\preinv=\preinv f$ is shorthand for $\prod_{q:\pt_G=x}\prod_{p:\pt_G=p}f(pq^{-1})=f(p)q^{-1}$ and the rest of the data is redundant at the level of symmetries) and under these identities $q_G$ is given by 
$$(\post,!):(x=y)\to \sum_{f:(\pt_G=x)=(\pt_G=y)}f\preinv=\preinv f.$$
Given an element
$(f,!):\sum_{f:(\pt_G=x)=(\pt_G=y)}f\preinv=\preinv f$, the preimage 
$(\post,!)^{-1}(f,!)$ is equivalent to the set
$\sum_{r:x=y}(f=\post_r)$.  But if $(r,!),(s,!): \sum_{r:x=y}(f=\post_r)$, then for all $p:\pt_G=x$ we get that $r\,p=f(p)=s\,p$, that is $r=s$, so that the preimage is in fact a proposition.  
To show that the preimage is contractible, it is enough to construct a function $(\pt_G=x)\to \sum_{r:x=y}(f=\post_r)$, and sending $p$ to $f(p)p^{-1}$ will do.
\end{proof}

% \begin{definition}
%   A $G$-torsor is a $G$-set which is isomorphic to the underlying $G$-set of $G$ (write out - avoid conflict of notation wrt $|G|$)
% \end{definition}

% $$, $\pre(q)(p)=pq$
% $\preinv:(y=x)\to ((y=z)=((x=z))$, $\preinv(q)(p)=pq^{-1}$
% $\post:(y=z)\to ((x=y)=((x=z))$, $$
%\footnote{how deeply do we want to integrate univalence?}
\begin{example}
  \label{ex:abstrconcrG}
  Let ${\agp G}=(S,e,\mu,\iota)$ be an abstract group.  
Then the underlying set of $\abstr(\concr({\agp G}))$ is $\princ {\agp G}^\abstr=_{\typetorsor^\abstr_{\agp G}}\princ {\agp G}^\abstr$.  
Unraveling the definitions we see that this set is equivalent to
$$\sum_{p:S=S}\prod_{q,s:S}(p(s\,q^{-1})=p(s)\,q^{-1}).
$$  
Setting $s\defequi e$ and renaming $t\defequi q^{-1}$ in the last equation, we see that $p(t)=p(e)t$; that is $p$ is simply multiplication with an element $p(e):S$.  in other words, the function 
$$r_{\agp G}:S\to  \sum_{p:S=S}\prod_{q,s:S}(p(s\,q^{-1})=p(s)\,q^{-1}),\qquad r_{\agp G}(u)\defequi(u\cdot\,,!)
$$  
is an equivalence of sets, which we by univalence is converted into an identity.  
The abstract group structure of $\abstr(\concr({\agp G}))$ is given by it being the symmetries of $\princ {\agp G}^\abstr$; translated to $\sum_{p:S=S}\prod_{q,s:S}(p(s\,q^{-1})=p(s)\,q^{-1})$ this corresponds via the first projection to the symmetries of $S$. %that of $S=S$ of the first projection.  
This means that we need to know that if $u,v:S$ and consider the two symmetries $u\cdot,v\cdot:S=S$, then their composite (the operation on the symmetry on $S$) is equal to $(u\cdot v)\cdot:S=S$ (the abstract group operation), but this is true by associativity ($u\cdot(v\cdot s)=(u\cdot v)\cdot s$).  That $r_{\agp G}$ also sends $e:S$ to $\refl S$ is clear.
Hence our identity $r_{\agp G}$ underlies an identity of abstract groups
$$r_{\agp G}:{\agp G}=_{\typegroup^\abstr}\abstr(\concr({\agp G})).$$
\end{example}

This shows that every abstract group encodes the symmetries of something essentially unique.  Summing up the information we get
\begin{theorem}
  \label{lem:Groupsareidentitytypes}Let ${\agp G}$ be an abstract group.  
Then
$$\abstr:\typegroup\to\typegroup^\abstr$$ is an equivalence% : \ie the type of groups and the type of abstract groups are equal
.
\end{theorem}
%\begin{proof}
 %  First consider $q_G$.  To prove that $q_G$ is an equivalence it is, by \cref{lem:eqandcovofconntypes}, enough to show that if $x,y:BG$ then the induced map
% $$q_G:(x=_{BG}y)\to (q_G(x)=q_G(y))%(\pt_G=x)=_\UU(\pt_G=y))
% $$
% is an equivalence.\footnote{something to be said for the homotopies vs. base point}  Now, $q_G(x)=q_G(y)$ is equivalent to the set 
% \begin{align*}
%   &((\pt_G=x),\preinv)=_{\abstr(G)-set}((\pt_G=y),\preinv)\\
% =&\sum_{f:(\pt_G=x)=(\pt_G=y)}f\preinv=\preinv f
% \end{align*}
%  ($f\preinv=\preinv f$ is shorthand for $\prod_{q:\pt_G=x}\prod_{p:\pt_G=p}f(pq^{-1})=f(p)q^{-1}$ and the rest of the data is redundant at the level of symmetries) and under these identities $q_G$ is given by 
% $$(\post,!):(x=y)\to \sum_{f:(\pt_G=x)=(\pt_G=y)}f\preinv=\preinv f.$$
% Given an element
% $(f,!):\sum_{f:(\pt_G=x)=(\pt_G=y)}f\preinv=\preinv f$, the preimage 
% $(\post,!)^{-1}(f,!)$ is equivalent to the set
% $\sum_{r:x=y}(f=\post_r)$.  But if $(r,!),(s,!): \sum_{r:x=y}(f=\post_r)$, then for all $p:\pt_G=x$ we get that $r\,p=f(p)=s\,p$, that is $r=s$, so that the preimage is in fact a proposition.  To show that the preimage is contractible, it is enough to construct a function $(\pt_G=x)\to \sum_{r:x=y}(f=\post_r)$, and sending $p$ to $f(p)p^{-1}$ will do.
%\end{proof}

\section{Homomorphisms, abstract vs.~concrete}
\label{sec:homabsisconcr}

Now that we know that the type of groups is equal to the type of abstract groups, it is natural to ask if the notion of group homomorphisms also coincide.  That is, if $G$ and $H$ are groups, then we defined 
$$\abstr:\Hom(G,H)\to\Hom^\abstr(\abstr(G),\abstr(H))$$
in \cref{def:abstrisfunctor} as the function which takes a homomorphism, aka a pointed map $f=(Bf_\div,p_f):BG\to_*BH$ to the induced map of identity types 
$$f^\abstr\defequi \mathrm{ad}_{p_f}\ap{Bf_\div}:(\pt_G=\pt_G)\to(\pt_H=\pt_H)$$
 together with the proofs that this is an abstract group homomorphism from $\abstr(G)$ to $\abstr(H)$, c.f~\cref{def:grouphomomaxioms}.


\begin{lemma}
  If $G$ and $H$ are groups, then 
$$\abstr:\Hom(G,H)\to\Hom^\abstr(\abstr(G),\abstr(H))$$
is an equivalence.
\end{lemma}
\begin{proof}
\newcommand{\we}{\overset\sim\to}
  The equivalence of $\pathsp{}^G:BG\we(\typetorsor_G,\princ G)$ of \cref{lem:BGbytorsor} gives an equivalence
$$\pathsp{}:\Hom(G,H)\oldequiv (BG\to_*BH)\we((\typetorsor_G,\princ G)\to_*(\typetorsor_H,\princ H))
$$
Consider the map
$$A:((\typetorsor_G,\princ G)\to_*(\typetorsor_H,\princ H)\to \Hom^\abstr(\abstr(G),\abstr(H))$$
given by letting $A(f,p)$ be the composite 
$$\xymatrix{(\pt_G=\pt_G)\ar@{=}[d]^{\pathsp{}^G}_\downarrow&&&
  (\pt_H=\pt_H)\ar@{=}[d]^{\pathsp{}^H}_\downarrow\\
  (\princ G=\princ G)\ar[r]^-f&
  (f\princ G=f\princ G)\ar@{=}[rr]^-{q\mapsto p^{-1}qp}_\to&&
  (\princ H=\princ H)
}$$
(together with the proof that this is an abstract group homomorphism).  We see that we have shown the desired result if we prove instead $A$ is an equivalence.  The reason to complicate $\abstr$ this way is that it gets easier to write out a homotopy inverse.

If $(\phi,!):\Hom^\abstr(\abstr(G),\abstr(H))$ and $X:BG\to\Set$ is a $G$-torsor, recall the induced $H$-torsor $\phi_*X$ from \cref{rem:inducedGsetfromabstracthomomorphisms} and the identity $\eta_\phi:\phi_*\princ G=\princ H$. 
 %  let
%$\phi_*X:BH\to\Set$ be the $H$-set given by sending $z:BH$ to the set 
%$$\phi_*X(z)\defequi(\pt_H=z)\times_{\pt_G=\pt_G}X(\pt_G). \footnote{here I need set-quotients: $(\pt_H=z)\times_{\pt_G=\pt_G}X(\pt_G)$ is the coequalizer of the two maps $(\pt_H=z)\times X(\pt_G)\gets (\pt_H=z)\times{(\pt_G=\pt_G)}\times X(\pt_G)$ sending $(p,q,x)$ to $(p\,f(q),x)$ and $(p,X(q) x)$. I use $[p,x]$ to denote an element of this type}$$
%For $X$ the principal $G$-torsor we get an identity  $p_\phi:\phi_*\princ G=\princ H$ which, when evaluated at $z:BH$, corresponds under univalence to the equivalence 
%$$(\pt_H=z)\times_{\pt_G=\pt_G}(\pt_G=\pt_G)\to (\pt_H=z)$$ 
%sending $[p,q]:(\pt_H=z)\times_{\pt_G=\pt_G}(\pt_G=\pt_G)$ to $p\,\phi(q):(\pt_H=z)$.
Let 
$$B: \Hom^\abstr(\abstr(G),\abstr(H))\to ((\typetorsor_G,\princ G)\to_*(\typetorsor_H,\princ H)$$
be given by $B(\phi,!)=(\phi_*X,\eta_\phi)$

We show that $A$ and $B$ are inverse equivalences.  Given an abstract group homomorphism $(\phi,!):\Hom^\abstr(\abstr(G),\abstr(H))$, then $AB(\phi,!)$ has as underlying set map
$$\xymatrix{(\pt_G=\pt_G)\ar@{=}[d]^{\pathsp{}^G}_\downarrow&&&
  (\pt_H=\pt_H)\ar@{=}[d]^{\pathsp{}^H}_\downarrow\\
  (\princ G=\princ G)\ar[r]^-{\phi_*}&
  (\phi_*\princ G=\phi_*\princ G)\ar@{=}[rr]^-{q\mapsto \eta_\phi^{-1}q\eta_\phi}_\to&&
  (\princ H=\princ H),
}$$
and if we start with a $g:(\pt_G=\pt_G)$, then $\pathsp{}^G$ sends it to $\pathsp g^G\oldequiv\preinv (g)$.  Furthermore, $\phi_*\preinv (g)$ is $[\id,\preinv (g)]$ which is sent to $\preinv (\phi(g))$ in $\princ H=\princ H$ which corresponds to $\phi(g):(\pt_H=\pt_H)$ under $\pathsp{}^H$.  In other words, $AB(\phi,!)=(\phi,!)$.  The composite $BA$ is similar.
\end{proof}



\section{Sums of groups}
\label{sec:coprod}
We have seen how the group of integers $\ZZ=(S^1,\base)$ synthesizes the notion of one symmetry with no relations: every symmetry of the circle is of the form $\Sloop^n$ for some unique $n$.  Also, given any group $G=\aut_A(a)$, the set $a=a$ of symmetries of $a$ corresponds to the set of homomorphisms $\ZZ\to G$, \ie to pointed functions $(S^1,\base)\to_*(A,a)$ by evaluation at $\Sloop$.  What happens if we want to study more than one symmetry at the time?  

For instance, is there a group $\ZZ\vee%\boxplus
\ZZ$ so that for any group $G=\aut_A(a)$ a homomorphism $\ZZ\vee%\boxplus
\ZZ\to G$ corresponds to {\bf two} symmetries of $a$?  
At the very least, $\ZZ\vee\ZZ$ itself would have to have two symmetries and these two can't have any relation, since in a general group $G=\aut_A(a)$ there is a priori no telling what the relation between the symmetries of $a$ might be.  
Now, \emph{one} symmetry is given by a pointed function $(S^1,\base)\to_*(A,a)$ and so a \emph{pair} of symmetries is given by a function $f:S^1+S^1\to A$ with the property that $f$ sends each of the base points of the circles to $a$.  But $S^1+S^1$ is not connected, and so not a group.  To fix this we take the clue from the requirement that both the base points were to be sent to a common base point and \emph{define} $S^1\vee S^1$ to be what we get from $S^1+S^1$ when we \emph{insert an identity} between the two basepoints.
$$\xymatrix{\base\ar@(ul,dl)[]|{\Sloop}\ar@{.>}[rr]^{\text{identify!}}&&\base\ar@(ur,dr)[]|{\Sloop}}
$$
The amazing thing is that this works -- an enormous simplification of the classical construction of the ``free products'' or ``amalgamated sum'' of groups.  We need to show that the ``wedge'' $S^1\vee S^1$ is indeed a group, and this proof simultaneously unpacks the classical description.

% \begin{definition}
%   \label{def:wedge}
%   Let $(A_1,a_1)$ and $(A_2,a_2)$ be pointed types.  Their wedge is the pointed type $(A_1\vee A_2,a_{12})$ given as a higher inductive type\footnote{how/where discussed?} by
%   \begin{enumerate}
%   \item functions $i_1:A_1\to A_1\vee A_2$ and $i_2:A_2\to A_1\vee A_2$
%   \item an element $a_{12}: A_1\vee A_2$ (where we point the type),
%   \item identities $g_1:i_1a_1=a_{12}$ and $g_2:i_2a_2=a_{12}$.
%   \end{enumerate}
%   The function 
% $$i^g_1:(a_1=_{A_1}a_1)\to(a_{12}=_{A_1\vee A_2}a_{12})$$ 
% is defined by $i^g_1(p)\defequi g_1i_1(p)g_1^{-1}$, and likewise $i_2^g(q)\defequi g_2i_2(q)g_2^{-1}$.
% \end{definition}
% ((PICTURE))

% \begin{lemma}
%   \label{lem:wedgeofgpoidisgpoid}
%   Let $\aut_{A_1}(a_1)$ and $\aut_{A_2}(a_2)$ be decidable groups, then the wedge sum $\aut_{A_1\vee A_2}(a_{12})$ is a decidable group.
% \end{lemma}
% \begin{proof}
% That ${A_1\vee A_2}$ is connected follows by transitivity of identity, passing through the identities $g_1$ and $g_2$ in the wedge if necessary.

% We must prove that the wedge is a groupoid, \ie that all identity types are sets, which we do by giving an explicit description of the universal \covering.  The idea is that an identity in $a_{12}=x$ can be factored into a string of identities, each lying solely in $A_1$ or in $A_2$.  We define a family of sets consisting of exactly such strings of identities --  it is a set since $A_1$ and $A_2$ are groupoids -- and prove that it is equivalent to the family $P(x)\defequi(a_{12}=_{A_1\vee A_2}x)$ which consequently must be a family of sets.

%   We use the notation of \cref{def:wedge} freely, and for ease of notation, let $a_{2k+i}\defequi a_i$ for $i=1,2$, $k:\NN$.
% Define families of sets
% $$C_i:A_i\to\Set,\qquad i=1,2$$
% by 
% $$C_i(x)\defequi(a_i=_{A_i}x)\times\sum_{n:\NN}\prod_{1\leq k\leq n}\sum_{p_k:a_{i+k}=% _{A_{i+k}}
%   a_{i+k}}(p_k\neq\refl {a_{i+k}})$$
% when $x:A_i$.  Note that $p_k\neq\refl{a_{i+k}}$ makes sense and is a proposition since our groups are decidable; we leave it out when naming elements.  Also, set
% $$C(a_{12})\defequi \sum_{i:\bn 1}(a_i=a_i)\times\sum_{n:\NN}\prod_{1\leq k\leq n}\sum_{p_k:a_{i+k}=% _{A_{i+k}}
%   a_{i+k}}(p_k\neq\refl {a_{i+k}})
% $$
% Define $C(g_i):C_i(a_1)\to C(a_{12})$ by
% $$C(g_i)(p_0,n,p_1\dots,p_n)=
% \begin{cases}
%   (1-i,p_1,n-1,p_2\dots,p_n)& \text{ if }p_0=\refl{a_i}\\
%   (i,\refl{a_i},p_0,n+1,p_1\dots,p_n)& \text{ if }p_0\neq\refl{a_1}.
% \end{cases}
% $$
% $C_{12}$ is ((obviously or write out)) an equivalence, and so the triple $(C_1,C_2,C_{12})$ defines a family
% $$C:A_1\vee A_2\to\Set.$$
% We will show that $C$ is equivalent to $P\defequi \pathsp{a_{12}}$, which is given by $P(x)=(a_{12}=x)$, and so that the identity types of the wedge are sets.

% One way is the ``inclusion''; more precisely, 
% $$\alpha:\prod_{x:A_1\vee A_2}(P(x)\to C(x))$$ is given by letting identities be considered as strings of length zero: $\alpha_i(i_ia)(p)=(0,p):C_i(a)$.  This is well defined since $\alpha_2(i_2a_2)(gpg^{-1})=C_{12}\alpha_1(i_1a_1)(p)$ ((is this how you'd say this?  Feel free to fix.  Remember that $C(x)$ is a set)).
% The other way, 
% $$\beta:\prod_{x:A_1\vee A_2}(C(x)\to P(x)),$$ is given by composing the identities, using the glue $g_1$ and $g_2$ to make their ends meet: $\beta_1(n,p_0,\dots,p_n,!)\defequi i_1(p_0)g^{-1}i_2^g(p_1)i_1^g(p_2) \dots i_{n+1}^g(p_n)$. % (ending in $\dots gi_1(p_n)$ if $n$ is even and $\dots g^{-1}i_2(p_n)g$ if $n$ is odd) 
% and likewise for $\beta_2$ and the glue ((write)).

% That $\beta\alpha(p)=p$ follows by path induction: it is enough to prove it for 
% $p\defequi\refl{}$ ((here the assymmetry of our definition makes saying this slightly awkward since the basepoint is in $i_1A_1$; fix)).  That $\alpha\beta(n,p_0,\dots,p_n)=(n,p_0,\dots,p_n)$ follows by induction on $n$ ((write)).
% \end{proof}

We start by giving a definition of the wedge construction which is important for pointed types in general and then prove that the wedge of two groups is a group whose symmertries are arbitrary ``words'' in the original symmetries.

\begin{definition}
  \label{def:wedge}
  Let $(A_1,a_1)$ and $(A_2,a_2)$ be pointed types.  The \emph{wedge}\index{wedge of pointed types} is the pointed type $(A_1\vee A_2,a_{12})$ given as a higher inductive type\footnote{how/where discussed?} by
  \begin{enumerate}
  \item functions $i_1:A_1\to A_1\vee A_2$ and $i_2:A_2\to A_1\vee A_2$
  \item an identity $g:i_1a_1=i_2a_2$.
  \end{enumerate}
We point this type at $a_{12}\defequi i_1a_1$.
  The function 
$$i^g_2:(a_2=_{A_2}a_2)\to(a_{12}=_{A_1\vee A_2}a_{12})$$ 
is defined by $i^g_2(p)\defequi g^{-1}i_2(p)g$, whereas (for notational consistency only) we set $i_1^g\defequi i_1:(a_1=_{A_1}a_1)\to(a_{12}=_{A_1\vee A_2}a_{12})$.
Simplifying by writing $i:A_1+A_2\to A_1\vee A_2$ for the function given by $i_1$ and $i_2$ (with basepoints systematically left out of the notation), the induction priciple is
$$\prod_{C:(A_1\vee A_2)\to\UU}\sum_{s:\prod_{a:A_1+A_2}Ci(a)}%\sum_{s_2:\prod_{a:A_2}Ci_2(a)}
((s(a_1)=C(g^{-1})s(a_2))\,\to\,\prod_{x:(A_1\vee A_2)}C(x)).$$
\end{definition}


Unraveling the induction principle we see that if $B$ is a pointed type, then a  pointed function $f:A_1\vee A_2\to_* B$ is given by providing pointed functions $f_1:A_1\to_* B$ and $f_2:A_2\to_* B$  -- the identity $f_1(a_1)=f_2(a_2)$ which seems to be missing is provided by the requirement of the functions being pointed.  For the record
\begin{lemma}
  \label{lem:univvee}
  If $B$ is a pointed type, then the function 
  $$i^*(A_1\vee A_2\to_*B)\to(A_1\to_*B)\times(A_2\to_*B),\qquad i^*(f)=(fi_1,fi_2)
$$
is an equivalence.
\end{lemma}

Here is a picture of $i_2^g(p)$: it is the symmetry of the base point $a_{12}\defequi i_1a_1$ you get by \emph{first} moving to $i_2a_2$ with $g$, \emph{then} travel around with $p$ ($i_2p$, really) and finally go home to the basepoint with the inverse of $g$.
% $$\xymatrix{i_1a_1\ar@/^/[rr]^{g}&&i_2a_2\,\,\,\ar@/^/[ll]^{g^{-1}}\ar@(ur,dr)[]^{p}}
% $$

% $$\xy (-20,20)*+{A};(0,20)*+{B}
% **\crv{}
% \endxy$$
% $$% \xy (-20,20)*+{i_1a_1\,\,};(0,20)*+{}
% % **\crv{}\endxy
% % \xy (0,20)*+{i_2a_2};(0,20)*+{}
% % **\crv{(20,30)&(0,40)&(-20,30)}
% % \endxy
% \xy (-20,20)*+{a_{12}\,\,};(-20,20)*+{}
% **\crv{(10,20)&(-20,35)&(0,45)&(20,35)&(15,20)&(-10,20)}
% %?>*\dir{>}
% ?(.38)*{} *!LD!/^3pt/{>}
% ?(.95)*{} *!LD!/^-15pt/{g^{-1}}
% ?(.03)*{} *!LD!/^-5pt/{g}
% ?(.55)*{} *!LD!/^-7pt/{i_2p}
% \endxy
% $$
% $$\xy (0,20)*+{A};(60,0)*+{B}
% **\crv{(20,20)&(30,20)&(50,-20)&(60,-10)}
%  ?<*\dir{<} ?>*\dir{>}
%  ?(.65)*{\oplus} *!LD!/^-5pt/{x}
%  ?(.65)/12pt/*{\oplus} *!LD!/^-5pt/{x’}
%  ?(.28)*=0{\otimes}-/40pt/*+{Q}="q"
%  +/100pt/*+{P};"q" **\dir{-}
% \endxy
% $$
$$
\xy (-20,20)*+{};(-20,20)*+{}
**\crv{(15,20)&(18,20)&(-10,35)&(10,45)&(25,30)&(20,19)&(0,20)}
%?>*\dir{>}
?(0)*{} *!LD!/^-20pt/{i_1A_1}
?(.45)*{} *!LD!/^2pt/{>}
?(.95)*{} *!LD!/^-15pt/{g^{-1}}
?(.03)*{} *!LD!/^-5pt/{g}
?(.55)*{} *!LD!/^-7pt/{i_2p}
?(.65)*{} *!LD!/^-30pt/{i_2A_2}
?(.87)*{} *!LD!/^-12pt/{i_2a_2}
?(.86)*{} *!LD!/^-2pt/{\bullet}
?(1)*{} *!LD!/^-2pt/{\bullet}
?(1)*{} *!LD!/^-12pt/{a_{12}}
\endxy
$$

We now prove that wedges of decidable groups are decidable groups.   The idea is that an identity in $a_{12}=x$ can be factored into a string of identities, each lying solely in $A_1$ or in $A_2$.  We define a family of sets consisting of exactly such strings of identities --  it is a set since $A_1$ and $A_2$ are groupoids -- and prove that it is equivalent to the family $P(x)\defequi(a_{12}=_{A_1\vee A_2}x)$ which consequently must be a family of sets.
We need to be able to determine whether a symmetry is reflexivity or not, but once we know that, the symmetries of the base point in the wedge are then given by ``words $p_0p_1\dots p_n$'' where the $p_j$ alternate between being symmetries in the first or the second group, and none of the $p_j$ for positive $j$ are allowed to be reflexivity% : effectively a symmetry in the wedge can be decomposed into composites of symmetries in each of the groups
.  Note that there order of the $p_j$s is not negotiable: if I shuffle them I get a new symmetry.
\begin{lemma}
  \label{lem:wedgeofgpoidisgpoid}
  Let $\aut_{A_1}(a_1)$ and $\aut_{A_2}(a_2)$ be decidable groups, then the wedge sum $\aut_{A_1\vee A_2}(a_{12})$ is a decidable group.  

Let $C_1$ be the set of strings $(p_0,n,p_1,\dots,p_n)$ with $n:\NN$ and, for $0\leq j\leq n$ 
\begin{itemize}
\item $p_{j}:a_1=a_1$ for even $j$ 
\item $p_{j}:a_2=a_2$ for odd $j$ and 
\item $p_j$ is not reflexivity for $j$ positive (makes sense and is a proposition since our groups are decidable).
\end{itemize}
  Then the function given by composition in $a_{12}=a_{12}$
$$\beta:C_1\to(a_{12}=a_{12}),\qquad\beta(p_0,n,p_1,\dots p_n)\defequi i_1^gp_0i_2^gp_1i_1^gp_2\dots i_?^gp_n$$ 
(where $i_?^gp_n$  is $i_1^gp_n$ or $i_2^gp_n$ according to whether $n$ is even or odd) is an equivalence.
\end{lemma}
\begin{proof}
That the wedge is connected follows by transitivity of identity, if necessary passing through the identity $g:i_1a_1=i_2a_2$ in the wedge.

We must prove that the wedge is a groupoid, \ie that all identity types are sets, which we do by giving an explicit description of the universal \covering. 

 We use the notation of \cref{def:wedge} freely, and for ease of notation, let $a_{2k+i}\defequi a_i$ and $i_{2k+i}^g\defequi i_i^g$ for $i=1,2$, $k:\NN$.  % Let $i_1:A_1\to A_1\vee A_2$ and $i_2:A_2\to A_1\vee A_2$ be the two inclusions, let $g:i_1\pt_{A_1}=i_2\pt_{A_2}$ be the imposed identity in the (non-symmetric formulation of the) wedge sum based in $\pt_{A_1\vee A_2}\defequi i_1\pt_{A_1}$.  For ease of notation, let $a_{2k+i}$ denote $\pt_{A_i}$ for $i=1,2$, $k:\NN$.
Define families of sets
$$C_i:A_i\to\Set,\qquad i=1,2$$
by 
$$C_i(x)\defequi(a_i=_{A_i}x)\times\sum_{n:\NN}\prod_{1\leq k\leq n}\sum_{p_k:a_{i+k}=% _{A_{i+k}}
  a_{i+k}}(p_k\neq\refl {a_{i+k}})$$
when $x:A_i$.  Note that $p_k\neq\refl{a_{i+k}}$  is a proposition; we leave it out when naming elements. Hence, an element in $C_1(a)$ is a tuple
$(p_0,n,p_1,\dots,p_n)$ where $p_0:a_1=_{A_1}a$, $p_1:a_2=_{A_2}a_2$, $p_2:a_1=_{A_1}a_1$, and so on -- alternating between symmetries of $a_1$ and $a_2$, and where $p_0$ is the only identity allowed to be $\refl{}$. Define $C_{12}:C_1(a_1)\to C_2(a_2)$ by
$$C_{12}(p_0,n,p_1\dots,p_n)=
\begin{cases}
  (\refl{a_2}0,)&\text{ if }p_0=\refl{a_1}, n=0,\\
  (p_1,n-1,p_2\dots,p_n)& \text{ if }p_0=\refl{a_1},n\neq0,\\
  (\refl{a_2},n+1,p_0,\dots,p_n)& \text{ if }p_0\neq\refl{a_1}.
\end{cases}
$$
It is perhaps instructive to see a table of the values $C_{12}(p_0,n,p_1,\dots,p_n)$ for $n<3$:
\begin{center}
  \begin{tabular}{r|c cc}
    &$(p_0,0)$&$(p_0,1,p_1)$&$(p_0,2,p_1,p_2)$\\
    \hline
    $p_0=\refl{a_1}$&$(\refl{a_2},0)$&$(p_1,0)$&$(p_1,1,p_2)$\\
    $p_0\neq\refl{a_1}$&$(\refl{a_2},1,p_0)$&$(\refl{a_2},2,p_0,p_1)$&$(\refl{a_2},3,p_0,p_1,p_2)$
  \end{tabular}
\end{center}
Since $C_{12}$ is an equivalence, the triple $(C_1,C_2,C_{12})$ defines a family
$$C:A_1\vee A_2\to\Set.$$
In particular, $C(a_{12})\defequi C_1(a_1)$.
For $x:A_1$ we let $i^C_1:C_1(x)\to C(i_1(x))$ be the induced equivalence, and likewise for $i^C_2$.
We will show that $C$ is equivalent to $P\defequi \pathsp{a_{12}}$, where $P(x)\defequi(a_{12}=x)$, and so that the identity types in the wedge are equal to the sets provided by $C$.

One direction is by transport in $C$; more precisely, 
$$\alpha:\prod_{x:A_1\vee A_2}(P(x)\to C(x))$$ is given by transport with $\alpha(a_{12})(\refl{a_{12}})\defequi(\refl{a_{1}},0):C(a_{12})$.  %This is well defined since $\alpha_2(i_2\pt_{A_2})(gpg^{-1})=C_{12}\alpha_1(i_1\pt_{A_1})(p)$ ((is this how you'd say this?)).
The other way, 
$$\beta:\prod_{x:A_1\vee A_2}(C(x)\to P(x))$$ is given by composing identities, using the glue $g$ to make their ends meet: 
$$\beta(i_1a)(p_0,n,p_1,\dots,p_n)\defequi i_1(p_0)i_2^g(p_1)i_3^g(p_2) \dots i_{n+1}^g(p_n)$$ 
(here the definition $\dots i_3^g\defequi i_1^g\defequi i_1$ proves handy since we don't need to distinguish the odd and even cases)  % (ending in $\dots gi_1(p_n)$ if $n$ is even and $\dots g^{-1}i_2(p_n)g$ if $n$ is odd) 
and likewise 
$$\beta(i_2a)(p_0,n,p_1,\dots,p_n)\defequi i_2(p_0)g\,i_1^g(p_1)i_2^g(p_2) \dots i_{n}^g(p_n)$$ and compatibility with the glue $C_{12}$ is clear since the composite $\refl{x}p$ is equal to $p$.
%$\beta_1(p_0,n,p_1,\dots,p_n,!)\defequi i_1(p_0)g^{-1}i_2(p_1)gi_1(p_2) g^{-1}\dots $ (ending in $\dots gi_1(p_n)$ if $n$ is even and $\dots g^{-1}i_2(p_n)g$ if $n$ is odd) and likewise for $\beta_2$ and the glue ((write)).

For notational convenience, we hide the $x$ in $\alpha(x)(p)$ and $\beta(x)(p)$ from now on.

That $\beta\alpha(p)=p$ follows by path induction: it is enough to prove it for $x=a_{12}$ and
$p\defequi\refl{a_{12}}$:
$$\beta\alpha(\refl{a_{12}})=\beta(\refl{a_1},0)=i_1^g\refl{a_1}=\refl{a_{12}}.$$  

That $\alpha\beta(p_0,n,p_1\dots,p_n)=(p_0,n,p_1,\dots,p_n)$ follows by induction on $n$ and $p_0$.  For $n=0$ it is enough to consider  $x=a_{12}$ and $p_0=\refl{a_1}$, and then 
$\alpha\beta(\refl{a_1},0)\defequi\alpha(\refl{a_{12}})\defequi(\refl{a_1},0)$.  In general, (for $n>0$) 
\begin{align*}
  \alpha\beta(p_0,n,p_1\dots,p_n)
=&\trp{C,i_1(p_0)i_2^g(p_1)i_1^g(p_2) \dots i_{n+1}^g(p_n)}(\refl{a_1,0})\\
=&\trp{C,i_1(p_0)}\dots\trp{C,i_{n+1}^g(p_n)}(\refl{a_1,0}).
\end{align*}
  The induction step is as follows: let $0< k\leq n$, then 
\begin{align*}
  &\trp{C,i_k^gp_{k-1}}i^C_{k-1}(p_k,n-k-1,p_{k+1},\dots,p_n)\\
  =&\trp{C,i_k^gp_{k-1}}i^C_k(\refl{a_{k-1}},n-k,p_k,\dots,p_n)\\
  =&i^C_k\trp{C_k,p_{k-1}}(\refl{a_{k-1}},n-k,p_k,\dots,p_n)\\
  =&(p_{k-1},n-k,p_k,\dots,p_n).
\end{align*}
((please see whether this makes sense to anybody but yvt))
\end{proof}

\begin{definition}
  \label{def:sumofgroup}
  If $G_1=\aut_{A_1}(a_1)$ and $G_1=\aut_{A_1}(a_1)$ are groups, then their \emph{sum}\index{sum of groups} is defined as
  $$G_1\vee G_2\defequi \aut_{A_1\vee A_2}(a_{12}).$$ The homomorphisms $i_1:G_1\to G_1\vee G_2$ and $i_2:G_2\to G_1\vee G_2$ induced from the structure maps  $i_1:A_1\to A_1\vee A_2$ and  $i_2:A_2\to A_1\vee A_2$ are also referred to as structure maps.
\end{definition}
\begin{lemma}
  \label{lem:sumofgroupsISsum} If $G_1$, $G_2$ and $G$ are groups, then the function
  $$\Hom(G_1\vee G_2,G)\to\Hom(G_1,G)\times\Hom(G_2,G)$$ 
given by restriction along the structure maps is an equivalence.
\end{lemma}
\begin{proof}
  ((write))
\end{proof}
Specializing, we return to our initial motivation and see that mapping out of a wedge of two circles \emph{exactly} captures the information of two independent symmetries:
\begin{corollary}
  \label{cor:ZplusZuniv}
  If $G$ is a group, then the functions
  $$\Hom(\ZZ\vee\ZZ,G)\to \Hom(\ZZ,G)\times\Hom(\ZZ,G)\to G\times G$$
  are equivalences.((fix language))
\end{corollary}


%\section{structure of identity types}
%\section{automorphism 1-group = fundamental group (hint at higher groups)}
%\section{homomorphisms induced by functions (early)}
\section{more examples: symmetric groups, integers, cyclic groups and modular arithmetic}
\section{group actions, orbits and fixed points}

\footnote{TO BE MOVED TO AFTER the chapter on symmetry (need Burnside -  a $G$-set splits into orbits - etc) has been covered.  
Also, some proofs are written in a pseudoclassical way just to remind me of the idea.  It is not a typo and will by typied}

\section{Subgroup}
\label{sec:subgroups}
In our discussion of the group $\ZZ=\aut_{S^1}(\base)$ of integers in we discovered that the ``subsymmetries'' formed a very organized structure.  For each natural number $n$ we obtained a set of subsymmetry the entire identity type $\base=\base$, namely the set of all the iterates $(\Sloop^{n})^m$ where $m$ varies over the integers.  When $n$ was positive this was realized as the $n$-fold \covering of $S^1$ , when $n=0$ this was given by the universal \covering.  

For other groups the ``subsymmetries'' form more involved structures.  One thing is that our concept of a subtype of $B$ is merely the first projection $\sum_{b:B}P(b)\to B$, where the $P$ is a family of propositions.  Another thing is that for group the ``sub'' refers to the associated abstract groups, so that ``$BH$ is a subgroup of $BG$'' should \emph{not} mean that ``$BH$ is a subtype of $BG$'', but that we have a group homomorphism $f:\Hom(H,G)$ so that the induced function $(\pt_H=\pt_H)\to(\pt_G=\pt_G)$ is an ``inclusion of a subset''. 

Now, as we've seen, that   $f:(\pt_H=\pt_H)\to(\pt_G=\pt_G)$ is an injection (preimages are propositions) is equivalent to the preimages of $Bf:BH\to BG$ being sets.  Hence we get the following neat formulation.
\newcommand{\typesubgroup}{\mathbf{Subgp}}
    \begin{definition}
      \label{def:subgroup}
      Let $G$ be a group.  
      The \emph{type of subgroups of $G$}\index{type!subgroup} is the type
      $$\typesubgroup_G\defequi\sum_{H:\typegroup}\sum_{f:\Hom(H,G)}\isset(Bf^{-1}(\pt_G)).$$
       A subgroup $(H,f,!)$ is
      \begin{enumerate}
      \item \emph{non-trivial}\index{non-trivial subgroup} if $BH$ is not contractible
      \item \emph{proper}\index{proper subgroup} if $Bf$ is not an equivalence.
      \end{enumerate}
    \end{definition}
In other words, a subgroup of $G$ is a group homomorphism $f:\Hom(H,G)$ such that the fibers of $Bf$ are sets and an identity between two subgroups $f:\Hom(H,G)$ and $f':\Hom(H',G)$ is an identity $p:H'=_{\typegroup}H$ such that $f'=f\,p$ (a proposition since $\Hom(H',G)$ is a set).
\begin{remark}
  If you're familiar with the set-theoretic flavor of things, it is important to distinguish between subgroups and injective group homomorphisms.  
Our use of ``subgroup'' can be defended as follows.  It corresponds in set-theoretic language to saying that a subgroup is an injective homomorphism modulo that precomposing with an isomorphism yields identical subgroups.  
Set-theory offers the luxury of having a representative in every equivalence class: namely the image of the injection, type theory does not.
\end{remark}
%\newcommand{\ker}{\mathrm{ker}}
If $\phi:\Hom^\abstr(\mathcal G,\mathcal G')$ is an abstract group homomorphism, the kernel of $\phi$ consists of the $g$ in $\mathcal G$ with the property that $\phi(g)=e_{G'}$ and the cokernel is the quotient set of $\mathcal G'$ by the relation that if $g:\mathcal G$ and $g':\mathcal G'$, then $g'\sim g'\cdot\phi(g)$.  In our setup with a group homomorphism $f:\Hom(G,G')$ these two definition are joined into one construction we call $F(f)$, where the cokernel is the set of components and the kernel is given as a component of $F(f)$.
\begin{definition}
  \label{def:kernel}\label{def:cokernel}
  Let $f:\Hom(G,G')$ be a homomorphism and let $F(f)$ be the groupoid
$\sum_{z:BG}(\pt_{G'}=f(z))$.\footnote{where we, as usual, have written ``$f(z)$'' instead of ``$Bf_\div(z)$'' and $p_f:\pt_{G'}=f(\pt_G)$ is part of the data of which $f$ consists} 

The \emph{image}\index{image} of $f$ is the subgroup \footnote{COMEBACK/GOODNIGHT}

The \emph{cokernel}\index{cokernel} of $f$ is the set of components of $F(f)$.  If $f:\Hom(G,G')$ is clear from the context and displays $G$ as a subgroup of $G'$, we often write $G'/G$ for the cokernel of $f$.

Using the path $p_f:\pt_{G'}=f(\pt_G)$ have an element $\pt_{\ker f}\defequi(\pt_G,p_f):F(f)$.  
Define the \emph{kernel}\index{kernel} $\ker f$ of $f$ to be the group defined by the component 
$B\ker f_\div\defequi \sum_{z:BG}\sum_{p:\pt_{G'}=f(z)}||\pt_{\ker f}=(z,p)||$ of $\pt_{\ker f}$ in $F(f)$.  

The first projection $B\ker f\to_* BG$ shows that $\ker f$ is a subgroup of $G$ since the preimages are equivalent to the sets $\sum_{z:BG}\sum_{p:\pt_{G'}=f(z)}||\pt_{\ker f}=(z,p)||$.
\end{definition}

\begin{lemma}
  \label{lem:kerandcoker}\footnote{((WRITE))}
  Show the characterizing properties of kernels and cokernels.  
\end{lemma}

\begin{lemma}
  \label{lem:iso2}
  Let $G$ be a group, $H$ a subgroup and $N$ a normal subgroup of $G$.
  
\end{lemma}

\begin{definition}
  \label{def:finitegrd}Let $n:\NN$ be positive.  
A \emph{finite group of order $n$}\index{finite group! of order $n$} is a group $G$ such that the set $pt_G=\pt_G$ is in $\fin_n$. 
A \emph{finite group}\index{finite group} is a group such that $\pt_G=\pt_G$ is a finite set. 
\end{definition}
So, the trivial group has order $1$, the cyclic group $C_n$ of order $n$ has order $n$ (which is good) and the permutation group $\Sigma_n$ has order $n!$.
    \begin{lemma}
      If $p$ is a prime, then  the cyclic group $C_p$ has no non-trivial proper subgroups.
    \end{lemma}
    \begin{proof}
      By Lagrange's theorem (\cref{thm:lagrange}) a subgroup has order dividing $p$, \ie either $1$ or $p$.
    \end{proof}


\section{Cauchy's theorem}
\begin{lemma}
  \label{lem:fixedptsize}
  Let $p$ be a prime and $G$ a group of order $p^n$ for some positive $n:\NN$.  If $X:BG\to\Set$ is a non-empty finite $G$-set such that the cardinality of $X(\pt_G)$ is divisible by $p$, then the cardinality of the set of fixed points $X^G\defequi\prod_{z:BG}X(z)$ is divisible by $p$.
\end{lemma}
\begin{proof}
  Recall that the evaluation at $\pt_G$ gives an injection of sets $X^G\to X(\pt_G)$ through which we identify $X^G$ with the subset ``$X(\pt_G)^G$'' of all trivial orbits of $X(\pt_G)$.
 The orbits of $X(\pt_G)$\footnote{or of $X$?  Reference for identification of orbits with quotiens by stabilizers} all have cardinalities that divide the order $p^n$ of $G$.  
This means that all the the cardinalities of the non-trivial orbits (as well as of $X(\pt_G)$) are positive integers divisible by $p$. 

 Burnside's Lemma \cref{lem:burnsides-lemma} states that $X(\pt_G)$ is the sum of its orbits.
Hence the cardinality of the set of all trivial orbits, \ie of $X^G$, is the difference of two numbers both divisible by $p$.  
\end{proof}

\begin{theorem}
  \label{thm:cauchys}
  Let $p$ be a prime and let $G$ be a finite group of order divisible by $p$.  
Then $G$ has a subgroup which is cyclic of order $p$. 
\end{theorem}
\begin{proof}
\footnote{NOTE: three variants follow.  Have to choose one.  The middle needs some background essentially boiling down to $BC_n$ being the truncation of the $n$th Moore space.}
Recall the cyclic group $C_p$ of order $p$ given by the pointed connected groupoid
$$BC_p\defequi(\sum_{S:\Set}\sum_{j:S=S}||(S,j)=\zet/p||,(\zet/p,!)),
$$
 where $\zet/p:\sum_{S:\Set}S=S$ was a particular model of a set with $p$ element together with a successor modulo $p$.  Informally, $BC_p$ consists of pairs $(S,j)$, where $S$ is a set of cardinality $p$ and $j:S=S$ is a cyclic permutation in the sense that for $0<k<p$ we have that $j^k$ is not $\refl{}$ while $j^p=\refl{}$.  Note also that $j^?:\bn p\to ((S,j)=(S,j))$ given by $j^?(k)=j^k$ is an equivalence (just as for the integers, a symmetry of $(S,j)$, \ie an $f:S=S$ so that $fj=jf$, must be $j^k$ for some $k:\bn p$, and if $k\neq l$, then $j^k\neq j^l$)

If $(S,j):BC_p$ let 
$$A(S,j)\defequi ((S,j)=(S,j)\to (\pt_G=\pt_G)).$$  Since we have an equivalence $j^?:\bn p\to ((S,j)=(S,j))$ we get that $J:A(S,j)\to \prod_{\bn p}(\pt_G=\pt_G)$ given by $J(g)=(g_{j^0},g_{j_1},\dots,g_{j^{p-1}})$ is an equivalence.  Define $\mu:\prod_{(S,j):BC_p}(A(S,j)\to(\pt_G=\pt_G))$ by $\mu_{(S,j)}(g)\defequi g_{j^0}\cdot\dots\cdot g_{j^{p-1}}$ and let $X:BC_p\to\Set$ be the $G$-set defined by  
$$X(S,j)\defequi\sum_{g:A(S,j)}\mu_{(S,j)}g=e_G.$$ 
The map from $X(S,j)$ to the $p-1$-fold product of $(\pt_G=\pt_G)$ with itself sending $(g,!)$ to $(g_{j^1},\dots,g_{j^{p-1}})$ is an equivalence ($\mu_{(S,j)}g=e_G$ says exactly that $g_{j^0}$ can be reconstructed as $(g_{j^1}\cdot\dots\cdot g_{j^{p-1}})^{-1}$), so $X(S,j)$ is a set of cardinality $p-1$ times the order of $G$.  In particular, $p$ divides the order of $X(S,j)$.

Specializing to $(S,j)$ being $\zet/p$ and allowing to index the elements in $A(\zet/p)$ with $i:\bn p$ (instead of the very awkward ``$(\sqrt[p]\id)^i$'' as purism would dictate) we proceed as follows.

Now, a $C_p$-fixed point of $X(\zet/p)$ is an element $(g_0,\dots,g_{p-1},!)$ such that $(g_0,\dots,g_{p-1},!)=(g_1,\dots,g_{p-1},g_0,!)$, \ie $g_0=g_1=g_2=\dots=g_{p-1}$\footnote{if I am allowed to write that}.  In other words, a fixed point is of the form $(g,\dots,g,!)$, where $!$ expresses that $g^p=e_G$:
$X(\zet/p)^{C_p}$ is equivalent to $\sum_{g:\pt_G=\pt_G}g^p=e_G.$  If we can show that $(g,!):X(\zet/p)^{C_p}$ is nonempty, we'd have established an abstract cyclic subgroup consisting of the powers of $g$.  Of course, setting $g=e_G$ will give us such a fixed point, but if $g\neq e_G$ we get a cyclic subgroup of order $p$ of $G$.

 Now, \cref{lem:fixedptsize} claims that $p$ divides the cardinality of $X(\zet/p)^{C_p}$, and since there \emph{are} fixed points, there must be at lest $p$ fixed points.  One of them is the trivial one (given by $g=e_G$ above), but the others are nontrivial.

% $X(\zet/p)$ splits as a disjoint union of its orbits.\footnote{the formulation here depends on things to come: how do we express the decomposition into orbits?}  Since $p$ is prime, the group $C_p$ is simple ((where proved?)), and so the orbits are either singletons (fixed points) or free orbits:
% $$X(\zet/p)= X(\zet/p)^{C_p}+(\text{free part of }X(\zet/p)).$$
% Since the order of $G$ is divisible by $p$, so is the cardinality of $X(\zet/p)$ \emph{and} of the free part of $X(\zet/p)$.  Hence, the number of fixed points is divisible by $p$, and it is not zero, so there must be nontrivial fixed points.

\footnote{ALTERNATIVELY:
  Consider  the $p-1$-fold product $(\bn{(p-1)}\to (\pt_G=\pt_G))$ of $\pt_G=\pt_G$ with itself.  We give this set the structure of a $\ZZ/p$-set as follows: ((here it is convenient to say that a $\ZZ/p$-set is the same as a $\ZZ$-set commuting with the $p$-fold cover)) define $X:S^1\to\Set$ by $X(\base)\defequi (\bn{(p-1)}\to (\pt_G=\pt_G))$ and by setting $X(\Sloop)$ to be the element in $X(\base)=(\base)$ sending $(g_1,\dots,g_{p-1})$ to $(g_2,\dots,g_{p-1},(g_1\cdots g_{p-1})^{-1})$.  Note that
$$\xymatrix{S^1\ar[d]_{(-)^p}\ar[dr]^X&\\S^1\ar[r]_X&\Set}$$
commutes and so we get a $\ZZ/p$-action ((expand)).  A fixed point of this action is an element of $X(\base)$ of the form $(g,g,\dots,g)$ such that $g^{p-1}=g^{-1}$  (expand)). The choice $g=e$ always gives such a fixed point, but if there is any other point ... (do you need decidability here?), the pointed map $S^1\to BG$ given by sending $\Sloop$ to $g$ gives the desired subgroup. Hence we need to show that $X$ has fixed points.

Now, $X(\base)$ splits as a disjoint union of its orbits.  Since $p$ is prime, the group $\ZZ/p$ is simple ((where proved?)), and so the orbits are either singletons (fixed points) or free orbits:
$$X(\base)= X(\base)^{\ZZ/p}+(\text{free part of }X(\base)).$$
Since the order of $G$ is divisible by $p$, so is the cardinality of $X(\base)$ \emph{and} of the free part of $X(\base)$.  Hence, the number of fixed points is divisible by $p$, and it is not zero, so there must be nontrivial fixed points.


ALTERNATIVELY:
Recall the cyclic group of order $p$.  For the sake of convenience, we identify $\bn p\times\bn 1$ and $\bn p$, so that elements will be denoted $0,1,2,\dots$ and not $(0,0), (1,0),(2,0)\dots$, and we also write $s$ instead of $\sqrt[p]{\id}$ for the element in $\bn p=\bn p$ that shifts to the successor (mod $p$).  In other words we choose an identification $\zet/p=(\bn p,s)$ once and for all.  If 
$$(S,j):BC_p\defequi\sum_{S:\UU}\sum_{j:X=X}||(S,j)=(\bn p,s)||,$$ consider the set 
$$X(S,j)\defequi\sum_{g:S\to (\pt_G=\pt_G)}||\sum_{a:(S,j)=(\bn p,s)}g_{a^{-1}(0)}\cdot\dots\cdot g_{a^{-1}(p-1)}=e_G||.$$ 
Note that if $g:\bn p\to(\pt_G=\pt_G)$, then we have an equality of propositions
$$||\sum_{a:(\bn p,s)=(\bn p,s)}g_{a^{-1}(0)}\cdot\dots\cdot g_{a^{-1}(p-1)}=e_G||=g_{0}\cdot\dots\cdot g_{p-1}=e_G$$
since $(g_1\cdot\dots\cdot g_{p-1}=e_G)=(g_2\cdot\dots\cdot g_{p-1}\cdot g_1=e_G)$,
and so 
$$X(\bn p,s)=\sum_{g:\bn p\to (\pt_G=\pt_G)}g_{0}\cdot\dots\cdot g_{p-1}=e_G$$ 
which is equivalent to the $p-1$-fold product of $\pt_G=\pt_G$ with itself ($g_0=(g_1\cdot\dots\cdot g_{p-1})^{-1}$, but all the other $g_i$s are then chosen arbitrarily).  Consequently, the number of elements in $X(\bn p,s)$ is divisible by $p$.

Now, a $C_p$-fixed point of $X(\bn p,s)$ is an element $(g_0,\dots,g_{p-1},!)$ such that $(g_0,\dots,g_{p-1},!)=(g_1,\dots,g_{p-1},g_0,!)$, \ie $g_0=g_1=g_2=\dots=g_{p-1}$\footnote{if I am allowed to write that}.  In other words, a fixed point is of the form $(g,\dots,g,!)$, where $!$ expresses that $g^p=e_G$:
$$X(\bn p,s)^{C_p}=\sum_{g:\pt_G=\pt_G}g^p=e_G.$$  If we can show that $(g,!):X(\bn p,s)^{C_p}$ is nonempty, we'd have established an abstract cyclic subgroup consisting of the powers of $g$.  Of course, setting $g=e_G$ will give us such a fixed point.

 Now, $X(\bn p,s)$ splits as a disjoint union of its orbits.  Since $p$ is prime, the group $C_p$ is simple ((where proved?)), and so the orbits are either singletons (fixed points) or free orbits:
$$X(\bn p,s)= X(\bn p,s)^{C_p}+(\text{free part of }X(\bn p,s)).$$
Since the order of $G$ is divisible by $p$, so is the cardinality of $X(\bn p,s)$ \emph{and} of the free part of $X(\bn p,s)$.  Hence, the number of fixed points is divisible by $p$, and it is not zero, so there must be nontrivial fixed points.
}%endfootnote
\end{proof}
\begin{lemma}
  \label{lem:nontrivcenter}
  Let be $G$ be a finite subgroup of order $p^n$, where $p$ is prime and $n$ a positive integer.  
Then the center $Z(G)$ of $G$ is nontrivial. 
(point to center in the symmetry chapter)
\end{lemma}
\begin{proof}
  Recall the $G$-set $\Ad_G:BG\to\Set$ given by $Ad_G(z)=(z=z)$.  Then the map  $$\ev_{\pt_G}:\prod_{z:BG}(z=z)\to(\pt_G=\pt_G),\quad \ev_G(f)=f(\pt_G)$$  
has the structure of a (n abstract) inclusion of a subgroup; namely the inclusion of the center $Z(G)$ in $G$.  
The center thus represents the fixed points of the $G$-set $\Ad_G$.  
Since $G$ has order a power of $p$, all orbits but the fixed points have cardinality divisible by $p$.  
Consequently, Burnside's lemma states that the number of fixed points, \ie the order of $Z(G)$, must be divisible by $p$.
\end{proof}
\begin{corollary}
  \label{cor:orderpsquaredgroups}
  If $G$ is a noncyclic group of order $p^2$, then $G$ of the form $C_p\times C_p$.
\end{corollary}
\begin{proof}
  The center $Z(G)$ is by \cref{lem:nontrivcenter} of order $p$ or $p^2$.
  Since $G$ is not cyclic we have that $g^p=e_G$ for all $g:\pt_G=\pt_G$.    
((To be continued: the classical proof involves choosing nontrivial elements  -- see what can be done about that))
\end{proof}
\section{Sylow's Theorems}
\label{sec:sylow}
\begin{theorem}
  \label{thm:sylow1}
  If $p$ is a prime, $n:\NN$ and $G$ a finite group whose order divides $p^n$, then $G$ has a subgroup of order $p^n$.
\end{theorem}
\begin{proof}\footnote{In this proof I refer to \cref{lem:aut-orbit} (which says that $N_G(K)/K$ is equivalent to the set of automorphisms of $G/K$ in the orbit category) to claim that the $K$-fixed points of $G/K$ may be identified with $N_G(K)/K$.   Also I use that the order of a pullback of a subgroup $L$ along a surjection is the product of the order of the kernel and the order of $L$}
  We prove the result by induction on $n$.  
If $n=0$ we need to have a subgroup of order $1$, which is witnessed by the trivial subgroup.
%If $n=1$, this is Cauchy's \cref{thm:cauchys}.  
If $n>0$, assume by induction that $G$ contains a subgroup $K$ of order $p^{n-1}$.  
Now, $K$ acts on the set $G/K$.  
The cardinality of $G/K$ is divisible by $p$ (since $p^n$ divides the order of $G$), and so by \cref{lem:fixedptsize} the fixed point set $(G/K)^K$ has cardinality divisible by $p$.  

Recall the normalizer subgroup $N_G(K)$ from \cref{sec:noether-theorems} -- the ``largest subgroup of $G$ containing $K$ as a normal subgroup''.
By \cref{lem:aut-orbit}, the quotient group $N_G(K)/K$ may be identified with $(G/K)^K$, and so has order divisible by $p$.  
By Cauchy's \cref{thm:cauchys} there is a subgroup $L$ of $N_G(K)/K$ of order $p$.  
Taking the preimage of $L$ under the projection $\mathrm{pr}:N_G(K)\to N_G(K)/K$ we obtain a subgroup $H$ of $N_G(K)$ of order $p^n$ ($H$ is a free $K$-set with $p$ orbits).  The theorem is proven by considering $H$ as a subgroup of $G$.
\end{proof}



\section{Historical remarks}
\label{sec:grouphistory}

% Move in place

% \begin{remark}
%   Notice that the last statement  (``More precisely\dots'')  not only asserts that there \emph{exist} inverses, but that there actually is a (preferred and consistent) way to produce them.

% Classically this was in many instances unnecessay to say because there was a unique inverse, and the distinction is not mentioned in introductory texts.  However, then this very point had to be revisited later on.  In our proof relevant setting it is obvious that the ultimate statement will have to go beyond an assertion that inverses exist.
% \end{remark}

%%% Local Variables:
%%% mode: latex
%%% fill-column: 144
%%% TeX-master: "book"
%%% End:


%the below is the illustration used for the n-fold \covering in the deck trafo section.
% Move in place
% \begin{figure}
%   \centering
%   \begin{tikzpicture}
%     \node (A) at (2,2) {$\sqrt[n]X$};
%     \node (B) at (2,-2) {$\bn{n}$};
%     \draw[->] (A) -- node[auto] {$p$} (B);
%     \foreach \y in {-2,0,1,2}
%     { \begin{scope}[shift={(0,\y)}]
%         \foreach \x in {0,...,4}
%         { \node[fill,circle,inner sep=1pt] at (180+72*\x:1 and .3) {}; }
%         \foreach \x in {0,...,3}
%         { \draw[-stealth] (180+72*\x:1 and .3) arc(180+72*\x:252+72*\x:1 and .3); }
%       \end{scope} }
%     \begin{scope}[shift={(0,-2)}]
%       \draw[-stealth] (108:1 and .3) arc(108:180:1 and .3);
%     \end{scope}
%     \foreach \y in {1,2}
%     { \begin{scope}[shift={(0,\y)}]
%         \draw[-stealth] (108:1 and .3)
%         .. controls ++( 5:-.3) and ++(80:.2) .. (-.7,-.4)
%         .. controls ++(80:-.2) and ++(90:.2) .. (-1,-1);
%       \end{scope} }
%     \draw[-stealth] (108:1 and .3)
%     .. controls ++( 5:-.3) and ++(80:.2) .. (-.7,-.4);
%     \node (dz) at (-.7,-.7) {\footnotesize $\vdots$};
%     \begin{scope}[shift={(0,3)}]
%       \draw[-stealth] (-.7,-.4)
%       .. controls ++(80:-.2) and ++(90:.2) .. (-1,-1);
%     \node (da) at (-.7,0) {\footnotesize $\vdots$};
%     \end{scope}
%   \end{tikzpicture}
%   \caption{The $n$'th root of an endomorphism, with projection}
%   \label{fig:rootproj}
% \end{figure}
