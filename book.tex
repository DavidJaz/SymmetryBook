\documentclass[a4,12pt]{amsbook}
\input macros
\begin{document}
\input top
\chapter{Introduction to the topic of this book}
\input intro
\chapter{An introduction to univalent mathematics}
\input intro-uf
\chapter{The universal symmetry: the circle}
\input circle
\chapter{Groups}
\input group
\chapter{Symmetry}
\input symmetry
\chapter{Gerbes}
\input gerbes
\chapter{Finite group theory}
\input fingp
\section{cycle decompositions}
\section{Lagrange}
\section{Sylow stuff?}

\chapter{Euclidean geometry}
\input EuclideanGeometry

\chapter{Geometry (first look)}
\section{incidence geometries and the Levi graph}
\section{euclidean planes}
\section{ruler and compass constructions}
\section{affine planes and Pappus' law}
\section{projective planes}
\chapter{Vector spaces and linear groups}
Quotients; subspaces (= ?). Bases and so. Dual space; orthogonality. (all of this depends on good implementations of subobjects). Eigen-stuff. Characteristic polynomials; Hamilton-Cayley. 
\section{the algebraic hierarchy: groups, abelian groups, rings, fields}
\section{vector spaces}
\section{the general linear group as automorphism group}
\section{determinants(†)}
\chapter{Field theory}
\section{examples: rationals, polynomials, adding a root, field extensions}
\section{ordered fields, real-closed fields, pythagorean fields, euclidean fields}
\section{complex fields, quadratically closed fields, algebraically closed fields}
\section{Diller-Dress theorem(†)}
\chapter{Classification of wallpaper groups(†)}
\chapter{Affine geometry}
Barycentric calculus. Affine transformations. Euclidean / Hermitian geometry (isometries, conformity...)
\section{affine frames, affine planes}
\section{the affine group as an automorphism group}
\section{the affine group as a semidirect product}
\section{affine properties (parallelism, length ratios)}
\chapter{Bilinear forms}
\chapter{Inversive geometry (Möbius)}
\section{residue at a point is affine}
\section{Miquel's theorem}
\chapter{Projective geometry}
Projective spaces (projective invariance, cross ratio, harmonic range...). Conics/quadrics. (Classification in low dimensions?)
\par
complex algebraic plane projective curves (tangent complexes, singular points, polar, hessian, ...).
\section{projective frames}
\section{the projective group and projectivities}
\section{projective properties (cross-ratio)}
\section{fundamental theorem of projective geometry}
\chapter{Minkowski space-time geometry}
Affine spaces + vector spaces + quadric with some signature.
\chapter{Kleinian geometries}
\section{conics and dual conics}
\section{elliptic geometry}
\chapter{Galois theory}
\section{Covering spaces and field extensions}
\section{separable/normal/etc}
\section{fundamental theorem}
\chapter{Impossible constructions}
\section{doubling the cube}
\section{trisecting the angle}
\section{squaring the circle}
\section{7-gon}
\section{quintic equations}
\chapter{Possible constructions}
\section{5-gon and the icosahedron}
(cyclotomic field of deg 5 has galois group $(\ZZ/5\ZZ)^\times \equiv \ZZ/4\ZZ$)
\section{17-gon and 257-gon}
\section{cubics and quartics}
\chapter{Witt theory, SOSs, Artin-Schreier}
\section{quadratic forms}
\section{Grothendieck-Witt ring}
\chapter{Dual numbers and split-complex numbers}
\section{minkowski and galilaean spacetimes}
\bibliographystyle{amsplain}
\bibliography{papers}
% \printindex
\end{document}
% Local Variables:
% fill-column: 144
% latex-block-names: ("lemma" "theorem" "remark" "definition" "corollary" "fact" "properties" "conjecture" "proof" "question" "proposition")
% TeX-master: t
% End:
